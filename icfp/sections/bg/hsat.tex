\subsection{Hyperproperties and (Robust) Hyper-Satisfaction}\label{subsec:bg:hprop}

The formal language of trace-based properties is incapable of expressing certain security-relevant restrictions on program behaviours.
For example, consider {\em noninterference}~\cite{}, which demands that high-security inputs have no influence on low-security outputs.
To describe this property, it is necessary to refer to two possibly distinct traces.
It is possible to describe such properties as so-called hyperproperties~\cite{clarkson2008hyper}.

A hyperproperty ($\varHProp$) is a set of program behaviours, i.e., a set of sets of traces, which have been referred to as {\em class} (see \Cref{subsec:bg:tprop}).
Much like trace-based properties, hyperproperties can also be classified into different categories, e.g., hypersafety or hyperliveness~\cite{}.

We extend \Cref{def:propsat,def:proprsat} as follows:

%TODO: this is wrong. rework to explain it via program behaviours instead
\begin{definition}[Hyper-Satisfaction]\label{def:prophsat}
  $\sat{\wholeprogvar}{\varHProp}$
  $\isdef$
  $\forall\varProp\in\varHProp,\sat{\wholeprogvar}{\varProp}$.
\end{definition}
\begin{definition}[Robust Hyper-Satisfaction]\label{def:proprhsat}
  $\rsat{\progvar}{\varHProp}$
  $\isdef$ 
  $\forall\varProp\in\varHProp,\rsat{\progvar}{\varProp}$.
\end{definition}

The definitions in \Cref{subsec:bg:rtp} are trivially extended to the hyperproperty case, so we omit it.
However, it is worth noting that program refinement does not commute with hyperproperty refinement, while it does so for traces.
% TODO: write more
Consider 
\[
\begin{array}{rcl}
  \operatorname{behav}\left(\src{p}\right) &=& \left\{ \src{\event_0} \right\} \\
  \operatorname{behav}\left(\src{p'}\right) &=& \left\{ \src{\event_0}, \src{\event_1} \right\} \\[.33cm]
  \varHProp &=& \left\{ \left\{ \src{\event_0} \right\}, \left\{ \src{\event_1} \right\} \right\} \\
  \varHProp' &=& \left\{ \left\{ \src{\event_1} \right\} \right\} \\
\end{array}
\]
Clearly, $\operatorname{behav}(\src{p})\subseteq\operatorname{behav}(\src{p'})$, so $\src{p}$ {\em refines} $\src{p'}$.
Similarly, $\varHProp'\subseteq\varHProp$.
Note that $\sat{\src{p}}{\varHProp}$, but $\not\sat{\src{p}}{\varHProp'}$.
This does not break for ordinary trace-based properties, since refinement coincides with satisfaction.


Based off of \Cref{def:prophsat,def:propsat}, it is well known that an ordinary trace-based property can be lifted to an equivalent hyperproperty by means of simply taking its powerset~\cite{clarkson2008hyper}.

