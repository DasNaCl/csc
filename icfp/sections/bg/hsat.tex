\subsection{Hyperproperties and (Robust) Hyper-Satisfaction}\label{subsec:bg:hprop}

The formal language of trace-based properties is incapable of expressing certain security-relevant restrictions on program behaviours.
For example, consider {\em noninterference}~\cite{}, which demands that high-security inputs have no influence on low-security outputs.
To describe this property, it is necessary to refer to two possibly distinct traces.
It is possible to describe such properties as so-called hyperproperties~\cite{clarkson2008hyper}.

A hyperproperty ($\varHProp$) is a set of program behaviours, i.e., a set of sets of traces, which have been referred to as {\em class} (see \Cref{subsec:bg:tprop}).
Much like trace-based properties, hyperproperties can also be classified into different categories, e.g., hypersafety or hyperliveness~\cite{clarkson2008hyper}.
Based off of \Cref{def:prophsat,def:propsat}, it is well known that an ordinary trace-based property can be lifted to an equivalent hyperproperty by means of simply taking its powerset~\cite{clarkson2008hyper}.

We extend \Cref{def:propsat,def:proprsat} as follows:

\begin{definition}[Hyper-Satisfaction]\label{def:prophsat}
  $\sat{\wholeprogvar}{\varHProp}$
  $\isdef$
  $\behav{\wholeprogvar}\in\varHProp$.
\end{definition}
\begin{definition}[Robust Hyper-Satisfaction]\label{def:proprhsat}
  $\rsat{\progvar}{\varHProp}$
  $\isdef$ 
  $\forall\contextvar\ \wholeprogvar,\text{if}\ \link{\contextvar}{\progvar}=\wholeprogvar, \text{then}\sat{\wholeprogvar}{\varHProp}$.
\end{definition}

The definitions in \Cref{subsec:bg:rtp} are trivially extended to the hyperproperty case, so we omit them.
However, it is worth noting that program refinement does not commute with hyper-satisfaction, while it does commute for ordinary satisfaction.
To see this, consider the following example:
 %\[
 %\begin{array}{c}
 %  \operatorname{behav}\left(\src{\wholeprogvar}\right) = \left\{ \src{\trace[_0]} \right\} \quad
 %  \operatorname{behav}\left(\src{\wholeprogvar'}\right) = \left\{ \src{\trace[_0]}, \src{\trace[_1]} \right\} \qquad
 %  \varHProp = \left\{ \left\{ \src{\trace[_0]} \right\}, \left\{ \src{\trace[_1]} \right\} \right\} \quad
 %  \varHProp' = \left\{ \left\{ \src{\trace[_1]} \right\} \right\} \\
 %\end{array}
 %\]
\begin{center}
\begin{tikzpicture}
  \node (a) {$\behav{\src{\wholeprogvar}}=\left\{\src{\trace[_0]}\right\}$};
  \node[right=2cm of a] (b) {$\behav{\src{\wholeprogvar[']}}=\left\{\src{\trace[_0]},\src{\trace[_1]}\right\}$};
  \node[below=1cm of b] (d) {$\varHProp=\left\{\left\{\src{\trace[_0]},\src{\trace[_1]}\right\},\left\{\src{\trace[_1]}\right\}\right\}$};

  \node[right=.8cm of a] (ab) {$\subseteq$};
  \node[below=.5cm of b,rotate=270] (bd) {$\in$};
  \node[below=.3cm of ab,rotate=-33] (ad) {$\not\in$};
\end{tikzpicture}
\end{center}

Clearly, $\operatorname{behav}(\src{\wholeprogvar})\subseteq\operatorname{behav}(\src{\wholeprogvar'})$, so $\src{\wholeprogvar}$ {\em refines} $\src{\wholeprogvar'}$.
However, while $\src{\wholeprogvar[']}$ satisfies $\varHProp$ ($\sat{\src{\wholeprogvar[']}}{\varHProp}$), we have that $\not\sat{\src{\wholeprogvar}}{\varHProp}$.
This does not break for ordinary trace-based properties, since refinement coincides with satisfaction, i.e., taking subsets is a transitive operation.



