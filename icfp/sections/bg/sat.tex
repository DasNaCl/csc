\subsection{Trace Properties and (Robust) Satisfaction}\label{subsec:bg:tprop}

Running a (whole) program yields a trace ($\trace$), i.e., a potentially infinite sequence\footnote{
Throughout the paper, sequences are indicated with an overbar (i.e., $\trace$), empty sequences with $\hole{\cdot}$, and concatenation of sequences $\trace[_{1}],\trace[_{2}]$ as $\trace[_{1}]\cdot\trace[_{2}]$.
Prepending element $\event$ to a sequence $\trace$ uses the same notation: $\event\cdot\trace$.
} of events ($\event$).
Events include security-relevant actions (e.g., reading from and writing to memory, as detailed in \Cref{sec:compprop}), the unobservable event ($\emptyevent$), and a crash ($\lightning$).
For a given program, we refer to the set of traces as its behaviour. 
Moreover, sets of traces are called properties ($\varProp$) which describe admissible program behaviours. % ascribing what said property considers valid.
% 
The set of all properties can be split-up into different {\em classes} ($\class$), i.e., safety, liveness, and neither safety nor liveness~\cite{clarkson2008hyper}, so a class is simply a set of properties.
% 
%The compositionality framework (\Cref{sec:sequential}) presented in this paper considers arbitrary classes, while the case-study (\Cref{sec:casestud:rtp}) fixes them to concrete instances of safety properties, since it is decidable whether a trace satisfies a safety property with just a finite trace (i.e., a \emph{prefix}).

As an example, consider the trace $\ev{Dealloc\ \loc}\cdot\ev{Read\ \loc\ 1729}\cdot\dots$ which describes the interaction with memory where the deallocation of an address ($\loc$) precedes a read (of some value $1729$) at that address in memory.
% 
This behaviour is insecure w.r.t. the canonical notion of \gls*{tms} dictating no use-after-frees~\cite{nagarakatte2010cets,azevedo2018meaningsofms}, because it reads from a memory location that was freed already.
The first two events in the shown trace are enough to decide that the trace does not satisfy \gls*{tms}, since there is no way to append events to this prefix which would result in the trace being admissible for \gls*{tms}.

%In the following, the execution of a whole program $\wholeprogvar$ that terminates in state $\runtimetermvar$ according to the language semantics and produces trace $\trace$ is written as $\progstepto{\wholeprogvar}{\runtimetermvar}{\trace}$.
%With this, we defined property satisfaction as follows:
\bul{A whole program $\wholeprogvar$ satisfies a property $\varProp$} iff \iul{$\wholeprogvar$ yields a trace $\trace$} such that \oul{$\trace$ satisfies $\varProp$}.

\begin{definition}[Property Satisfaction]\label{def:propsat}
  \bul{$\sat{\wholeprogvar}{\varProp}$}
  $\isdef$
  if \iul{$\forall\trace,\tracegen{\wholeprogvar}{\trace}$},
  then \oul{$\trace\in\varProp$}.
\end{definition}

Property satisfaction is defined on whole programs, i.e., programs without missing definitions.
Thus, from a security perspective, \Cref{def:propsat} considers only a passive attacker model, where the attacker observes the execution and, e.g., retrieves secrets from that.
To consider a stronger model similarly to what existing work does~\cite{abate2019jour,abate2021extacc,maffeis2008code-carrying,gordon2003authenticity,fournet2007authorization,bengtson2011refine,backes2014uniontyps,michael2023mswasm,swasey2017robust,sammler2019benefits}, we extend the concept of satisfaction with {\em robustness}.
Robust satisfaction considers partial programs $\progvar$, i.e., components with missing imports, which cannot run until said imports are fulfilled.
To fill those gaps, {\em linking} takes two partial programs $\progvar[_{1}],\progvar[_{2}]$ and produces a whole program $\wholeprogvar$, i.e., $\link{\progvar[_{1}]}{\progvar[_{2}]}=\wholeprogvar$.

As typically done in works that consider the execution of partial programs~\cite{abate2019jour,devriese2018parametricity,patrignani2021rsc,korashy2021capableptrs,strydonck2019lincap,devriese2017modular,bowman2015noninterference,ahmed2011equivcps,patterson2017linkingtyps},
this paper assumes that whole programs are the result of linking partial programs referred to as {\em context} ($\ctx$) and {\em component} ($\comp$).
The context is an arbitrary program and thus has the role of an {\em attacker} that can interact with the component by means of any features the programming language has, while the component is what is security-relevant.
%In this work, the semantics of the programming language is expected to differentiate between the component and the context.
With this, \Thmref{def:propsat} can be extended as follows: for \bul{a component $\progvar$ to robustly satisfy a property $\varProp$}, take an \iul{attacker context $\contextvar$ and link it with $\progvar$} such that \oul{the resulting whole program must satisfy $\varProp$}.

\begin{definition}[Robust Satisfaction]\label{def:proprsat}
  \bul{$\rsat{\progvar}{\varProp}$}
  $\isdef$ \iul{$\forall \contextvar\ \wholeprogvar$, if $\link{\contextvar}{\progvar}=\wholeprogvar$}, then \oul{$\sat{\wholeprogvar}{\varProp}$}.
\end{definition}

% \begin{exampleenv}[Double Free in Bluetooth Subsystem]
%   Consider \texttt{CVE-2021-3564}~\cite{doublefree-bluetooth}, one of many submissions for a double-free vulnerability.
%   The vulnerability arises due to a race condition where the context-level function \texttt{hci\_cmd\_work} was not expected to behave maliciously, since it resides in the same source-code repository where the vulnerability occurs.
%   Nevertheless, the component-level code of \texttt{hci\_dev\_do\_open} is linked with \texttt{hci\_cmd\_work} and does not atomically check whether a pointer has been freed already:
%   Therefore, \texttt{hci\_dev\_do\_open} does not satisfy the no-double-frees property robustly, since there is an implementation for \texttt{hci\_cmd\_work} that leads to a violation of that property when linked with \texttt{hci\_dev\_do\_open}.
% \end{exampleenv}
