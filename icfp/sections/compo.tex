\section{Secure Composition}\label{sec:sequential}

Notably, real-world compilation chains also perform a series of (sequential) passes whose main purpose is not necessarily to translate from one language to another, but to, e.g., optimise the code or enforce a certain property.
Both examples can be seen in practice, e.g.,~\cite{nagarakatte2009soft,nagarakatte2010cets,akritidis2009baggy,wegman1991ccp,manjikian1997fusion} and many more.
% \MPin{survey to cite? this is a bit informal MK: hard to find a survey on different compiler passes... I can't even find one just for compiler enforced memsafety (that looks good) or for compiler translations in general}.
% 
% Consider the compilation passes of LLVM~\cite{lattner2004llvm}, which are handled in the \texttt{PassManager} class.
% By manipulating the internal state of a \texttt{PassManager} object, the schedule of compiler optimisation passes can be hardcoded.
% That way, an order of optimisations can be found by hand that yields the best results.
% \MP{what's the point of the previous 3 sentences?(commented)}
Consider the following two LLVM optimisation passes: \gls*{cf}, which rewrites constant expressions to the constant they evaluate to, and \gls*{dce}, which removes dead code by rewriting conditional branches.
The order in which \gls*{cf} and \gls*{dce} are performed influences the final result of the compilation (see \Cref{fig:cfdceex}).
\begin{figure}[!ht]
  \begin{center}
    \begin{tikzpicture}[scale=0.62, every node/.style={transform shape}]
      \node[minimum width=3cm,minimum height=2.5cm,draw=black,very thick,align=left] (progunopt) {\begin{lstlisting}[language=ml,basicstyle=\small\ttfamily]
let a = true in
if a then
  print "a"
else
  print "b"
\end{lstlisting}};
      \node[minimum width=3cm,minimum height=2.5cm,draw=black,very thick,align=left,right=1.0 of progunopt] (progoptcf) {\begin{lstlisting}[language=ml,basicstyle=\large\ttfamily]
if true then
  print "a"
else
  print "b"
\end{lstlisting}};
      \node[minimum width=3cm,minimum height=2.5cm,draw=black,very thick,align=left,right=1.0 of progoptcf] (progoptcfdce) {\begin{lstlisting}[language=ml,basicstyle=\Large\ttfamily]
print "a"
\end{lstlisting}};% the following is a hack to get my syntax highlighting to work again \end{lstlisting}
      % edges
      \draw[->,very thick] (progunopt) edge[loop left,left] node {\gls*{dce}} (progunopt);
      \draw[->,very thick] (progunopt) edge[sloped,above] node {\gls*{cf}} (progoptcf);
      \draw[->,very thick] (progoptcf) edge[sloped,above] node {\gls*{dce}} (progoptcfdce);
    \end{tikzpicture}
  \end{center}
  \caption{Example program where the level of optimisations differ for one pass of applying \gls*{cf} and \gls*{dce} in any order. %
  Every edge is a compilation pass and the label on the edge states what the pass does, i.e., \gls*{cf} or \gls*{dce}. %
  The source code in the nodes is a glorified compiler intermediate representation and the code gets more optimised towards the right hand side of the figure.}\label{fig:cfdceex}
\end{figure}
This {\em phase ordering problem} is well--known in literature and a practical solution is to simply perform a fixpoint iteration of the optimisation pipeline~\cite{click1995combining}.

\smallskip

To analyse the security of compilation passes and their interaction within a compilation pipeline, we rely on a few key notions: the definition of a trace relation, and the definition of when is a trace relation well-formed with respect to a class.
%
Consider traces $\src{\trace}$ and $\trg{\trace}$ as well as two trace relations $\sim_1$ and $\sim_2$. 
% 
The traces are related $\src{\trace}\sim_1\bullet\sim_2\trg{\trace}$ if there is another trace $\irl{\trace}$ such that $\src{\trace}\sim_1\irl{\trace}$ and $\irl{\trace}\sim_2\trg{\trace}$.
% 
\bul{A relation $\sim$ is well-formed w.r.t. a class of target-level properties $\trg{\class}$} iff \rul{the universal image preserves set membership}.
\begin{definition}[Well-formedness of $\sim$ for a Class $\trg{\class}$]\label{def:wfc:sig:tracerel}
  % $$

  \begin{nscenter}
  \noindent
  \text{\bul{$\wfcsig{\sim}{\trg{\class}}$}} \isdef \text{\rul{$\forall \trg{\pi}\in\trg{\class}, \sigma_\sim(\trg{\pi})\in\sigma_\sim(\trg{\class})$}}
  \end{nscenter}
  % $$
\end{definition}

We can now state our main result: secure compilers in the robust compilation framework~\cite{abate2021extacc} compose {\em sequentially}. 
% 




Let \bul{$\cc{\src{L}}{\trg{L}}$ robustly preserve the class $\sigma_{\sim_2}\left(\obj{\class[_1]}\right)$ under $\sim_1$} and let \rul{$\cc{\trg{L}}{\obj{L}}$ robustly preserve the class $\obj{\class[_2]}$ under $\sim_2$}.
Then, \iul{when the cross-language relation $\sim_2$ is well-formed w.r.t. class $\obj{\class[_1]}$}, it follows that \oul{the composed compiler\footnote{Here, $\cc{\src{L}}{\trg{L}}\circ\cc{\trg{L}}{\obj{L}}$ means first running $\cc{\src{L}}{\trg{L}}$ and then $\cc{\trg{L}}{\obj{L}}$.} $\cc{\src{L}}{\trg{L}}\circ\cc{\trg{L}}{\obj{L}}$ robustly preserves the intersection of classes $\obj{\class[_1]}\cap\obj{\class[_2]}$ under $\sim_1\bullet\sim_2$}.
% 
%For space reasons, \Thmref{thm:urtp} and \Thmref{thm:lrtp} are not presented here, but can be found in the technical appendix ($\CoqSymbol$).
% Note that it is not necessary to require $\sim_2$ to be well-formed w.r.t. $\src{\class[_1]}$, since the 
% \MPin{why}
\begin{theorem}[Composition of Secure Compilers w.r.t. $\sigma$]\label{thm:rtpsim:sig}
  $\;$ 

  If \bul{$\rtpsig{\cc{\src{L}}{\trg{L}}}{\tilde{\sigma}_{\sim_2}\left(\obj{\class[_{1}]}\right)}{\sim_1}$}, \rul{$\rtpsig{\cc{\trg{L}}{\obj{L}}}{\obj{\class[_2]}}{\sim_2}$}, and \iul{$\wfcsig{\sim_2}{\obj{\class[_1]}}$}, \\ then \oul{$\rtpsig{\cc{\src{L}}{\trg{L}}\circ\cc{\trg{L}}{\obj{L}}}{\obj{\class[_{1}]}\cap\obj{\class[_{2}]}}{\sim_1\bullet\sim_2}$}. \Coqed
\end{theorem}

Since the composition of secure compilers is again a secure compiler, the theorems generalise to a whole chain of $n$ secure compilers. 
% 
\Cref{thm:rtpsim:sig} can also be stated for the existential image $\tau_\sim\left(\src{\pi}\right)$, but in the interest of space that result has been moved to the appendix.
 % (\Cref{subsec:extimg}).
% 
Crucially, \Cref{thm:rtpsim:sig} also holds for \emph{classes of hyperproperties}, and thus, compilers that robustly preserve hyperproperties can be composed with each other as well as with compilers that robustly preserve properties.
% 
%Unfortunately, proofs of compilers that robustly preserve hyperproperties (including hypersafety) are scarce in the literature due to their complexity.
%\MP{ so what ? }
%  we don't discuss this further?
% 

% It is furthermore possible to conflate the notion of classes and hyperproperties by intersecting all elements of a class of hyperproperties, to obtain one combined hyperproperty.
% This idea has been discussed in other work~\cite{clarkson2008hyper}.
%  MP: i think i stated the point of these 2 lines above

If we take SecurePtrs and FaCT from \Cref{subsec:bg:rtp} and compose them according to \Cref{thm:rtpsim:sig}, we obtain a compiler that robustly preserves the intersection of safety properties and the \gls*{cct} hyperproperty. 
That is, for a source component that robustly satisfies any set of safety properties {\em and} \gls*{cct}, the compiled target component also robustly satisfies the same set of safety properties and \gls*{cct}.
% The intersection simply uses the lifted version of the respective safety properties.
%\MP{meaning?}

Compiler engineers typically try to find an order of optimisations that yields well-optimised programs for either code size~\cite{cooper1999geneticphases} or performance~\cite{kulkarni2006exhaustivephase}.
\Cref{corr:swappable} justifies that any such order of compilation passes is valid w.r.t. security, as long as the trace-relations have no effect on the respective classes.

So, given two compilation passes \bul{$\cc[_{1}]{\trg{L}}{\trg{L}}$, $\cc[_{2}]{\trg{L}}{\trg{L}}$, both robustly preserving class $\trg{\class[_{1}]}$ or $\trg{\class[_{2}]}$, respectively}, \rul{their corresponding well-formed trace-relations}, \iul{and indifference of the classes with respect to these trace relations}, \oul{for any order of their composition, the composed compiler robustly preserves the intersection of $\trg{\class[_{1}]}$ and $\trg{\class[_{2}]}$}.
% 
% \MPin{
% 	later on, say that optimisation passes remove elements from traces, so they should be fine w.r.t. the result with $\tau$.
% 	also, yes, if $\tau$ adds we have a problem, if it removes we don't
% \MKin{ Optimisations that remove elements from traces are not necessarily sound,
%  e.g., remove all deallocations to "optimise" memory management by offloading
%  the work to the OS.
% }
% }

\begin{corollary}[Swapping Secure Compiler Passes]\label{corr:swappable}
  $\;$ 

  If \bul{$\rtpsig{\cc[_{1}]{\trg{L}}{\trg{L}}}{\trg{\class[_{1}]}}{\sim_1}$ and $\rtpsig{\cc[_{2}]{\trg{L}}{\trg{L}}}{\trg{\class[_{2}]}}{\sim_2}$}, %
  \rul{$\wfcsig{\sim_1}{\trg{\class[_2]}}$ and $\wfcsig{\sim_2}{\trg{\class[_1]}}$}, %
  and \iul{$\tilde{\sigma}_{\sim_2}\left(\trg{\class[_1]}\right)=\trg{\class[_1]}$ as well as $\tilde{\sigma}_{\sim_1}\left(\trg{\class[_2]}\right)=\trg{\class[_2]}$},
  then \oul{$\rtpsig{\cc[_{1}]{\trg{L}}{\trg{L}}\circ\cc[_{2}]{\trg{L}}{\trg{L}}}{\trg{\class[_{1}]}\cap\trg{\class[_{2}]}}{\sim_1\circ\sim_2}$ and $\rtpsig{\cc[_{2}]{\trg{L}}{\trg{L}}\circ\cc[_{1}]{\trg{L}}{\trg{L}}}{\trg{\class[_{2}]}\cap\trg{\class[_{1}]}}{\sim_2\circ\sim_1}$}. \Coqed
\end{corollary}

Coming back to the example composing SecurePtrs with FaCT, it is likely the case that \Cref{corr:swappable} is not applicable.
While first running SecurePtrs and then FaCT should be fine, the other direction has potential security hazards, since the SecurePtrs compiler does not account for cryptographic-constant time primitives, such as \texttt{ctselect}.

\subsection{Secure Compiler Composition with Same Trace Models}
When the cross-language trace relation is an equality, \Cref{thm:rtpsim:sig} collapses:
Given \bul{$\cc{\src{L}}{\trg{L}}$ robustly preserves $\class[_{1}]$} and \rul{$\cc{\trg{L}}{\obj{L}}$ robustly preserves $\class[_{2}]$}, it follows that \oul{their sequential composition $\cc{\src{L}}{\trg{L}}\circ\cc{\trg{L}}{\obj{L}}$ robustly preserves the intersection of classes $\class[_{1}]$ and $\class[_{2}]$}.

\begin{corollary}[Composition of Secure Compilers]\label{corr:rtp}
  $\;$ 

  If \bul{$\rtp{\cc{\src{L}}{\trg{L}}}{\class[_{1}]}$} and \rul{$\rtp{\cc{\trg{L}}{\obj{L}}}{\class[_{2}]}$}, then \oul{$\rtp{\cc{\src{L}}{\trg{L}}\circ\cc{\trg{L}}{\obj{L}}}{\class[_{1}]\cap\class[_{2}]}$}. \Coqed
\end{corollary}

\Cref{corr:rtp} provides an easy way to compose secure compilers without well-formedness of trace relations. 
% 
However, while \Cref{thm:rtpsim:sig} explicitly requires that $\sim_1$ is well-formed w.r.t. $\trg{\class[_2]}$, if $\sim_2$ is not well-formed w.r.t. $\trg{\class[_1]}$, care must be taken. 
%Worst-case is that the resulting class $\sigma_{\sim_1\bullet\sim_2}\left(\trg{\class[_1]}\cap\trg{\class[_2]}\right)$ becomes empty, even though $\trg{\class[_1]}\cap\trg{\class[_2]}$ is not necessarily empty.
This is further discussed in \Cref{subsec:compatsecpasses}.

%If possible, optimisations are merged to enhance the analysis' result, since even when iterating to a fixpoint the final program may not be as optimised as it could be~\cite{click1995combining}.
We can also obtain a specialised version of \Cref{corr:swappable}:
%So, given two compilation passes \bul{$\cc[_{1}]{\trg{L}}{\trg{L}}$, $\cc[_{2}]{\trg{L}}{\trg{L}}$, both robustly preserving class $\class[_{1}]$ or $\class[_{2}]$, respectively}, \oul{for any order of their composition, the composed compiler robustly preserves the intersection of $\class[_{1}]$ and $\class[_{2}]$}.
% 
% \MPin{
% 	later on, say that optimisation passes remove elements from traces, so they should be fine w.r.t. the result with $\tau$.
% 	also, yes, if $\tau$ adds we have a problem, if it removes we don't
% \MKin{ Optimisations that remove elements from traces are not necessarily sound,
%  e.g., remove all deallocations to "optimise" memory management by offloading
%  the work to the OS.
% }
% }

\begin{corollary}[Swapping Secure Compiler Passes]\label{corr:swappable:one}
  $\;$ 

  If {$\rtp{\cc[_{1}]{\trg{L}}{\trg{L}}}{\class[_{1}]}$ and $\rtp{\cc[_{2}]{\trg{L}}{\trg{L}}}{\class[_{2}]}$}, then {$\rtp{\cc[_{1}]{\trg{L}}{\trg{L}}\circ\cc[_{2}]{\trg{L}}{\trg{L}}}{\class[_{1}]\cap\class[_{2}]}$ and $\rtp{\cc[_{2}]{\trg{L}}{\trg{L}}\circ\cc[_{1}]{\trg{L}}{\trg{L}}}{\class[_{2}]\cap\class[_{1}]}$}. \Coqed
\end{corollary}
