\section{Conclusion\pages{$\sfrac{1}{2}$}}\label{sec:concl}
This paper tackles the problem of understanding what kind of security properties a secure compiler preserves, when said compiler is the combination of compiler passes that preserve possibly different security properties.
% 
%For this, this paper first formalised security properties of interest and their composition.
% 
The paper proves that composing secure compilers that preserve certain properties results in a secure compiler that preserves the composition of these properties.
% While the presented security property composition relied on monitors that check only trace-properties, the composition of secure compilers does not make any restriction towards the kind of property involved in the composition. 
% It is subject to future work to develop techniques to verify relational hyperproperties~\cite{abate2019jour}, while the composition of hyperproperties could be very similar to the composition of ordinary properties as presented in this paper, since hyperproperties can be checked with an automata-based model-checker~\cite{beutner23hyperltl}.
% 
Finally, this paper defines a multi-pass compiler and proves that it preserves \gls*{specms}.
Crucially, this paper derives the security of the multi-pass compiler from the composition of the security properties preserved by its individual passes, which include security-preserving as well as optimisation passes.
% For future work, it is interesting to look at more sophisticated optimisation passes that, e.g., reorder events that appear on traces.
