\section{Case Study: Language Formalisations\pages{2}}\label{sec:casestud:defs}

This section defines programming languages $\Ltms$, $\Ltrg$, $\Lms$, $\Lscct$, and $\Lspec$, all of which share common elements (presented in \Cref{subsec:cs:defs}).
$\Ltms$ is the only statically-typed language, and it exhibits the property that all well-typed programs are \gls*{tms} (\Cref{subsec:ltms}).
However, not all $\Ltms$ programs are \gls*{sms}.
That is, there are well-typed $\Ltms$ programs that perform an out-of-bounds access.
Language $\Ltrg$ is untyped and does not provide any guarantees with regards to \gls*{ms} (\Cref{subsec:lsms}).
$\Lms$ is exactly the same language as $\Ltrg$, but this paper still distinguishes the two for sake of readability (\Cref{subsec:lms}).
All three languages --- so $\Ltms$, $\Ltrg$, and $\Lms$ --- assume \gls*{scct} to hold, since this is -- in an ideal world -- what the programmer would expect, too: it is the job of the compiler to preserve and (potentially) enforce \gls*{scct} security~\cite{cauligi2019fact,nagarakatte2010cets,nagarakatte2009soft,akritidis2009baggy}.
% This paper considers \gls*{cct} as an architecture-dependent property\MPin{can we justify this}, so

Such consideration is also backed up by architecture providing a data (operand) independent timing mode, such as processors by Arm~\cite[p.~543]{arm-refman} and Intel~\cite[p.~80]{intel-refman}.
%In spirit of this, all languages besides $\Lscct$ are trivially \gls*{scct}. 
This kind of processor feature is modelled in language $\Lscct$ (\Cref{subsec:lscct}), where programs have access to a ``\gls*{cct}-mode'' and can change the leakage of emitted events according to the value of this mode (either $\obj{ON}$ or $\obj{OFF}$). 

Finally, modern processors also employ speculative execution to achieve speedups---and unfortunately generate Spectre attacks~\cite{kocher2019spectre}---and this is the extension of $\Lspec$ (\Cref{subsec:lspec}).
 % extends $\Lscct$ to model speculative execution.
%
Thus, all previous languages trivially satisfy \gls*{ss}, since they do not support speculative execution at all.
% All languages share large parts of syntactic elements, which are presented in the next subsection (\Cref{subsec:cs:defs}).
% Subsequent subsections extend on this accordingly.

\subsection{Shared Language Definitions}\label{subsec:cs:defs}

All presented programming languages share a common fragment which is partially presented here and in full detail in the technical report. 
%\MP{put omitted stuff in appendix + ref there (and add textual explanation)}

% \vspace{-.5em}
{
  \renewcommand{\src}[1]{\mathsf{#1}}
 \small
\[
  \begin{array}{rrcl}
    \text{(Base-Events)} & \src{\preevent} &:=& \src{\ev{Alloc \loc\ n}} \mid \src{\ev{Dealloc \loc}} \mid \src{\ev{Get \loc\ n}} \mid \src{\ev{Set \loc\ n}} \\
    \text{(Control Tags)} & \src{\sandboxtag{}} &:=& \src{\ctx} \mid \src{\comp} \\
    \text{(Events)} & \src{\event} &:=& \src{\preevent;\sandboxtag{}} \mid \src{\emptyevent} \mid \src{\lightning} \\ 
    \text{(Values)} & \src{v} &:=& \src{n} \\
    \text{(Expressions)} & \src{\expr} &:=& \src{x} \mid \src{n} \mid \src{\lbinop{\expr[_1]}{\expr[_2]}} \mid \src{\lifz{\expr[_1]}{\expr[_2]}{\expr[_3]}} \\ 
                         &&&\mid \src{\llet{x}{\expr[_1]}{\expr[_2]}} \mid \src{\llet{x}{new\ \expr[_1]}{\expr[_2]}} \\
                         &&&\mid \src{\ldelete{x}} \mid \src{\lget{x}{\expr}} \mid \src{\lset{x}{\expr[_1]}{\expr[]}} \mid \src{stuck}\\
                         &&&\mid \src{\lcall{f}{\expr}} \mid \src{\lreturn{\expr}} \\
    \text{(Functions)} & \src{F} &:=& (\src{x};\src{\expr}) \\
    \text{(Libraries)} &&&\hspace{-3.8em} \src{\library_{\ctx}},\src{\library_{\comp}} : \src{Vars} \to \src{F} 
    % \hspace{2em}
    %  : \src{Vars} \to \src{F} 
     \\
    \text{(Heaps)} & \src{H} &:=& \src{\hole{\cdot}} \mid \src{n},\src{H}
    % \hspace{0.5cm}    
    \\
    \text{(Pointer Maps)} & \src{\Delta} && \text{ omitted for simplicity}
    \\
    \text{(Memory States)} & \src{\memstate} &:=& \left(\src{H^{\ctx}};\src{H^{\comp}};\src{\Delta}\right)\\
    \text{(Control States)} & \src{\cfstate} & & \text{ omitted for simplicity} 
    \\
    \text{(States)} &\src{\Omega} &:=& \left(\src{\cfstate};\src{\sandboxtag{}};\src{\memstate}\right)\\
  \end{array}
\]
}
{
  \renewcommand{\src}[1]{\mathsf{#1}}
The trace models of all languages are similar to those presented earlier (\Cref{sec:compprop}).
One technical detail is the addition of a control tag, indicating who is to blame for an emitted action: context ($\ctx$) or component ($\comp$).
This is a standard technique in secure compilation in order to rule out irrelevant context-level events. 
For example, a context immediately deallocating an allocated memory region twice trivially violates memory safety, but the main interest in secure compilation is to preserve component-level security, and thus component events.
Even though this tagging could be used for blame preservation~\cite{patrignani2023blame}, this is beyond the focus of this paper.
Another key difference is that memory accesses are now explicitly modelled as reads ($\src{\ev{Read\ \loc\ n}}$) and writes ($\src{\ev{Write\ \loc\ n}}$) instead of just uses.

All languages have at least numbers as values ($\src{v}$) and second class functions ($\src{F}$).
Functions are modelled as pairs containing the name of one argument and the body of the function.
Bodies are just ordinary expressions, which can be simple binary operations ($\src{\lbinop{\expr[_1]}{\expr[_2]}}$), conditionals ($\src{\lifz{\expr[_1]}{\expr[_2]}{\expr[_3]}}$), function calls ($\src{\lcall{f}{e}}$) and returns ($\src{\lreturn{\expr}}$), as well as C-like memory operations. 
Programs have sets of pre-determined functions called libraries and they are marked as being part of some component ($\src{\library_{\comp}}$) or context ($\src{\library_{\ctx}}$).

For the operational semantics, the runtime state ($\src{\Omega}$) is a triple consisting of a control-flow state ($\src{\cfstate}$), a control tag ($\src{\sandboxtag{}}$), and a memory state ($\src{\memstate}$). 
The latter carries information about pointers that are kept ``alive'' in pointer maps ($\src{\Delta}$), so that the semantics does not get stuck when encountering, e.g., a double-free.
The memory state also carries two heaps to model sandboxing between a context ($\src{H^{\ctx}}$) and a component ($\src{H^{\comp}}$).

}

\subsection{$\src{L_{\tmssafe}}$: A Temporal but Not Spatial Memory Safe Language}\label{subsec:ltms}

$\src{L_{\tmssafe}}$ is the only statically-typed language in this case study and restricts functions ($\src{F}$) to the typing signature $\src{\natt\to\natt}$. 
The type system of $\src{L_{\tmssafe}}$ is inspired by $L^{3}$~\cite{morrisett2005L3,scherer2018fabulous} and enforces that every well-typed $\src{L_{\tmssafe}}$ program satisfies \gls*{tms}.
 % (\Cref{thm:wt:tms}).

\begin{theorem}[$\src{L_{\tmssafe}}$-programs are \gls*{tms}]\label{thm:wt:tms}
$\rsat{\src{\library_{\comp}}}{\tmssafe}$ \Coqed
  % $\;$ 
% 
  % \begin{nscenter}
  %   \hfill For any component $\src{\library_{\comp}}$, $\rsat{\src{\library_{\comp}}}{\tmssafe}$ \Coqed
  % \end{nscenter}
\end{theorem}

\subsection{$\Ltrg$: A Memory-Unsafe Language}\label{subsec:lsms}

$\Ltrg$ extends the syntax presented earlier (\Cref{subsec:cs:defs}) with dynamic typechecks ($\trg{\lhast{\expr}{\type}}$), evaluating to $\trg{0}$ if the type matches with the shape of $\trg{\expr}$ and $\trg{1}$ otherwise.
%The pointers carry poison tags, which signify a pointer being allocated ($\trg{\poisoned}$) or freed ($\trg{\poisonless}$). 
Furthermore, the syntax of $\Ltrg$ is extended with a way to inspect whether a pointer is freed ($\trg{\lispoisoned{x}}$), evaluating to $\trg{0}$ if it is freed and to $\trg{1}$ otherwise.
Functions may receive arguments that are not $\trg{\natt}$, values are extended with tuples, and expressions are also extended with pair projections.

% \vspace{-1em}
\[
  \begin{array}{rrcl}
    \text{(Values)} & \trg{v} &:=& \dots \mid \trg{\lpair{v_1}{v_2}} \\
    \text{(Expressions)} & \trg{\expr} &:=& \dots \mid \trg{\lhast{\expr}{\tau}} \mid \trg{\lispoisoned{x}} \mid \trg{e.1} \mid \trg{e.2}
  \end{array}
\]

No changes are done to the trace model, but $\Ltrg$ has no static typing, making double-free code patterns possible.

\subsection{$\Lms$: Another Memory-Unsafe Language}\label{subsec:lms}
$\Lms$ is exactly equal to $\Ltrg$ (\Cref{subsec:lsms}), but used to emphasize that this is code after applying $\cc{\Ltrg}{\Lms}$ (\Cref{subsec:cs:ms}).

\subsection{$\Lscct$: A Memory-Unsafe Language with a Constant Time Mode}\label{subsec:lscct}

$\Lscct$ extends $\Lms$ (\Cref{subsec:lms}) with a ,,constant-time mode''. 
The activation of the mode can be checked ($\obj{\lrddoit{x}{\expr}}$), changed ($\obj{\lwrdoit{D}}$), and is stored in program states ($\obj{\Omega}$).
At the beginning of program execution of $\Lscct$ programs, the mode is turned off ($\obj{OFF}$).
If the mode is enabled (i.e., set to $\obj{ON}$), the intuition is that no secrets are leaked. 
This models the real-world\footnote{As present in Intel~\cite[p.80]{intel-refman} and ARM~\cite[p.~543]{arm-refman} processors.} data-independent timing mode as well as the result of compiling a program with FaCT~\cite{cauligi2019fact}.
For example, FaCT rewrite code that branches to use a constant-time selection primitive.
To not obfuscate our formalisation unnecessarily by duplicating syntax, we simply added the ,,constant-time mode'' to $\Lscct$.

% $\Lscct$ extends $\Lms$ (\Cref{subsec:lms}) with a way to read from ($\obj{\lrddoit{x}{\expr}}$) and to write to ($\obj{\lwrdoit{D}}$) a model-specific register that controls a constant-time mode. 
% This language is a combination of the effect of the FaCT compiler~\cite{cauligi2019fact} as well as a data-independent timing mode, a feature that is present in both Arm~\cite[p.~543]{arm-refman} and Intel~\cite[p.~80]{intel-refman} processors.
% \MP{ can we be more specific about the fact pass? }
% To this end, states ($\obj{\Omega}$) are extended with the value of the register ($\obj{D}$), which is initially set to be not active ($\obj{OFF}$).
% If the register is marked active ($\obj{ON}$), the intuition is that no secrets can appear on specification traces.
% If the register is marked inactive, secrets may appear on traces.
% The state of the register can be read and written to, e.g., \Cref{tr:e-wrdoit-off}.
% For the other languages seen earlier, the mode is intuitively always-on, i.e., the mode of execution always uses constant-time mode.

% \vspace{-.5em}
\[
  \begin{array}{rrcl}
    \text{(Mode Values)} & \obj{D} &:=& \obj{ON} \mid \obj{OFF} \\
    \text{(Security Tags)} & \obj{\securitytag{}} &:=& \obj{\lock} \mid \obj{\unlock} \\
    \text{(Base-Events)} & \obj{\preevent} &:=& \dots \mid \obj{iGet\ \loc\ v} \mid \obj{iSet\ \loc\ v} \\
                       &&& \mid\ \obj{Binop\ n} \mid \obj{Branch\ n} \\
    \text{(Events)} & \obj{\event} &:=& \obj{\preevent;\sandboxtag{};\securitytag{}} \mid \obj{\emptyevent} \mid \obj{\lightning} \\
    \text{(Expressions)} & \obj{\expr} &:=& \dots \mid \obj{\lrddoit{x}{\expr}} \mid \obj{\lwrdoit{D}} \\
                         &&&\mid\ \obj{\llet{x^{\securitytag{}}}{\expr[_1]}{\expr[_2]}}\\
    \text{(States)} & \obj{\Omega} & := & \left(\obj{\cfstate};\obj{\sandboxtag{}};\obj{D};\obj{\memstate}\right)\\
  \end{array}
\]
\begin{center}
\newcommand{\expreval}[5]{{#1}\triangleright\xspace {#2}\xrightarrow{#5}\ {#3}\triangleright\xspace {#4}\xspace}
\newcommand{\exprevalo}[5]{\expreval{\obj{#1}}{\obj{#2}}{\obj{#3}}{\obj{#4}}{\obj{#5}}}
  \typerule{$e-\obj{wrdoit}-\text{off}$}{
  }{
    \pstepto{\rtt{\obj{\cfstate;\sandboxtag{};D;\memstate}}{\obj{\lwrdoit{OFF}}}}{\rtt{\obj{\cfstate;\sandboxtag{};OFF;\memstate}}{\obj{0^{\securitytag{}}}}}{\obj{\emptyevent}}
  }{e-wrdoit-off}

  \typerule{$e-\obj{get}-\in-$noleak}{
    \obj{\memstate}=\obj{H^{\ctx};H^{\comp};\Delta_1,x\mapsto(\loc;\sandboxtag{};\poison),\Delta_2} &
    \obj{\loc}+\obj{n}\in\text{dom }\obj{H^{\sandboxtag{}}} \\
    \obj{\securitytag{''}}=\obj{\securitytag{}\sqcap\securitytag{'}} &
    \obj{\event} = \obj{{iGet}\ \loc\ n;\sandboxtag{};\securitytag{''}}
  }{
    \exprevalo{
    	\cfstate;\sandboxtag{'};ON;\memstate
    }{
      	x^{\securitytag[n^{\securitytag{'}}] }
    }{
    	\cfstate;\sandboxtag{'};ON;\memstate
    }{
    	(H^{\sandboxtag{}}(\loc+n))^{\securitytag{''}}
    }{
      \event
    }
  }{to-e-get-in-noleak}
\end{center}

The language also adds user annotations $\obj{\securitytag{}}$ for the secrecy of variables, which can be either private (high secrecy) $\obj{\lock}$ or public (low secrecy) $\obj{\unlock}$.
Security tags $\obj{\securitytag{}}$ are arranged in the usual secrecy lattice~\cite{zdancewicphd}, where $\obj{\unlock}\sqsubseteq\obj{\lock}$.
% $\obj{\lock}\sqsubseteq\obj{\securitytag{}}$ and $\obj{\securitytag{}}\sqsubseteq\obj{\unlock}$.

Memory accesses to secret data need to be present to reason about memory safety, even when execution is in constant-time mode, e.g, \Cref{tr:to-e-get-in-noleak}.
$\Lscct$ extends base-events with $\obj{\ev{iGet}\ \loc\ \valueexpr}$ and $\obj{\ev{iSet}\ \loc\ \valueexpr}$ (for data $\obj{i}$ndependent get and set) to prevent secrets from leaking but still enable reasoning about memory safety. 
Due to the technical setup, the rule needs to check if the access is in bounds ($\obj{\loc}+\obj{n}\in\obj{H^t}$) and update the secrecy tag ($\obj{\securitytag{''}}=\obj{\securitytag{}\sqcap\securitytag{'}}$) with the least upper bound of the tags, according to the aforementioned lattice.
The precise information carried by the event, e.g., location ($\obj{\loc}$) is taken from the pointer map, which carries information irrelevant to this rule ($\obj{\poison}$).

Base-events include $\obj{\ev{Branch}\ n}$ and $\obj{\ev{Binop}\ n}$ that are emitted when evaluating a branch or certain binary expressions, such as division, respectively, whenever the constant-time mode is inactive.
Events are extended with a security-tag ($\obj{\securitytag{}}$) to signal the secrecy of the involved data.

The evaluation steps are amended to propagate the security-tag annotations $\obj{\securitytag{}}$.
When the constant-time mode is inactive, base-events $\obj{\ev{Branch}\ n}$ and $\obj{\ev{Binop}\ n}$ are emitted for conditionals and binary operations, respectively.
Otherwise, just like in the semantics of the earlier languages, $\obj{\emptyevent}$ is emitted for binary and branching operations.
%To mark this mode as active or inactive, \Cref{tr:e-wrdoit} allows to access the register in $\obj{\statevar}$.


\Cref{ex:lscct} illustrates the differences between $\Lscct$ and other languages.

\begin{exampleenv}[$\Lscct$ with and without constant-time mode]\label{ex:lscct}
  Consider again \Cref{ex:strncpy}, with a context copying the string \texttt{Hello World}, where everything is marked with a high security tag: $\lock$.
  The top half of \Cref{fig:ex-cct} (titled $\Lscct$), describes the execution trace of the program, while the bottom side of the table (titled ``Specification''), describes the related specification trace (\Cref{subsec:msctss:tracemodel}).
  Read in parallel from top to bottom, the figure shows parts of the execution trace. 
  In each half, the left column (\textit{Active}) has constant-time mode $\obj{ON}$ and the right one (\textit{Inactive}) has it $\obj{OFF}$.
  
  When the constant-time mode is off, the execution yields events without constant-time leaks (\Cref{subsec:ltms,subsec:lsms,subsec:lms}).
  But, if it is turned on, then the branching event \colorbox{yellow}{does not fire} anymore and both reading and writing to memory is related to a specification trace with \colorbox{lightgray}{no exposed secrets}.
\end{exampleenv}
\begin{figure}[!htb]
	\vspace{-1em}\small
  $$
  % {
  \begin{array}{ccc}
    & \Lscct &        \\
    \text{Active} & \mid & \text{Inactive} \\\hline
    \obj{\ev{Alloc}\ \loc_{x}\ 12;\comp;\unlock} & \mid & \obj{\ev{Alloc}\ \loc_{x}\ 12;\comp;\unlock} \\
    \obj{\ev{Alloc}\ \loc_{y}\ 12;\comp;\unlock} & \mid & \obj{\ev{Alloc}\ \loc_{y}\ 12;\comp;\unlock} \\
    \obj{{\ev{iGet}}\ \loc_{x}\ 0;\comp;}\text{\colorbox{lightgray}{$\obj{\lock}$}} & \mid & \obj{\ev{Get}\ \loc_{x}\ 0;\comp;\lock} \\
    \text{\colorbox{yellow}{$\obj{\emptyevent}$}} & \mid & \obj{\ev{Branch}\ 0;\comp;\lock} \\
    \obj{{\ev{iGet}}\ \loc_{x}\ 0;\comp;}\text{\colorbox{lightgray}{$\obj{\lock}$}} & \mid & \obj{\ev{Get}\ \loc_{x}\ 0;\comp;\lock} \\
    \obj{{\ev{iSet}}\ \loc_{y}\ 0\ \mathtt{'H'};\comp;}\text{\colorbox{lightgray}{$\obj{\lock}$}} & \mid & \obj{\ev{Set}\ \loc_{y}\ 0\ \mathtt{'H'};\comp;\lock} \\
    \obj{{\ev{iGet}}\ \loc_{x}\ 1;\comp;}\text{\colorbox{lightgray}{$\obj{\lock}$}} & \mid & \obj{\ev{Get}\ \loc_{x}\ 1;\comp;\lock} \\
    \text{\colorbox{yellow}{$\obj{\emptyevent}$}} & \mid & \obj{\ev{Branch}\ 0;\comp;\lock} \\
    \vdots & \mid & \vdots \\
    \obj{{\ev{iGet}}\ \loc_{x}\ 12;\comp;}\text{\colorbox{lightgray}{$\obj{\lock}$}} & \mid & \obj{\ev{Get}\ \loc_{x}\ 12;\comp;\lock} \\
    \text{\colorbox{yellow}{$\obj{\emptyevent}$}}& \mid & \obj{\ev{Branch}\ 1;\comp;\lock} \\
    \obj{\ev{Dealloc}\ \loc_{y};\comp;\unlock} & \mid & \obj{\ev{Dealloc}\ \loc_{y};\comp;\lock} \\
    \obj{{\ev{iGet}}\ \loc_{y}\ 6;\comp;}\text{\colorbox{lightgray}{$\obj{\lock}$}} & \mid & \obj{\ev{Get}\ \loc_{y}\ 6;\comp;\lock} \\
  \end{array}
  % }
  $$
  $$
  % {
  \begin{array}{ccc}
          & \text{Specification} & \\
  \text{Active} & \mid & \text{Inactive} \\\hline
     \ev{Alloc}\ \loc_{x}\ 12;\unlock & \mid & \ev{Alloc}\ \loc_{x}\ 12;\unlock\\
     \ev{Alloc}\ \loc_{y}\ 12;\unlock & \mid & \ev{Alloc}\ \loc_{x}\ 12;\unlock\\
     \ev{Use}\ \loc_{x}\ 0;\text{\colorbox{lightgray}{$\unlock$}} & \mid & \ev{Use}\ \loc_{x}\ 0;\lock\\
     \emptyevent & \mid & \ev{Branch}\ 0;\lock\\
     \ev{Use}\ \loc_{x}\ 0;\text{\colorbox{lightgray}{$\unlock$}} & \mid & \ev{Use}\ \loc_{x}\ 0;\lock\\
     \ev{Use}\ \loc_{y}\ 0;\text{\colorbox{lightgray}{$\unlock$}} & \mid & \ev{Use}\ \loc_{y}\ 0;\lock\\
     \ev{Use}\ \loc_{x}\ 1;\text{\colorbox{lightgray}{$\unlock$}} & \mid & \ev{Use}\ \loc_{x}\ 1;\lock\\
     \emptyevent & \mid & \ev{Branch}\ 0;\lock\\
     \vdots & \mid & \vdots \\
      \ev{Use}\ \loc_{x}\ 12;\text{\colorbox{lightgray}{$\unlock$}} & \mid & \ev{Use}\ \loc_{x}\ 12;\lock\\
      \emptyevent & \mid & \ev{Branch}\ 1;\lock\\
      \ev{Dealloc}\ \loc_{y};\unlock & \mid & \ev{Dealloc}\ \loc_{y};\unlock\\
      \ev{Use}\ \loc_{y}\ 6;\text{\colorbox{lightgray}{$\unlock$}} & \mid & \ev{Use}\ \loc_{y}\ 6;\lock\\
  \end{array}
  % }
  $$
  \caption{Traces for \Cref{ex:lscct}.}
  \label{fig:ex-cct}
\end{figure}


\subsection{$\Lspec$: A Memory-Unsafe Language with Speculation}\label{subsec:lspec}

$\Lspec$ extends $\Lscct$ (\Cref{subsec:lscct}) with speculative dynamic semantics that is inspired by existing work~\cite{guarnieri2018spectector,fabian2022automatic}.
After branching, speculation starts by pushing the current configuration into the speculation state ($\ird{S}$), and run subsequent code in a predetermined window (whose size is $\omega$) until a rollback of operational state is performed. 
The window is set inside the speculation state.
Within such windows, data may leak, this is marked as high $\ird{\lock}$ and with an annotation that indicates the respective speculative execution variant. 
In this paper, we just consider SPECTRE-PHT~\cite{kocher2019spectre}, whose starting point is \Cref{tr:e-ifz-true-spec}. % where $\omega$ is the size of the speculation window, i.e., a number of reduction steps to be taken under speculative semantics.
Additionally, $\ird{\lock}$ may also carry a $\ird{NONE}$ annotation, to signal that this is a leak irrespective of speculation, i.e., as defined earlier (\Cref{subsec:lscct}).
The semantics is kept general enough to allow for future extension to support different variants~\cite{kocher2019spectre,maisuradze2018ret2spec,horn2019zero}.

In terms of language features, $\Lspec$ includes a new barrier operation $\ird{\lbarrier}$ that blocks speculative execution. 
To facilitate speculative execution in a non-assembly-like language, a stack of operational state is used and speculation is active if that stack has a size greater than one.
This is exploited in \Cref{tr:e-spec-eat} to consume any leftover speculation.

\vspace{-.5em}
\[
  \begin{array}{rrcl}
    \text{(Base-Events)} & \ird{\preevent} &:=& \dots \mid \ird{Spec} \mid \ird{Rlb} \mid \ird{Barrier}\\
    \text{(Expressions)} & \ird{\expr} &:=& \dots \mid \ird{\lbarrier} \\
    \text{(Speculation State)} & \ird{S} &:=& \ird{\Omega\triangleright\expr} \mid \ird{S},\left(\ird{\Omega};\ird{n};\ird{\expr}\right)
  \end{array}
\]

\begin{center}
\newcommand{\expreval}[5]{{#1}\triangleright\xspace {#2}\xrightarrow{#5}\ {#3}\triangleright\xspace {#4}\xspace}
\newcommand{\exprevald}[5]{\expreval{\ird{#1}}{\ird{#2}}{\ird{#3}}{\ird{#4}}{\ird{#5}}}

  %
  \typerule{$e-\ird{ifz}-\ird{true}-\text{spec}$}{
    \ird{S} = \rtt{(\ird{\cfstate['];\sandboxtag{};D;\memstate});\ird{S}}{\ird{\expr[_1]}},({\ird{\cfstate;\sandboxtag{};D;\memstate}};\omega;{\ird{\expr[_2]}})
  }{
    \pstepto{\rtt{\ird{\cfstate;\sandboxtag{};D;\memstate}}{\ird{\lifz{0^{\securitytag{}}}{\expr[_1]}{\expr[_2]}}}}{\ird{S}}{\ird{{Branch}\ 0;\sandboxtag{};\securitytag{}}\cdot\ird{{Spec};\sandboxtag{};\securitytag{}}}
  }{e-ifz-true-spec}
  %
  \typerule{$e-\ird{spec}-\text{eat}$}{
    \ird{n} > 0 &
    \pstepto{\rtt{\ird{\Omega}}{\ird{\expr}}}{\rtt{\ird{\Omega'}}{\ird{\expr[']}}}{\ird{\event}}
  }{
    \ghoststepto{\ird{S},(\ird{\Omega};\ird{n};\ird{\expr})}{\ird{S},(\ird{\Omega'};\ird{n}-1;\ird{\expr'})}{\ird{\event}}
  }{e-spec-eat}
  % 
  \typerule{$e-\ird{spec}-\text{eaten}$}{
  }{
    \ghoststepto{\ird{S},(\ird{\Omega};\ird{0};\ird{\expr})}{\ird{S}}{\ird{Rlb;\Omega.\sandboxtag{};\unlock}}
  }{e-spec-eaten}
  %
  \typerule{$e-\ird{spec}-\text{barrier}$}{
  }{
    \ghoststepto{\ird{S},(\ird{\Omega};\ird{n};\ird{\lbarrier})}{\ird{S},(\ird{\Omega};\ird{0};\ird{0})}{\ird{Barrier;\Omega.\sandboxtag{};\unlock}}
  }{e-spec-barrier}
\end{center}

The trace model is extended with $\ird{{Spec}}$, $\ird{{Rlb}}$, and $\ird{{Barrier}}$ events that signal the respective operational action. 
A $\ird{\lbarrier}$ prevents any execution whatsoever besides $\ird{{Rlb}}$ when run in speculative execution mode and does nothing when run in normal mode, e.g., \Cref{tr:e-spec-barrier}.%
\footnote{In the rule, the notation $\cfstate' = \cfstate[.\text{win} = 0]$ means that $\cfstate'$ is a copy of $\cfstate$ up to field $\text{win}$, which is set to $0$.}
Lastly, $\ird{Rlb}$ is emitted when the speculation window is zero and the operational state is rolled back (see \Cref{tr:e-spec-eaten}). 


