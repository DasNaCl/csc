\section{Security Properties Formalisation \& Composition\pages{1.5}}\label{sec:compprop}

This section introduces trace properties of interest for this paper: \gls*{tms}, \gls*{sms}, \gls*{ms}, \gls*{scct}, and \gls*{ss}.
These properties are of practical importance (as mentioned in \Cref{sec:introduction}) and also of interest in the case study presented later (\Cref{sec:casestud:defs,sec:casestud:rtp}). 
This section presents all of them, despite the fact that they are inspired by existing work, in order to showcase all that is required for a formal proof of security for a realistic compilation toolchain.
%\MP{ say something about this in the rebuttal: only the defs are borrowed, the universal images and all that work is new}
%\MK{ universal images are not new. }
% , none of the properties is novel, for self-containment of the paper, their definition is necessary for the case study.
% For each of the presented properties, the technical report defines corresponding monitors, which refine a given property and have a key role in the proofs of this paper. 
The technical report defines monitors for each of the presented properties.
% 
Monitors refine each property and have a key role in the proofs of this paper.
% managed to prove 
%  if tms-monitor --[trace]-->* ∅  
%  then trace ∈ tms
% but not
%  if trace ∈ tms
%  then tms-monitor --[trace]-->* ∅

\subsection{A Trace Model for Memory Safety}\label{subsec:basic:memsafety:tracemodel}

For simple memory safety composed of temporal and spatial memory safety, the trace model defines events ($\event[_{\mssafe}]$) as either the empty event ($\emptyevent$), a crash ($\lightning$), or a base-event ($\preevent[_{\mssafe}]$).

\vspace{-1.0em}
\[
  \begin{array}{rcll}
    \text{(Base-Events)}&\preevent[_{\mssafe}] &:=& \ev{Alloc \loc\ n} \mid \ev{Dealloc \loc} \mid \ev{Use \loc\ n} \\
    \text{(Events)}&\event[_{\mssafe}] &:=& \preevent[_{\mssafe}] \mid \emptyevent \mid \lightning \\ 
  \end{array}
\]

Base-events describe the actual kind of event that happened.
For the basic memory-safety properties, these are three variants:
First, the allocation event ($\ev{Alloc\ \loc\ n}$) that fires whenever a program claims $n$ cells of memory and stores them at address $\loc$, where addresses are assumed to be unique.
Second, deallocation ($\ev{Dealloc\ \loc}$) announces that the object at location $\loc$ is freed.
Third, an event to describe reads from and writes to the $n$-th memory cell from address $\loc$ ($\ev{Use\ \loc\ n}$).

\subsubsection{Temporal Memory Safety}

\gls*{tms}~\cite{nagarakatte2010cets} is a safety property that describes that an unallocated object must not be (re-)used.

\begin{definition}[\glsfirst*{tms}]\label{def:trace:tmsdef}
  % \footnotesize
  $$
  \tmssafe:=\left\{\trace[_{\mssafe}] \left| \begin{array}{rcl}
    \ev{Alloc\ \loc\ n}&\le_{\trace[_{\mssafe}]}&\ev{Dealloc\ \loc} \\
    \ev{Use\ \loc\ n}&\le_{\trace[_{\mssafe}]}&\ev{Dealloc\ \loc} \\
    \text{at most one }\ev{Dealloc\ \loc}&\text{in}&\trace[_{\mssafe}] \\
    \text{at most one }\ev{Alloc\ \loc\ n}&\text{in}&\trace[_{\mssafe}] \\
  \end{array}\right.\right\}
  $$
\end{definition}
Hereby, the notation $\event[_{1}]\le_{\trace}\event[_{2}]$ means that if $\event[_{1}]$ is in $\trace$ and if $\event[_{2}]$ is in $\trace$, then $\event[_{1}]$ appears before $\event[_{2}]$.

\subsubsection{Spatial Memory Safety}

\gls*{sms}~\cite{nagarakatte2009soft} is a safety property that prohibits out-of-bounds accesses.

\begin{definition}[\glsfirst*{sms}]\label{def:trace:smsdef}
% \small
  % \begin{nscenter}

  \noindent
  \[
  \smssafe:=\left\{\trace[_{\mssafe}] \left|\begin{array}{rcl}
      \text{If }\ev{Alloc\ \loc\ n}\le_{\trace_{\mssafe}}\ev{Use\ \loc\ m}, \text{ then }m<n
  \end{array}\right.\right\}
  \]
  % \end{nscenter}
\end{definition}

\subsubsection{Memory Safety}

In spirit of earlier work~\cite{nagarakatte2009soft,nagarakatte2010cets,jim2002cyclone,necula2005ccured,michael2023mswasm}, full \gls*{ms} is the intersection of \Cref{def:trace:tmsdef,def:trace:smsdef}.

\begin{definition}[\glsfirst*{ms}]\label{def:trace:msdef}
  $
  \mssafe:=\tmssafe \cap \smssafe
  $
\end{definition}

Note that \Cref{def:trace:msdef} ignores data isolation, so there may still be memory-safety issues introduced by side-channels.

\subsection{A Trace Model for Memory Safety with Constant Time}\label{subsec:scct:tracemodel}

To express Constant Time, we extend the memory safety trace model with a {\em security tag} ($\securitytag{}$) that indicates whether events contain sensitive information ($\lock$) or not ($\unlock$).

\vspace{-.5em}
\[
  \begin{array}{rrcl}
    (\text{Base-Events}) & \preevent[_{\ctsafe}] &:=& \preevent[_{\mssafe}] \mid \ev{Branch\ }n \mid \ev{Binop\ }n\\
    (\text{Security Tags}) & \securitytag{} &:=& \lock \mid \unlock\\ 
    (\text{Events}) & \event[_{\ctsafe}] &:=& \preevent[_{\ctsafe}];\securitytag{} \mid \emptyevent \mid \lightning \\ 
  \end{array}
\]

For cryptographic code, there is a general guideline that secrets must not be visible on a trace~\cite{ctguidelines}, i.e., secrets should not be marked as $\unlock$.
In turn, an instruction whose timing is data-dependent must not have a secret as an operand.
Typical operations with data-dependent timing are branches and certain binary operations, such as division.%
\footnote{
	This is architecture-dependent, but division is an operation that serves as a classic example for a data-dependent timing instruction~\cite[p.~755]{arm-refman}.
}
Both operations are represented in the trace model by extending the set of base-events with branches ($\ev{Branch\ n}$) and binary operations ($\ev{Binop\ n}$).

\subsubsection{Strict Cryptographic Constant Time}

\gls*{cct} is a hypersafety property~\cite{barthe2018sec} and, thus, difficult to check with monitors.
This is because, intuitively, hypersafety properties can relate multiple execution traces with each other, but monitors work on a single execution.
It is a common trick to sidestep this issue by means of overapproximation: this section defines the property \gls*{scct}, a stricter variant of \gls*{cct} (inspired by earlier work~\cite{almeida2017jasmin}) that enforces the policy that no secret appears on a trace.
Programs that satisfy \gls*{scct} also satisfy \gls*{cct}, but programs that satisfy \gls*{cct} may not satisfy \gls*{scct}.

\begin{definition}[\glsfirst*{scct}]\label{def:trace:scctdef}
% \begin{nscenter}
  
  \noindent\[
  \scctsafe:=\left\{\trace[_{\ctsafe}] \left|\begin{array}{l}
      \trace[_{\ctsafe}]=\hole{\cdot} \text{ or } \\\exists\trace[_{\ctsafe}'],\trace[_{\ctsafe}]=\preevent[_{\ctsafe}];\unlock\cdot\trace[_{\ctsafe}'] \wedge \trace[_{\ctsafe}']\in\scctsafe
    \end{array}\right.\right\}
  \]
% \end{nscenter}
\end{definition}

\gls*{scct} may appear overly strict, since it seems that secrets must not occur on a trace (since $\securitytag{}$ is forced to be $\unlock$). 
However, this is considered standard practice in terms of coding guidelines~\cite{ctguidelines}.
Moreover, programs that have been compiled with FaCT~\cite{cauligi2019fact} and run with a ``data independent timing mode''~\cite{arm-refman,intel-refman} enabled do not leak secrets (see \Cref{ex:lscct}). 

\subsubsection{\gls*{ms}, Strict Cryptographic Constant Time}\label{sec:msscct-rel}

The combination of \gls*{ms} and \gls*{scct} is the intersection of these properties, \gls*{msscct}.
However, \gls*{ms} uses a different trace model than \gls*{scct}, so intersecting them would trivially yield the empty set. 
To remedy this issue, we introduce $\sim_{\ctsafe}: \preevent[_{\ctsafe}] \times \preevent[_{\mssafe}] $, a cross-language trace relation (whose key cases are presented below), that we use to intuitively unify the trace model in which the two properties are expressed:
\[
  \typerulenolabel{mscctrel:drop:tag}{}{\preevent[_{\mssafe}]\sim_{\ctsafe}\preevent[_{\mssafe}];\unlock{}}
  \typerulenolabel{mscctrel:drop:crash}{}{\lightning\sim_{\ctsafe}\lightning}
\]
\[
  \typerulenolabel{mscctrel:drop:branch}{}{\emptyevent\sim_{\ctsafe}\ev{Branch n};\securitytag{}}
  \typerulenolabel{mscctrel:drop:binop}{}{\emptyevent\sim_{\ctsafe}\ev{Binop n};\securitytag{}}
\]

Essentially, $\sim_{\ctsafe}$ ignores both the new $\ev{Branch\ n}$ and $\ev{Binop\ n}$ base-events as it relates security-insensitive actions ($\unlock$) to their equivalent counterparts.
Thus, \gls*{ms} traces trivially satisfy \gls*{scct}.
% 
The above relation is extended point-wise to traces, skipping the empty event $\emptyevent$ on either side, and it is now possible to define \gls*{msscct} using the universal image:

\begin{definition}[\gls*{ms} and \gls*{scct}]\label{def:trace:msscctdef}
  $
  \msscctsafe:=\mssafe\cap\sigma_{\sim_{\ctsafe}}\left(\scctsafe\right)
  $
\end{definition}

\subsection{Extending the Trace Model with Speculation}\label{subsec:msctss:tracemodel}

So far, the considered trace models do not let us express speculative execution attacks such as Spectre~\cite{kocher2019spectre}. 
For this, we extend the earlier trace model (see \Cref{subsec:scct:tracemodel}) so that the security tags ($\securitytag{}$) carry additional information about the kind of private data leakage, i.e., the type of speculative leak.
Moreover, we add base-events signalling the beginning of a speculative execution ($\ev{Spec}$), a barrier ($\ev{Barrier}$) that signals that any speculative execution may not go past it, as well as a rollback event ($\ev{Rlb}$), which signals that execution resumes to where speculation started.

\vspace{-1em}
{
\[
  \begin{array}{rrcl}
    (\text{Base-Events}) & \preevent[_{\mathghost}] &:=& \preevent[_{\ctsafe}] \\
    (\text{Spectre Variants}) & vX &:=& \operatorname{NONE} \mid \operatorname{PHT} \\
    (\text{Security Tags}) & \securitytag{} &:=& \lock_{vX} \mid \unlock\\ 
    (\text{Events}) & \event[_{\mathghost}] &:=& \preevent[_{\mathghost}];\securitytag{} \mid \emptyevent \mid \lightning \mid \ev{Spec} \mid \ev{Rlb} \mid \ev{Barrier}\\ 
  \end{array}
\]
}

Even though the considered Spectre variants are just SPECTRE-PHT~\cite{kocher2019spectre}, NONE just describes secret data as in \gls*{scct} (see \Cref{subsec:scct:tracemodel}), the trace model is general enough to allow for potential future extension with different variants~\cite{kocher2019spectre,maisuradze2018ret2spec,horn2019zero}.
%For sake of readability, this paper just uses the notation $\lock$ in place of $\lock_{\text{PHT}}$.

\subsubsection{Speculative Safety}

\gls*{ss}~\cite{patrignani2021exorcising}, similar to \gls*{scct}, is a sound overapproximation of a variant of noninterference.

\begin{definition}[\glsfirst*{ss}]\label{def:trace:ss}
  \noindent

  \begin{nscenter}
  $
    \sssafe:=\left\{\trace[_{\mathghost}] \left|\begin{array}{l}
      \trace[_{\mathghost}]=\hole{\cdot} \text{ or } \exists\trace[_{\mathghost}'].\\
      \left(\trace[_{\mathghost}]=\preevent[_{\mathghost}];\unlock\cdot\trace[_{\mathghost}'] \text{or }\trace[_{\mathghost}]=\preevent[_{\mathghost}];\lock_{\text{NONE}}\cdot\trace[_{\mathghost}']\right)\\
      \text{and }\ \trace[_{\mathghost}']\in\sssafe
                                 \end{array}\right.\right\}
  $ 
  \end{nscenter}
\end{definition}
The technical setup so far leads to the above definition, where only locks annotated with $\text{SPECTRE-PHT}$ are disallowed to occur on the trace.
That way, programs attaining \gls*{ss} do not necessarily attain \gls*{scct}.

\subsubsection{Speculation Memory Safety}\label{sec:spec-ms-rel}

As before, we need to relate the different trace models with each other, so that the memory safety property without speculation can be lifted to speculation. 
To this end, let $\sim_{\mathghost}: \preevent[_{\mathghost}] \times \preevent[_{\ctsafe}]$ be a cross-language trace relation whose key cases are below.
The intuition is that \gls*{ss} is trivially satisfied in \gls*{scct}, since speculation is inexpressible there, which amounts to dropping events $\ev{Spec}$, $\ev{Rlb}$, or $\ev{Barrier}$, as well as all base events tagged with $\lock_{\text{PHT}}$. 

\[
  \typerulenolabel{specrel:drop:tag}{}{\preevent[_{\ctsafe}];\lock\sim_{\mathghost}\preevent[_{\mathghost}];\lock_{\text{NONE}}}
  \typerulenolabel{specrel:drop:spectag}{}{\emptyevent\sim_{\mathghost}\preevent[_{\mathghost}];\lock_{\text{PHT}}}
\]
\[
  \typerulenolabel{specrel:drop:pub}{}{\preevent[_{\ctsafe}];\unlock\sim_{\mathghost}\preevent[_{\mathghost}];\unlock}
  \typerulenolabel{specrel:drop:crash}{}{\lightning\sim_{\mathghost}\lightning}
\]
\[
  \typerulenolabel{specrel:drop:spec}{}{\emptyevent\sim_{\mathghost}\ev{Spec}}
  \typerulenolabel{specrel:drop:rlb}{}{\emptyevent\sim_{\mathghost}\ev{Rlb}}
  \typerulenolabel{specrel:drop:barrier}{}{\emptyevent\sim_{\mathghost}\ev{Barrier}}
\]

We conclude by defining the ultimate property of interest for secure compilers: \gls*{specms}.
\begin{definition}[\gls*{specms}]\label{def:trace:specmsdef}
  $
  \specmssafe := \msscctsafe\cap\sigma_{\sim_{\mathghost}}\left(\sssafe\right)
  $
\end{definition}







