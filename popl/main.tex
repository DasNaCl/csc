%%
%% This is file `sample-sigconf.tex',
%% generated with the docstrip utility.
%%
%% The original source files were:
%%
%% samples.dtx  (with options: `sigconf')
%% 
%% IMPORTANT NOTICE:
%% 
%% For the copyright see the source file.
%% 
%% Any modified versions of this file must be renamed
%% with new filenames distinct from sample-sigconf.tex.
%% 
%% For distribution of the original source see the terms
%% for copying and modification in the file samples.dtx.
%% 
%% This generated file may be distributed as long as the
%% original source files, as listed above, are part of the
%% same distribution. (The sources need not necessarily be
%% in the same archive or directory.)
%%
%% Commands for TeXCount
%TC:macro \cite [option:text,text]
%TC:macro \citep [option:text,text]
%TC:macro \citet [option:text,text]
%TC:envir table 0 1
%TC:envir table* 0 1
%TC:envir tabular [ignore] word
%TC:envir displaymath 0 word
%TC:envir math 0 word
%TC:envir comment 0 0
%%
%%
%% The first command in your LaTeX source must be the \documentclass command.
\documentclass[acmsmall,review,screen]{acmart}

\usepackage{listings}

%% NOTE that a single column version may be required for 
%% submission and peer review. This can be done by changing
%% the \doucmentclass[...]{acmart} in this template to 
%% \documentclass[manuscript,screen]{acmart}
%% 
%% To ensure 100% compatibility, please check the white list of
%% approved LaTeX packages to be used with the Master Article Template at
%% https://www.acm.org/publications/taps/whitelist-of-latex-packages 
%% before creating your document. The white list page provides 
%% information on how to submit additional LaTeX packages for 
%% review and adoption.
%% Fonts used in the template cannot be substituted; margin 
%% adjustments are not allowed.
%%
%%
%% \BibTeX command to typeset BibTeX logo in the docs
\AtBeginDocument{%
  \providecommand\BibTeX{{%
    \normalfont B\kern-0.5em{\scshape i\kern-0.25em b}\kern-0.8em\TeX}}}

%% Rights management information.  This information is sent to you
%% when you complete the rights form.  These commands have SAMPLE
%% values in them; it is your responsibility as an author to replace
%% the commands and values with those provided to you when you
%% complete the rights form.
\setcopyright{acmcopyright}
\copyrightyear{2024}
\acmYear{2024}
\acmDOI{XXXXXXX.XXXXXXX}

%% These commands are for a PROCEEDINGS abstract or paper.
\acmConference[POPL '24]{Make sure to enter the correct
  conference title from your rights confirmation emai}{June 03--05,
  2018}{Woodstock, NY}
%
%  Uncomment \acmBooktitle if th title of the proceedings is different
%  from ``Proceedings of ...''!
%
%\acmBooktitle{Woodstock '18: ACM Symposium on Neural Gaze Detection,
%  June 03--05, 2018, Woodstock, NY} 
\acmPrice{15.00}
\acmISBN{978-1-4503-XXXX-X/18/06}


%%
%% Submission ID.
%% Use this when submitting an article to a sponsored event. You'll
%% receive a unique submission ID from the organizers
%% of the event, and this ID should be used as the parameter to this command.
%%\acmSubmissionID{123-A56-BU3}

%%
%% For managing citations, it is recommended to use bibliography
%% files in BibTeX format.
%%
%% You can then either use BibTeX with the ACM-Reference-Format style,
%% or BibLaTeX with the acmnumeric or acmauthoryear sytles, that include
%% support for advanced citation of software artefact from the
%% biblatex-software package, also separately available on CTAN.
%%
%% Look at the sample-*-biblatex.tex files for templates showcasing
%% the biblatex styles.
%%

%%
%% The majority of ACM publications use numbered citations and
%% references.  The command \citestyle{authoryear} switches to the
%% "author year" style.
%%
%% If you are preparing content for an event
%% sponsored by ACM SIGGRAPH, you must use the "author year" style of
%% citations and references.
%% Uncommenting
%% the next command will enable that style.
\citestyle{acmauthoryear}

%%
%% end of the preamble, start of the body of the document source.
\begin{document}

%%
%% The "title" command has an optional parameter,
%% allowing the author to define a "short title" to be used in page headers.
\title{Composing Robustly Safe Compilers}

%%
%% The "author" command and its associated commands are used to define
%% the authors and their affiliations.
%% Of note is the shared affiliation of the first two authors, and the
%% "authornote" and "authornotemark" commands
%% used to denote shared contribution to the research.
\author{Matthis Kruse}
\authornote{Both authors contributed equally to this research.}
\email{matthis.kruse@cispa.de}
\orcid{0000-0003-4062-9666}
\affiliation{%
  \institution{CISPA Helmholtz Center for Information Security}
  \streetaddress{Stuhlsatzenhaus 5}
  \city{Saarbr{\"u}cken}
  \state{Saarland}
  \country{Germany}
  \postcode{66123}
}

\author{Marco Patrignani}
\orcid{0000-0003-3411-9678}
\email{marco.patrignani@unitn.it}
\affiliation{%
  \institution{University of Trento}
  \streetaddress{Via Sommarive, 9}
  \city{Povo}
  \country{Italy}
  \postcode{38123}
}

%%
%% By default, the full list of authors will be used in the page
%% headers. Often, this list is too long, and will overlap
%% other information printed in the page headers. This command allows
%% the author to define a more concise list
%% of authors' names for this purpose.
\renewcommand{\shortauthors}{Kruse and Patrignani}

%%
%% The abstract is a short summary of the work to be presented in the
%% article.
\begin{abstract}
% To this date, no practical compiler implementation is proven as robustly safe.
% This is no surprise, since proving compilers robustly safe is a huge endeavor.
% But, compilers consist of several different passes, which themselves can be seen as compilers that translate from one intermediate representation to another.
%
% This paper investigates the compositionality of robustly safe compilers.
% Composing two compilers means to simply plug the output of one as an input to the other.
% We show that it suffices to prove individual passes as robustly safe.
% We demonstrate that one can simply prove robust memory safety preservation for one pass and prove the robust preservation of cryptographic constant time for another pass.
%
%  Robustly Safe Compilers ensure that programs that do not go wrong when linked with an arbitrary source-level attacker also do not go wrong in their compiled counterpart when linked with untrusted target-level code.
  Programs that do not go wrong should also not go wrong when linked with arbitrary, untrusted code.
  Robustly safe compilers ensure that, given such program at source level, this also holds at target level.
  The robust safety criterion requires challenging proofs which, so far, do not exist in real-world compiler implementations.
  Moreover, these implementations are modular:
  Compilers consist of several different so-called passes, which translate from one intermediate representation to another.
  It is an open question whether one can split the proof of robust safety modularily as well.

  This paper answers this question positively:
  The output of one robust safety preserving compiler may serve as input to another and this composition of both compilers remains robustly safe preserving.
  We generalise this compositional behaviour for compilers that robustly preserve different security properties.
  Thus, formal compiler engineers do not need to prove that all passes robustly preserve all security properties of interest, because their composition yields the intersection of the security properties.
  A consequence of this is that even in the context of the phase ordering problem, whenever the optimisations are proven robustly safe, we can freely swap the optimisation passes without compromising security.
  We validate our meta-level findings by means of a case study that looks at three different robust safety preserving compilers.

  Lastly, parts of our results are verified in Coq.
\end{abstract}

%%
%% The code below is generated by the tool at http://dl.acm.org/ccs.cfm.
%% Please copy and paste the code instead of the example below.
%%
\begin{CCSXML}
<ccs2012>
  <concept>
  <concept_id>10002978.10002986.10002989</concept_id>
  <concept_desc>Security and privacy~Formal security models</concept_desc>
  <concept_significance>500</concept_significance>
  </concept>
</ccs2012>
\end{CCSXML}
\ccsdesc[500]{Security and privacy~Formal security models}

%%
%% Keywords. The author(s) should pick words that accurately describe
%% the work being presented. Separate the keywords with commas.
\keywords{Memory-safety, Secure Compilation, Privacy}

%% A "teaser" image appears between the author and affiliation
%% information and the body of the document, and typically spans the
%% page.
%\begin{teaserfigure}
%  \includegraphics[width=\textwidth]{sampleteaser}
%  \caption{Seattle Mariners at Spring Training, 2010.}
%  \Description{Enjoying the baseball game from the third-base
%  seats. Ichiro Suzuki preparing to bat.}
%  \label{fig:teaser}
%\end{teaserfigure}

%\received{20 February 2007}
%\received[revised]{12 March 2009}
%\received[accepted]{5 June 2009}

%%
%% This command processes the author and affiliation and title
%% information and builds the first part of the formatted document.
\maketitle

\section{Introduction}

\begin{lstlisting}[language=c,caption=``Wrong bounds check of two 64-bit integers.'']
if at <= bounds {
  x[at] = 42;
}
\end{lstlisting}
Note the subtle bug of using \verb|<=| instead of \verb|<|, leading to an out-of-bounds access whenever \verb|at = bounds|.
Furthermore, when \verb|at| and \verb|bounds| are 64-bit integers on a 32-bit architecture, the comparison may not be performed in constant-time: The compiler may bail out as soon as the lower 32 bits are unequal, not bothering to compare the higher 32 bits.

\section{Compositionality}

\section{Robust MS Preservation}

\section{Robust MS+CCT Preservation}

\section{Related Work}


\section{Conclusion}


%% The acknowledgments section is defined using the "acks" environment
%% (and NOT an unnumbered section). This ensures the proper
%% identification of the section in the article metadata, and the
%% consistent spelling of the heading.
\begin{acks}
To Robert, for the bagels and explaining CMYK and color spaces.
\end{acks}

%%
%% The next two lines define the bibliography style to be used, and
%% the bibliography file.
\bibliographystyle{ACM-Reference-Format}
\bibliography{sample-base}

%%
%% If your work has an appendix, this is the place to put it.
\appendix

\end{document}
\endinput
%%
%% End of file `sample-sigconf.tex'.
