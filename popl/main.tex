%%
%% This is file `sample-sigconf.tex',
%% generated with the docstrip utility.
%%
%% The original source files were:
%%
%% samples.dtx  (with options: `sigconf')
%% 
%% IMPORTANT NOTICE:
%% 
%% For the copyright see the source file.
%% 
%% Any modified versions of this file must be renamed
%% with new filenames distinct from sample-sigconf.tex.
%% 
%% For distribution of the original source see the terms
%% for copying and modification in the file samples.dtx.
%% 
%% This generated file may be distributed as long as the
%% original source files, as listed above, are part of the
%% same distribution. (The sources need not necessarily be
%% in the same archive or directory.)
%%
%% Commands for TeXCount
%TC:macro \cite [option:text,text]
%TC:macro \citep [option:text,text]
%TC:macro \citet [option:text,text]
%TC:envir table 0 1
%TC:envir table* 0 1
%TC:envir tabular [ignore] word
%TC:envir displaymath 0 word
%TC:envir math 0 word
%TC:envir comment 0 0
%%
%%
%% The first command in your LaTeX source must be the \documentclass command.
\documentclass[acmsmall,review,screen,dvipsnames]{acmart}

\usepackage[colorinlistoftodos]{todonotes}
\usepackage{cleveref}
\usepackage{glossaries}
\usepackage{listings}
\usepackage{xspace}

\usepackage[T1]{fontenc}
\usepackage[dvipsnames]{xcolor}
\usepackage{halloweenmath}
\usepackage{fontawesome5}
\usepackage{listofitems}
\usepackage{breakcites}
\usepackage{mathpartir}
\usepackage{glossaries}
\usepackage{mathtools}
\usepackage{tcolorbox}
\usepackage{etoolbox}
\usepackage{graphicx}
\usepackage{stmaryrd}
\usepackage{marvosym}
\usepackage{enumitem}
\usepackage{listings}
\usepackage{hyperref}
\usepackage[nosort]{cleveref}
\usepackage{mdframed}
\usepackage{makecell}
\usepackage{amsmath}
\usepackage{amssymb} %uncomment for ieee
\usepackage{nameref}
\usepackage{xspace}
\usepackage{xfrac}
\usepackage{array}
\usepackage{tikz}
\usepackage{soul}
\usepackage{bm}

\usepackage{noindentafter}

%%%% patch for enumitem
%% https://github.com/jbezos/enumitem/issues/48
\makeatletter
\@namedef{enitkv@enumitem-resume@resume*@default}{%
   \let\enit@resuming\thr@@
  \enit@ifunset{enit@resumekeys@\@currenvir}
    % Nothing to resume if this is the first occurrance of \@currenvir.
    % An empty \enit@resumekeys results in \enitkv@setkeys{enumitem}{,resume}
    % called in \enit@setresume.
    {\def\enit@resumekeys{}}
    {\let\enit@resuming\thr@@
     \expandafter\let\expandafter\enit@resumekeys
       \csname enit@resumekeys@\@currenvir\endcsname
     \@nameuse{enit@resume@\@currenvir}\relax}%
  }
\makeatother
%%%%

%% Line numbering. Remember to use \linenumbers after \begin{document}
\usepackage[right]{lineno}
\renewcommand\thelinenumber{\color{red}\arabic{linenumber}}

%%%%
% TODO macros. not using todonotes package, since it's damn slow
\newcommand{\todo}[2][red]{\colorbox{#1}{\begin{minipage}{0.9\textwidth}{#2}\end{minipage}}}
\newcommand{\MKin}[1]{\todo[orange!30]{MK: #1}}
\newcommand{\MPin}[1]{\todo[blue!30]{MP: #1}}

%% General utility
% List of contributions
\newcounter{contrib}
\newcommand{\contribnum}[0]{\stepcounter{contrib}{\arabic{contrib}}.~}
\newcommand{\contribution}[1]{\smallskip\noindent\textbf{{#1.}\xspace}}

 % fonts
\newcommand{\mi}[1]{\ensuremath{\mathit{#1}}}
\newcommand{\mr}[1]{\ensuremath{\mathrm{#1}}}
\newcommand{\mt}[1]{\ensuremath{\texttt{#1}}}
\newcommand{\mtt}[1]{\ensuremath{\mathtt{#1}}}
\newcommand{\mf}[1]{\ensuremath{\mathbf{#1}}}
\newcommand{\mk}[1]{\ensuremath{\mathfrak{#1}}}
\newcommand{\mc}[1]{\ensuremath{\mathcal{#1}}}
\newcommand{\ms}[1]{\ensuremath{\mathsf{#1}}}
\newcommand{\mb}[1]{\ensuremath{\mathbb{#1}}}
\newcommand{\msc}[1]{\ensuremath{\mathscr{#1}}}

% underlines
\newcommand{\bul}[1]{{\setulcolor{RoyalBlue}\ul{#1}}}
\newcommand{\rul}[1]{{\setulcolor{RedOrange}\ul{#1}}}
\newcommand{\iul}[1]{{\setulcolor{Apricot}\ul{#1}}}
\newcommand{\oul}[1]{{\setulcolor{Emerald}\ul{#1}}}
\newcommand{\pul}[1]{{\setulcolor{CarnationPink}\ul{#1}}}

% Colors
\newcommand{\neutcol}[0]{black}
\newcommand{\stlccol}[0]{RoyalBlue}
\newcommand{\irccol}[0]{Apricot}
\newcommand{\ulccol}[0]{RedOrange}
\newcommand{\objcol}[0]{Emerald} %CarnationPink}
\newcommand{\commoncol}[0]{black}
\newcommand{\irdcol}[0]{CarnationPink}

\newcommand{\col}[2]{\ensuremath{{\color{#1}{#2}}}}

\newcommand{\com}[1]{\ensuremath\mathit{\col{\neutcol}{#1}}}
\newcommand{\src}[1]{\ensuremath\mathsf{\col{\stlccol}{#1}}}
\newcommand{\irl}[1]{\ensuremath\mathit{\col{\irccol}{#1}}}
\newcommand{\trg}[1]{\ensuremath\mathbf{\col{\ulccol}{#1}}}
\newcommand{\obj}[1]{\ensuremath\mathtt{\col{\objcol}{#1}}}
\newcommand{\ird}[1]{\ensuremath\mathit{\col{\irdcol}{#1}}}

%% theorem stuff
\newenvironment{proof}[1][\textbf{Proof}]%
{\everypar={{\setbox0=\lastbox}\everypar{}}\restartlist{passumptionslist}\mbox{\textsc{#1}.}$\;$\noindent\everypar={{\setbox0=\lastbox}\everypar{}}}
{$\;$\hfill$\square$\\\ignorespacesafterend%
\ifx\paslistend\undefined\else\global\let\paslistend\undefined\fi%
\vspace{-2ex}\hrule%
}
\newcommand{\incompleteProof}[1][]{\begin{center}todo #1\end{center}}
% Theorem environments
\Crefname{exampleenv}{Example}{Examples}
\Crefname{definitionenv}{Definition}{Definitions}
\Crefname{lemmaenv}{Lemma}{Lemmata}
\Crefname{theoremenv}{Theorem}{Theorems}
\Crefname{corollaryenv}{Corollary}{Corollaries}
\Crefname{axiomenv}{Axiom}{Axioms}
\newcounter{example}
\newcounter{definition}
\newcounter{lemma}
\newcounter{theorem}
\newcounter{corollary}
\newcounter{axiom}

% A symbol for Coq-verified theorems.
\newcommand{\BareCoqSymbol}{\includegraphics[height=0.9em]{coq.pdf}}
\newcommand{\CoqSymbol}{\raisebox{-.2ex}{\BareCoqSymbol\,}}
\newcommand{\Coqed}{{\hfill\CoqSymbol}}

% defining theorem environments for their specialized versions
\makeatletter
\newenvironment{base@thm}[4][\unskip]%
{\ifx\paslistend\undefined\else\global\let\paslistend\undefined\fi%
  $\;$\linebreak\noindent\mbox{$\triangleright\;\textsc{\textbf{\large #2} #3 (#4) #1.}$}}
{\hfill$\;$\linebreak}
\makeatother

% example environments take a label and a name
\crefname{example}{example}{examples}
\makeatletter
\newenvironment{example}[2][\unskip]%
{\refstepcounter{example}\begin{base@thm}[#1]{Example}{\theexample}{#2}}
{\end{base@thm}\noindent}
% referencing an example
\newcommand{\examplelabel}[2][PLACEHOLDER]{{#1}\label[example]{example:#2}\def\@currentlabel{#1}\label{t@example:#2}}
%\def\@currentlabel{#1}\label{t@example:#2}}
\newcommand{\exampleref}[1]{\Cref{example:#1}~(\ref{t@example:#1})}
\makeatother
% definition environments take a label and a name
\crefname{definition}{definition}{definitions}
\makeatletter
\newenvironment{definition}[2][\unskip]%
{\refstepcounter{definition}\begin{base@thm}[#1]{definition}{\thedefinition}{#2}}
{\end{base@thm}\noindent}
% referencing an definition
\newcommand{\definitionlabel}[2][PLACEHOLDER]{{#1}\label[definition]{definition:#2}\def\@currentlabel{#1}\label{t@definition:#2}}
%\def\@currentlabel{#1}\label{t@definition:#2}}
\newcommand{\definitionref}[1]{\Cref{definition:#1}~(\ref{t@definition:#1})}
\newcommand{\defref}[1]{\Cref{definition:#1}}
\makeatother
% lemma environments take a label and a name
\crefname{lemma}{lemma}{lemmas}
\makeatletter
\newenvironment{lemma}[2][\unskip]%
{\refstepcounter{lemma}\begin{base@thm}[#1]{lemma}{\thelemma}{#2}}
{\end{base@thm}\noindent}
% referencing an lemma
\newcommand{\lemmalabel}[2][PLACEHOLDER]{{#1}\label[lemma]{lemma:#2}\def\@currentlabel{#1}\label{t@lemma:#2}}
%\def\@currentlabel{#1}\label{t@lemma:#2}}
\newcommand{\lemmaref}[1]{\Cref{lemma:#1}~(\ref{t@lemma:#1})}
\newcommand{\lemref}[1]{\Cref{lemma:#1}}
\makeatother
% corollary environments take a label and a name
\crefname{corollary}{corollary}{corollarys}
\makeatletter
\newenvironment{corollary}[2][\unskip]%
{\refstepcounter{corollary}\begin{base@thm}[#1]{corollary}{\thecorollary}{#2}}
{\end{base@thm}\noindent}
% referencing an corollary
\newcommand{\corollarylabel}[2][PLACEHOLDER]{{#1}\label[corollary]{corollary:#2}\def\@currentlabel{#1}\label{t@corollary:#2}}
%\def\@currentlabel{#1}\label{t@corollary:#2}}
\newcommand{\corollaryref}[1]{\Cref{corollary:#1}~(\ref{t@corollary:#1})}
\makeatother
% axiom environments take a label and a name
\crefname{axiom}{axiom}{axioms}
\makeatletter
\newenvironment{axiom}[2][\unskip]%
{\refstepcounter{axiom}\begin{base@thm}[#1]{axiom}{\theaxiom}{#2}}
{\end{base@thm}\noindent}
% referencing an axiom
\newcommand{\axiomlabel}[2][PLACEHOLDER]{{#1}\label[axiom]{axiom:#2}\def\@currentlabel{#1}\label{t@axiom:#2}}
%\def\@currentlabel{#1}\label{t@axiom:#2}}
\newcommand{\axiomref}[1]{\Cref{axiom:#1}~(\ref{t@axiom:#1})}
\makeatother

% proofs
\crefname{asm}{assumption}{assumptions}
\crefname{goal}{goal}{goals}

%%% stack of passumptions 
\newtoks\passumptionsliststack
\passumptionsliststack={\empty}

\def\pushpassumptionsliststack#1#2{%
  \edef\tmp{{#1}\the#2}%
  #2=\expandafter{\tmp}%
}

\def\poppassumptionsliststack#1#2{%
  \expandafter\splitpassumptionsliststack\the#1\stop{#1}{#2}%
}

\def\splitpassumptionsliststack#1#2\stop#3#4{% 
  \def\tmp{#1}%
  \ifx\tmp\empty
  \else
    \def#4{#1}\global#3={#2}%
  \fi
} 
%%% stack of passumptions

\newlist{passumptionslist}{enumerate}{1}
\setlist[passumptionslist]{label=(\alph*)}

\newenvironment{assumptions}
    {\ifx\paslistend\undefined\else\global\let\paslistend\undefined\fi%
      \begin{passumptionslist}[start=1]
    }
    {
    \end{passumptionslist}
  }
\NoIndentAfterEnv{assumptions}
\newenvironment{goals}
    {\begin{enumerate}[label=(\roman*)]
    }
    {
    \end{enumerate}
  }
\NoIndentAfterEnv{goals}

\newcommand{\IH}[1][]{I{\kern-1.5pt}H#1}
\newenvironment{passumptions}[1][H]
  {%\thepassumptionslisti
  \ifx\paslistend\undefined%
    %{\color{red}\Large \thepassumptionslisti}
  \begin{passumptionslist}[resume*,label={$\left(#1_{\arabic*}\right)$}]%
  \else%
    %{\color{red}\Large \thepassumptionslisti === \paslistend}
  \begin{passumptionslist}[resume*,label={$\left(#1_{\arabic*}\right)$},start=\paslistend]%
  \global\let\paslistend\undefined%
  \fi%
  }
  {%
  \end{passumptionslist}%
  }
\NoIndentAfterEnv{passumptions}
% individual cases
\newcommand{\asm}[2]{\item\label[asm]{asm:#1} {$#2$}}
\newcommand{\goal}[2]{\item\label[goal]{goal:#1} {$#2$}}
\newcommand{\asmref}[1]{\Cref{asm:#1}}
\newcommand{\goalref}[1]{\Cref{goal:#1}}
% case environment
\makeatletter
\newenvironment{proofcase}[2][Case ]{%
    \global\edef\passumptionsliststackname{\thepassumptionslisti}%
    \pushpassumptionsliststack{\passumptionsliststackname}{\passumptionsliststack}%
    \par\noindent\ignorespaces%
    {\fbox{\textbf{#1}{$ #2 $}\textbf{:}}} %
    \begin{mdframed}[%
        topline=false,%
        rightline=false,%
        bottomline=false,%
        innertopmargin=0.2em,%
        innerbottommargin=0.4em,%
        innerrightmargin=0.7em,%
        rightmargin=0.7em,%
        innerleftmargin=0.7em,%
        leftmargin=0.7em,%
        linewidth=.15em,%
    ]\@ifpackageloaded{lineno}{\internallinenumbers}{}%
    \setlength{\parindent}{0pt} %
}{%
    \end{mdframed}\ignorespacesafterend%
    \poppassumptionsliststack{\passumptionsliststack}{\passumptionsliststackname}%
    \ifx\paslistend\undefined%
    \global\def\paslistend{\passumptionsliststackname+1}%
    \else%
    \global\def\paslistend{\passumptionsliststackname+1}%
    \fi%
    \setcounter{passumptionslisti}{\passumptionsliststackname}%
    %{\color{red}\Large\paslistend}
}
\makeatother
\NoIndentAfterEnv{proofcase}



% some cool operators
\newcommand{\isdef}[0]{\ensuremath{\mathrel{\overset{\makebox[0pt]{\mbox{\normalfont\tiny\sffamily def}}}{=}}}}

%% PL typesetting stuff 
% Lists
\newcommand{\mklist}[1]{\ensuremath\overline{#1}}
% Box explaining judgement
\newcommand{\textgraybox}[1]{\boxed{#1}}
\newdimen\zzfontsz
\newcommand{\fontsz}[2]{\zzfontsz=#1%
{\fontsize{\zzfontsz}{1.2\zzfontsz}\selectfont{#2}}}
\newcommand{\mathsz}[2]{\text{\fontsz{#1}{$#2$}}}
\newcommand{\instsymColon}{%
     \raisebox{-0.09ex}{\text{\normalfont{:}}}}
\newcommand{\judgboxfontsize}[1]{%
        \mathsz{11pt}{#1}%
}
\newcommand{\judgbox}[2]{%
      {\hrulefill\raggedright \textgraybox{\ensuremath{\judgboxfontsize{#1}}}\!\;%
        \fontsz{9pt}{\begin{tabular}[c]{l} #2 \end{tabular}\hrulefill} %
}}
\newcommand{\judgboxb}[2]{%
  \judgbox{#1}{#2}\hspace{1ex}\hrulefill%
}
\mdfdefinestyle{judgframe}{%
  topline=false, %
  innertopmargin=-2.84ex, %
  innerleftmargin=-.1ex, %
  innerrightmargin=0ex, %
  linewidth=.5pt %
}
\newcounter{judgements}
\crefname{judgements}{judgement}{judgements}
\newcommand{\judgementsref}[1]{\Cref{judgements:#1}}
\newenvironment{judgframe}[4][-2.84ex]
    {\begin{mdframed}[style=judgframe,innertopmargin=#1]\noindent%
      \judgbox{#3}{,,#4''}\def\thejudgements{\ensuremath{#3}}\refstepcounter{judgements}\label[judgements]{judgements:#2}\hrulefill\\[-0.9mm]%
      \begin{center}
    }
  {\end{center}%
   \end{mdframed}%
   \noindent%
  }
% Typerules
\newcounter{typerule}
\crefname{typerule}{rule}{rules}
% separator (may be different depending on used package, so its a macro)
\newcommand{\rulesep}{\qquad} %{\qquad\\\xspace}
\newcommand{\rulenewline}{\ensuremath \\\\}
% inductive predicates
\newcommand{\typeruleInt}[5]{%
	\def\thetyperule{#1}%
	\refstepcounter{typerule}%
	\label{tr:#4}%
	%
  %\ensuremath{\begin{array}{c}#5 \inference{#2}{#3}\end{array}}
  \ensuremath{\inferrule{ #2 }{ #3 }\quad #5}
}
\newcommand{\typerule}[4]{%
  \typeruleInt{#1}{#2}{#3}{#4}{\textsf{\scriptsize ({#1})}  }
}
\newcommand{\typerulenolabel}[2]{%
  \ensuremath{\inferrule{ #1 }{ #2 }}
}
\newcommand{\typederivX}[4][Right]{%
  %\ensuremath{\begin{array}{c} \inference{#2}{#3} #1\end{array}}
  \ensuremath{\inferrule*[#1=\textsc{\scriptsize(#2)}]{ #3 }{ #4 }}
}
\newcommand{\typederiv}[4][Right]{\typederivX[#1]{\trref{#2}}{#3}{#4}}
\newcommand{\asmderiv}[4][Right]{%
  %\ensuremath{\begin{array}{c} \inference{#2}{#3} #1\end{array}}
  \ensuremath{\inferrule*[#1=\textsc{\scriptsize(\asmref{#2})}]{ #3 }{ #4 }}
}
\newcommand{\lemderiv}[3]{%
  %\ensuremath{\begin{array}{c} \inference{#2}{#3} #1\end{array}}
  \ensuremath{\inferrule*[Right=\textsc{\scriptsize(\lemref{#1})}]{ #2 }{ #3 }}
}
%%%%%%%
% coinductive predicates
\newcommand{\cotyperuleInt}[5]{%
	\def\thetyperule{#1}%
	\refstepcounter{typerule}%
	\label{tr:#4}%
	%
  %\ensuremath{\begin{array}{c}#5 \inference{#2}{#3}\end{array}}
  \ensuremath{\mprset{fraction={===}}\inferrule{ #2 }{ #3 }~#5}
}
\newcommand{\cotyperule}[4]{%
  \cotyperuleInt{#1}{#2}{#3}{#4}{\textsf{\scriptsize ({#1})}  }
}
\newcommand{\cotyperulenolabel}[3]{%
	\def\thetyperule{#1}%
	\refstepcounter{typerule}%
  %\ensuremath{\begin{array}{c} \inference{#2}{#3}\end{array}}
  \ensuremath{\coinferrrule{ #2 }{ #3 }}
}
\newcommand{\cotyperulederiv}[3]{%
  %\ensuremath{\begin{array}{c} \inference{#2}{#3} #1\end{array}}
  \ensuremath{\mprset{fraction={===}}\inferrule{ #2 }{ #3 }~\textsc{(#1)}}
}
% referencing a rule
\newcommand{\trref}[1]{\Cref{tr:#1}}
\newcommand{\trrefshandler}[1]{,tr:#1}
\DeclareListParser*\forsemicolonlist{;}
\newcommand{\trrefs}[1]{\Cref{\forsemicolonlist\trrefshandler{#1}}}

% sandwiching
\newcommand{\lift}[1]{\ensuremath\lfloor\xspace{#1}\xspace\rfloor}
\newcommand{\hole}[1]{\ensuremath{\left[#1\right]}}
\newcommand{\denot}[1]{\ensuremath\left\llbracket#1\right\rrbracket\xspace}
\newcommand{\lrpars}[1]{\ensuremath\left(#1\right)\xspace}
\newcommand{\lrbrackets}[1]{\ensuremath\left[#1\right]\xspace}
\newcommand{\lrbraces}[1]{\ensuremath\left\{#1\right\}\xspace}
\newcommand{\lrbbraces}[1]{\ensuremath\left\llbracket{#1}\right\rrbracket\xspace}

%\newcommand{\tup}[2]{\ensuremath (#1 %
%  \readlist\myterms{#2}%
%  \foreachitem\x\in\myterms{;\x}%
%  )%
%}

%%%%%%%%%%%%%%%%%%%%%%%%%%%%%%%%
%% Properties Names
\newcommand{\tmssafe}{\text{tms}}
\newcommand{\smssafe}{\text{sms}}
\newcommand{\mssafe}{\text{ms}}
\newcommand{\ctsafe}{\text{ct}}
\newcommand{\scctsafe}{\text{sct}}
\newcommand{\msscctsafe}{\text{mssct}}
\newcommand{\sssafe}{\text{ss}}
\newcommand{\specmssafe}{\text{specms}}


\newcommand{\observer}[1][]{\ensuremath \llbracket\bullet\rrbracket#1}
\newcommand{\observe}[3][]{\ensuremath \llbracket#2\rrbracket#1\lrpars{#3}}
\newcommand{\stronger}{\ensuremath \sqsupseteq}

%%
\newcommand{\emptyevent}{\ensuremath \varepsilon}
\newcommand{\ev}[1]{\ensuremath\ulcorner #1 \urcorner}
\newcommand{\tmsev}[1]{\ensuremath\ulcorner #1 \urcorner\;^{\kern-3pt{\tmssafe}}}
\newcommand{\smsev}[1]{\ensuremath\ulcorner #1 \urcorner\;^{\kern-3pt{\smssafe}}}
\newcommand{\msev}[1]{\ensuremath\ulcorner #1 \urcorner\;^{\kern-3pt{\mssafe}}}
\newcommand{\scctev}[1]{\ensuremath\ulcorner #1 \urcorner\;^{\kern-3pt{\scctsafe}}}
\newcommand{\specev}[1]{\ensuremath\ulcorner #1 \urcorner\;^{\kern-3pt{\sssafe}}}
\newcommand{\seqctev}[1]{\ensuremath\ulcorner #1 \urcorner\;^{\kern-3pt{\text{seq}}}_{\kern-3pt\ctsafe}}
\newcommand{\specctev}[1]{\ensuremath\ulcorner #1 \urcorner\;^{\kern-3pt{\text{spec}}}_{\kern-3pt\ctsafe}}
\newcommand{\seqarchev}[1]{\ensuremath\ulcorner #1 \urcorner\;^{\kern-3pt{\text{seq}}}_{\kern-3pt\text{arch}}}
\newcommand{\specarchev}[1]{\ensuremath\ulcorner #1 \urcorner\;^{\kern-3pt{\text{spec}}}_{\kern-3pt\text{arch}}}
\newcommand{\seqmemev}[1]{\ensuremath\ulcorner #1 \urcorner\;^{\kern-3pt{\text{seq}}}_{\kern-3pt\text{mem}}}
\newcommand{\specmemev}[1]{\ensuremath\ulcorner #1 \urcorner\;^{\kern-3pt{\text{spec}}}_{\kern-3pt\text{mem}}}

\newcommand{\specfenceev}[1]{\ensuremath\ulcorner \trg{#1} \urcorner\;^{\kern-3pt{\text{spec}}}_{\kern-3pt\text{fence}}}
\newcommand{\specslhev}[1]{\ensuremath\ulcorner \trg{#1} \urcorner\;^{\kern-3pt{\text{spec}}}_{\kern-3pt\text{slh}}}


\newcommand{\lock}{\ensuremath\text{\scriptsize\faIcon{lock}}}
\newcommand{\unlock}{\ensuremath\text{\scriptsize\faIcon{lock-open}}}

\newcommand{\emptyTrace}{\ensuremath\hole{\cdot}}
\newcommand{\consTrace}[2]{\ensuremath{#1},{#2}}

\newcommand{\xlangrel}[2]{\ensuremath\sim^{#1}_{#2}}
\newcommand{\xlangreltr}[2]{\ensuremath\approx^{#1}_{#2}}

\newcommand{\seqctTOspecct}[2]{\xlangrel{seqct\kern2pt\lrpars{#1}}{specct\kern2pt\lrpars{#2}}}
\newcommand{\seqctTOspeccttr}[2]{\xlangreltr{seqct\kern2pt\lrpars{#1}}{specct\kern2pt\lrpars{#2}}}
\newcommand{\seqmemTOspecmem}[2]{\xlangrel{seqmem\kern2pt\lrpars{#1}}{specmem\kern2pt\lrpars{#2}}}
\newcommand{\seqmemTOspecmemtr}[2]{\xlangreltr{seqmem\kern2pt\lrpars{#1}}{specmem\kern2pt\lrpars{#2}}}
\newcommand{\seqarchTOspecarch}[2]{\xlangrel{seqarch\kern2pt\lrpars{#1}}{specarch\kern2pt\lrpars{#2}}}
\newcommand{\seqarchTOspecarchtr}[2]{\xlangreltr{seqarch\kern2pt\lrpars{#1}}{specarch\kern2pt\lrpars{#2}}}

\newcommand{\specmemTOspecct}{\xlangrel{specmem}{specct}}
\newcommand{\specmemTOspeccttr}{\xlangreltr{specmem}{specct}}
\newcommand{\seqmemTOseqct}{\xlangrel{seqmem}{seqct}}
\newcommand{\seqmemTOseqcttr}{\xlangreltr{seqmem}{seqct}}
\newcommand{\seqmemTOspecct}{\xlangrel{seqmem}{specct}}
\newcommand{\seqmemTOspeccttr}{\xlangreltr{seqmem}{specct}}
\newcommand{\specmemTOseqct}{\xlangrel{specmem}{seqct}}
\newcommand{\specmemTOseqcttr}{\xlangreltr{specmem}{seqct}}

\newcommand{\specctTOspecarch}{\xlangrel{specct}{specarch}}
\newcommand{\specctTOspecarchtr}{\xlangreltr{specct}{specarch}}
\newcommand{\seqctTOseqarch}{\xlangrel{seqct}{seqarch}}
\newcommand{\seqctTOseqarchtr}{\xlangreltr{seqct}{seqarch}}
\newcommand{\seqctTOspecarch}{\xlangrel{seqct}{specarch}}
\newcommand{\seqctTOspecarchtr}{\xlangreltr{seqct}{specarch}}
\newcommand{\specctTOseqarch}{\xlangrel{specct}{seqarch}}
\newcommand{\specctTOseqarchtr}{\xlangreltr{specct}{seqarch}}

\newcommand{\specmemTOspecarch}{\xlangrel{specmem}{specarch}}
\newcommand{\specmemTOspecarchtr}{\xlangreltr{specmem}{specarch}}
\newcommand{\seqmemTOseqarch}{\xlangrel{seqmem}{seqarch}}
\newcommand{\seqmemTOseqarchtr}{\xlangreltr{seqmem}{seqarch}}
\newcommand{\seqmemTOspecarch}{\xlangrel{seqmem}{specarch}}
\newcommand{\seqmemTOspecarchtr}{\xlangreltr{seqmem}{specarch}}
\newcommand{\specmemTOseqarch}{\xlangrel{specmem}{seqarch}}
\newcommand{\specmemTOseqarchtr}{\xlangreltr{specmem}{seqarch}}


\newcommand{\specarchTOspecfence}[2]{\xlangrel{specarch\kern2pt\lrpars{#1}}{specfence\kern2pt\lrpars{#2}}}
\newcommand{\specarchTOspecfencetr}[2]{\xlangreltr{specarch\kern2pt\lrpars{#1}}{specfence\kern2pt\lrpars{#2}}}
\newcommand{\specarchTOspecslh}[2]{\xlangrel{specarch\kern2pt\lrpars{#1}}{specslh\kern2pt\lrpars{#2}}}
\newcommand{\specarchTOspecslhtr}[2]{\xlangreltr{specarch\kern2pt\lrpars{#1}}{specslh\kern2pt\lrpars{#2}}}

%% Language-Specific events
\newcommand{\makeConcreteEvent}[3]{\ensuremath\ev{\textnormal{#1}\;\lrpars{#2}}\;^{\kern-3pt{#3}}}

\newcommand{\msEvent}[1][]{\ensuremath \varEvent[_{\mssafe}#1]}
\newcommand{\msTrace}[1][]{\ensuremath \varTrace[_{\mssafe}#1]}
\newcommand{\msAlloc}[1][\varLoc;n]{\ensuremath\makeConcreteEvent{Alloc}{#1}{\mssafe}}
\newcommand{\msDealloc}[1][\varLoc]{\ensuremath\makeConcreteEvent{Dealloc}{#1}{\mssafe}}
\newcommand{\msUse}[1][\varLoc;n]{\ensuremath\makeConcreteEvent{Use}{#1}{\mssafe}}

\newcommand{\smsEvent}[1][]{\ensuremath \varEvent[_{\smssafe}#1]}
\newcommand{\smsTrace}[1][]{\ensuremath \varTrace[_{\smssafe}#1]}
\newcommand{\smsAlloc}[1][\varLoc;n]{\ensuremath\makeConcreteEvent{Alloc}{#1}{\smssafe}}
\newcommand{\smsDealloc}[1][\varLoc]{\ensuremath\makeConcreteEvent{Dealloc}{#1}{\smssafe}}
\newcommand{\smsUse}[1][\varLoc;m]{\ensuremath\makeConcreteEvent{Use}{#1}{\smssafe}}

\newcommand{\tmsEvent}[1][]{\ensuremath \varEvent[_{\tmssafe}#1]}
\newcommand{\tmsTrace}[1][]{\ensuremath \varTrace[_{\tmssafe}#1]}
\newcommand{\tmsAlloc}[1][\varLoc]{\ensuremath\makeConcreteEvent{Alloc}{#1}{\tmssafe}}
\newcommand{\tmsDealloc}[1][\varLoc]{\ensuremath\makeConcreteEvent{Dealloc}{#1}{\tmssafe}}
\newcommand{\tmsUse}[1][\varLoc]{\ensuremath\makeConcreteEvent{Use}{#1}{\tmssafe}}

\newcommand{\scctEvent}[1][]{\ensuremath \varEvent[_{\scctsafe}#1]}
\newcommand{\scctTrace}[1][]{\ensuremath \varTrace[_{\scctsafe}#1]}
\newcommand{\scctAny}[1][\varSecuritytag]{\ensuremath\makeConcreteEvent{Any}{#1}{\scctsafe}}

\newcommand{\specEvent}[1][]{\ensuremath \varEvent[_{\mathghost}#1]}
\newcommand{\specTrace}[1][]{\ensuremath \varTrace[_{\mathghost}#1]}
\newcommand{\specAny}[1][\varSecuritytag]{\ensuremath\makeConcreteEvent{Any}{#1}{\mathghost}}

%% Language-Specific
\newcommand{\ctxtocomp}{\xspace ? \xspace}
\newcommand{\comptoctx}{\xspace ! \xspace}

\newcommand{\varLoc}[1][]{\ensuremath \ell_{#1}}
\newcommand{\varSecuritytag}[1][]{\ensuremath\sigma_{#1}}

\newcommand{\varWholeProg}[1][]{\ensuremath w_{#1}}
\newcommand{\varComponent}[1][]{\ensuremath p_{#1}}
\newcommand{\varContext}[1][]{\ensuremath C_{#1}}
\newcommand{\varRuntimeTerm}[1][]{\ensuremath r_{#1}}

\newcommand{\varEvent}[1][]{\ensuremath a_{#1}}
\newcommand{\varTrace}[1][]{\ensuremath \overline{a_{#1}}}
\newcommand{\varProperty}[1][]{\ensuremath \pi_{#1}}
\newcommand{\varHyperProperty}[1][]{\ensuremath \Pi_{#1}}
\newcommand{\varClass}[1][]{\ensuremath \mathbb{C}_{#1}}

\newcommand{\seqctEvent}[1][]{\ensuremath a^{\text{seq}}_{\text{ct}}\kern-3pt{#1}}
\newcommand{\specctEvent}[1][]{\ensuremath a^{\text{spec}}_{\text{ct}}\kern-3pt{#1}}
\newcommand{\seqmemEvent}[1][]{\ensuremath a^{\text{seq}}_{\text{mem}}\kern-3pt{#1}}
\newcommand{\specmemEvent}[1][]{\ensuremath a^{\text{spec}}_{\text{mem}}\kern-3pt{#1}}
\newcommand{\seqarchEvent}[1][]{\ensuremath a^{\text{seq}}_{\text{arch}}\kern-3pt{#1}}
\newcommand{\specarchEvent}[1][]{\ensuremath a^{\text{spec}}_{\text{arch}}\kern-3pt{#1}}

\newcommand{\specfenceEvent}[1][]{\ensuremath \trg{a}^{\text{spec}}_{\text{fence}}\kern-3pt{#1}}
\newcommand{\specslhEvent}[1][]{\ensuremath \trg{a}^{\text{spec}}_{\text{slh}}\kern-3pt{#1}}

\newcommand{\seqctTrace}[1][]{\ensuremath \overline{a^{\text{seq}}_{\text{ct}}\kern-3pt{#1}}}
\newcommand{\specctTrace}[1][]{\ensuremath \overline{a^{\text{spec}}_{\text{ct}}\kern-3pt{#1}}}
\newcommand{\seqmemTrace}[1][]{\ensuremath \overline{a^{\text{seq}}_{\text{mem}}\kern-3pt{#1}}}
\newcommand{\specmemTrace}[1][]{\ensuremath \overline{a^{\text{spec}}_{\text{mem}}\kern-3pt{#1}}}
\newcommand{\seqarchTrace}[1][]{\ensuremath \overline{a^{\text{seq}}_{\text{arch}}\kern-3pt{#1}}}
\newcommand{\specarchTrace}[1][]{\ensuremath \overline{a^{\text{spec}}_{\text{arch}}\kern-3pt{#1}}}

\newcommand{\specfenceTrace}[1][]{\ensuremath \trg{\overline{a}}^{\text{spec}}_{\text{fence}}\kern-3pt{#1}}
\newcommand{\specslhTrace}[1][]{\ensuremath \trg{\overline{a}}^{\text{spec}}_{\text{slh}}\kern-3pt{#1}}


\newcommand{\varMonitor}[1][]{\ensuremath T_{#1}}
\newcommand{\tmsMonitor}[1][]{\ensuremath\varMonitor[\tmssafe]#1}
\newcommand{\smsMonitor}[1][]{\ensuremath\varMonitor[\smssafe]#1}
\newcommand{\msMonitor}[1][]{\ensuremath\varMonitor[\mssafe]#1}
\newcommand{\scctMonitor}[1][]{\ensuremath\varMonitor[\scctsafe]#1}
\newcommand{\msscctMonitor}[1][]{\ensuremath\varMonitor[\msscctsafe]#1}
\newcommand{\specMonitor}[1][]{\ensuremath\varMonitor[\sssafe]#1}
\newcommand{\specmsMonitor}[1][]{\ensuremath\varMonitor[\specmssafe]#1}
\newcommand{\dualMonitor}[1][]{\ensuremath\varMonitor[A;B]#1}

\newcommand{\noncrashMonitor}[1]{\ensuremath{{\,^\circ}#1}}

% language specific versions of above
%% src
\newcommand{\srcWholeProg}[1][]{\ensuremath \src{\varWholeProg[#1]}}
\newcommand{\srcComponent}[1][]{\ensuremath \src{\varComponent[#1]}}
\newcommand{\srcContext}[1][]{\ensuremath \src{\varContext[#1]}}
\newcommand{\srcRuntimeTerm}[1][]{\ensuremath \src{\varRuntimeTerm[#1]}}

\newcommand{\srcEvent}[1][]{\ensuremath \src{\varEvent[#1]}}
\newcommand{\srcTrace}[1][]{\ensuremath \src{\varTrace[#1]}}
\newcommand{\srcProperty}[1][]{\ensuremath \src{\varProperty[#1]}}
\newcommand{\srcHyperProperty}[1][]{\ensuremath \src{\varHyperProperty[#1]}}
\newcommand{\srcClass}[1][]{\ensuremath \src{\varClass[#1]}}
%% trg
\newcommand{\trgWholeProg}[1][]{\ensuremath \trg{\varWholeProg[#1]}}
\newcommand{\trgComponent}[1][]{\ensuremath \trg{\varComponent[#1]}}
\newcommand{\trgContext}[1][]{\ensuremath \trg{\varContext[#1]}}
\newcommand{\trgRuntimeTerm}[1][]{\ensuremath \trg{\varRuntimeTerm[#1]}}

\newcommand{\trgEvent}[1][]{\ensuremath \trg{\varEvent[#1]}}
\newcommand{\trgTrace}[1][]{\ensuremath \trg{\varTrace[#1]}}
\newcommand{\trgProperty}[1][]{\ensuremath \trg{\varProperty[#1]}}
\newcommand{\trgHyperProperty}[1][]{\ensuremath \trg{\varHyperProperty[#1]}}
\newcommand{\trgClass}[1][]{\ensuremath \trg{\varClass[#1]}}
%% irl
\newcommand{\irlWholeProg}[1][]{\ensuremath \irl{\varWholeProg[#1]}}
\newcommand{\irlComponent}[1][]{\ensuremath \irl{\varComponent[#1]}}
\newcommand{\irlContext}[1][]{\ensuremath \irl{\varContext[#1]}}
\newcommand{\irlRuntimeTerm}[1][]{\ensuremath \irl{\varRuntimeTerm[#1]}}

\newcommand{\irlEvent}[1][]{\ensuremath \irl{\varEvent[#1]}}
\newcommand{\irlTrace}[1][]{\ensuremath \irl{\varTrace[#1]}}
\newcommand{\irlProperty}[1][]{\ensuremath \irl{\varProperty[#1]}}
\newcommand{\irlHyperProperty}[1][]{\ensuremath \irl{\varHyperProperty[#1]}}
\newcommand{\irlClass}[1][]{\ensuremath \irl{\varClass[#1]}}

% bops
\newcommand{\bopLink}[2]{\ensuremath {#1}\bowtie{#2}}
\newcommand{\fncompo}[2]{\ensuremath {#1}\circ{#2}}

% different kinds of satifsaction
\newcommand{\sat}[2]{\ensuremath \vdash {#1} : {#2}}
\newcommand{\rsat}[2]{\ensuremath \vdash_R {#1} : {#2}}

\newcommand{\rtp}[2]{\ensuremath\;\vdash{#1}:{#2}}
\newcommand{\rtpUniversal}[3]{\ensuremath\;\vdash^{\forall}_{{#3}}{#1}:{#2}}
\newcommand{\rtpExistential}[3]{\ensuremath\;\vdash^{\exists}_{{#3}}{#1}:{#2}}


\newcommand{\progstepto}[3]{\ensuremath{#1}\xRightarrow{#3}{#2}}


\newcommand{\cc}[3][\gamma]{\ensuremath {#1}^{#2}_{#3}}
\newcommand{\ccST}[1][\gamma]{\ensuremath \cc[#1]{\src{S}}{\trg{T}}}
\newcommand{\ccSI}[1][\gamma]{\ensuremath \cc[#1]{\src{S}}{\irl{I}}}
\newcommand{\ccIT}[1][\gamma]{\ensuremath \cc[#1]{\irl{I}}{\trg{T}}}
\newcommand{\ccII}[1][\gamma]{\ensuremath \cc[#1]{\irl{I}}{\irl{I}}}


\newcommand{\mapUniversal}[2]{\ensuremath \overset{#1}{\varsigma}\left({#2}\right)}
\newcommand{\mapExistential}[2]{\ensuremath \overset{#1}{\tau}\left({#2}\right)}
\newcommand{\xrelTraces}[2]{\ensuremath {#1}\sim{#2}}


\newcommand{\mapUniversalWF}[2]{\ensuremath \vdash {#1} : \mathit{WF}^{\forall}_{#2}}
\newcommand{\mapExistentialWF}[2]{\ensuremath \vdash {#1}\ ; \mathit{WF}^{\exists}_{#2}}


% Reductions
\usetikzlibrary{calc,decorations.pathmorphing,shapes,positioning}

\newcounter{sarrow}
\newcommand\xrsquigarrow[1]{%
\stepcounter{sarrow}%
\mathrel{\begin{tikzpicture}[baseline= {( $ (current bounding box.south) + (0,-0.5ex) $ )}]
\node[inner sep=.5ex] (\thesarrow) {$\scriptstyle #1$};
\path[draw,<-,decorate,
  decoration={zigzag,amplitude=0.7pt,segment length=1.2mm,pre=lineto,pre length=4pt}] 
    (\thesarrow.south east) -- (\thesarrow.south west);
\end{tikzpicture}}%
}
\newcommand{\monitorcheck}[4][{\kern-3.5pt}^*]{%
  \vdash\xspace{#2}\xspace \xrsquigarrow{#4}{#1}\xspace{#3}\xspace%
}




\loadglsentries{acronyms}
\makeglossaries

%% NOTE that a single column version may be required for 
%% submission and peer review. This can be done by changing
%% the \doucmentclass[...]{acmart} in this template to 
%% \documentclass[manuscript,screen]{acmart}
%% 
%% To ensure 100% compatibility, please check the white list of
%% approved LaTeX packages to be used with the Master Article Template at
%% https://www.acm.org/publications/taps/whitelist-of-latex-packages 
%% before creating your document. The white list page provides 
%% information on how to submit additional LaTeX packages for 
%% review and adoption.
%% Fonts used in the template cannot be substituted; margin 
%% adjustments are not allowed.
%%
%%
%% \BibTeX command to typeset BibTeX logo in the docs
\AtBeginDocument{%
  \providecommand\BibTeX{{%
    \normalfont B\kern-0.5em{\scshape i\kern-0.25em b}\kern-0.8em\TeX}}}

%% Rights management information.  This information is sent to you
%% when you complete the rights form.  These commands have SAMPLE
%% values in them; it is your responsibility as an author to replace
%% the commands and values with those provided to you when you
%% complete the rights form.
\setcopyright{acmcopyright}
\copyrightyear{2024}
\acmYear{2024}
\acmDOI{XXXXXXX.XXXXXXX}

%% These commands are for a PROCEEDINGS abstract or paper.
\acmConference[POPL '24]{Make sure to enter the correct
  conference title from your rights confirmation emai}{June 03--05,
  2018}{Woodstock, NY}
%
%  Uncomment \acmBooktitle if th title of the proceedings is different
%  from ``Proceedings of ...''!
%
%\acmBooktitle{Woodstock '18: ACM Symposium on Neural Gaze Detection,
%  June 03--05, 2018, Woodstock, NY} 
\acmPrice{15.00}
\acmISBN{978-1-4503-XXXX-X/18/06}


%%
%% Submission ID.
%% Use this when submitting an article to a sponsored event. You'll
%% receive a unique submission ID from the organizers
%% of the event, and this ID should be used as the parameter to this command.
%%\acmSubmissionID{123-A56-BU3}

%%
%% For managing citations, it is recommended to use bibliography
%% files in BibTeX format.
%%
%% You can then either use BibTeX with the ACM-Reference-Format style,
%% or BibLaTeX with the acmnumeric or acmauthoryear sytles, that include
%% support for advanced citation of software artefact from the
%% biblatex-software package, also separately available on CTAN.
%%
%% Look at the sample-*-biblatex.tex files for templates showcasing
%% the biblatex styles.
%%

%%
%% The majority of ACM publications use numbered citations and
%% references.  The command \citestyle{authoryear} switches to the
%% "author year" style.
%%
%% If you are preparing content for an event
%% sponsored by ACM SIGGRAPH, you must use the "author year" style of
%% citations and references.
%% Uncommenting
%% the next command will enable that style.
\citestyle{acmauthoryear}

%%
%% end of the preamble, start of the body of the document source.
\begin{document}

%%
%% The "title" command has an optional parameter,
%% allowing the author to define a "short title" to be used in page headers.
\title{Composing Robustly Safe Compilers}

%%
%% The "author" command and its associated commands are used to define
%% the authors and their affiliations.
%% Of note is the shared affiliation of the first two authors, and the
%% "authornote" and "authornotemark" commands
%% used to denote shared contribution to the research.
\author{Matthis Kruse}
% \authornote{Both authors contributed equally to this research.}
\email{matthis.kruse@cispa.de}
\orcid{0000-0003-4062-9666}
\affiliation{%
  \institution{CISPA Helmholtz Center for Information Security}
  \streetaddress{Stuhlsatzenhaus 5}
  \city{Saarbr{\"u}cken}
  \state{Saarland}
  \country{Germany}
  \postcode{66123}
}

\author{Marco Patrignani}
\orcid{0000-0003-3411-9678}
\email{marco.patrignani@unitn.it}
\affiliation{%
  \institution{University of Trento}
  \streetaddress{Via Sommarive, 9}
  \city{Povo}
  \country{Italy}
  \postcode{38123}
}

%%
%% By default, the full list of authors will be used in the page
%% headers. Often, this list is too long, and will overlap
%% other information printed in the page headers. This command allows
%% the author to define a more concise list
%% of authors' names for this purpose.
\renewcommand{\shortauthors}{Kruse and Patrignani}

%%
%% The abstract is a short summary of the work to be presented in the
%% article.
\begin{abstract}
% To this date, no practical compiler implementation is proven as robustly safe.
% This is no surprise, since proving compilers robustly safe is a huge endeavor.
% But, compilers consist of several different passes, which themselves can be seen as compilers that translate from one intermediate representation to another.
%
% This paper investigates the compositionality of robustly safe compilers.
% Composing two compilers means to simply plug the output of one as an input to the other.
% We show that it suffices to prove individual passes as robustly safe.
% We demonstrate that one can simply prove robust memory safety preservation for one pass and prove the robust preservation of cryptographic constant time for another pass.
%
%  Robustly Safe Compilers ensure that programs that do not go wrong when linked with an arbitrary source-level attacker also do not go wrong in their compiled counterpart when linked with untrusted target-level code.
  Programs that do not go wrong should also not go wrong when linked with arbitrary, untrusted code.
  Robustly safe compilers ensure that, given such program at source level, this also holds at target level.
  The robust safety criterion requires challenging proofs which, so far, do not exist in real-world compiler implementations.
  Moreover, these implementations are modular:
  Compilers consist of several different so-called passes, which translate from one intermediate representation to another.
  It is an open question whether one can split the proof of robust safety modularily as well.

  This paper answers this question positively:
  The output of one robust safety preserving compiler may serve as input to another and this composition of both compilers remains robustly safe preserving.
  We generalise this compositional behaviour for compilers that robustly preserve different security properties.
  Thus, formal compiler engineers do not need to prove that all passes robustly preserve all security properties of interest, because their composition yields the intersection of the security properties.
  A consequence of this is that even in the context of the phase ordering problem, whenever the optimisations are proven robustly safe, we can freely swap the optimisation passes without compromising security.
  We validate our meta-level findings by means of a case study that looks at three different robust safety preserving compilers.

  Lastly, parts of our results are verified in Coq.

\begin{center}\small\it
	{In the following, we use syntax highlighting accessible to both colourblind and black \& white readers~\citep{patrignani2020use}. Specifically, we use a $\src{\text{blue}}$,
	$\src{sans\text{-}serif}$ font for $\src{L_{TMS}}$,
	$\trg{bold}$ font for $\trg{L}$ and $\trg{L_{MS}}$, and $\obj{green}$ font for $\obj{L_{sCCT}}$.
	}
\end{center}
\end{abstract}

%%
%% The code below is generated by the tool at http://dl.acm.org/ccs.cfm.
%% Please copy and paste the code instead of the example below.
%%
\begin{CCSXML}
<ccs2012>
  <concept>
  <concept_id>10002978.10002986.10002989</concept_id>
  <concept_desc>Security and privacy~Formal security models</concept_desc>
  <concept_significance>500</concept_significance>
  </concept>
</ccs2012>
\end{CCSXML}
\ccsdesc[500]{Security and privacy~Formal security models}

%%
%% Keywords. The author(s) should pick words that accurately describe
%% the work being presented. Separate the keywords with commas.
\keywords{Memory-safety, Secure Compilation, Privacy}

%%
%% This command processes the author and affiliation and title
%% information and builds the first part of the formatted document.
\maketitle

\section{Introduction}

\gls{cct} code does not leak information through timing analyses.
%However, it is possible to write \gls{cct} code in a more high-level language like FaCT~\cite{cauligi2019fact}.
It is folklore that programmers prefer to write code in higher-level languages such as \verb|C|.
But, high-level languages do not run on common hardware, so they need to be translated by a compiler.
Due to their optimizing nature, compilers can break security properties, such as \gls{cct}, when performing this translation.
An example of this exists in literature~\cite{barthe2018sec}:

\begin{minipage}{0.49\textwidth}
  \begin{lstlisting}[language=c,caption={Before Compilation},numbers=left]
// x = (b ? x : y)
int mask = -b;
x = (y & mask) | (x & ~mask);
  \end{lstlisting}
\end{minipage}%
\begin{minipage}{0.49\textwidth}
  \begin{lstlisting}[language=c,caption={After Compilation},numbers=left]
if(b) {
  x = y;
}
  \end{lstlisting}
\end{minipage}

\noindent
Here, the compiler recognizes the pattern in the bit hacks and rewrites them to code branching on \verb|b|, violating \gls{cct}.
Because of this, typically, \gls{cct} code is written in assembly.
Thanks to DSLs like FaCT~\cite{cauligi2019fact}, however, this is not necessary.

But, FaCT's compiler gives no guarantees with regards to robustness.
That is, the code may not be \gls{cct} when linked with untrusted third-party libraries.
%For example, a simple table lookup may be fetched from one of many cache layers.
%The higher up layers lead to slower access times which is detectable from just observing the program's %execution time.
%With that, an attacker can identify the entry in the table that we wanted to access.
%A \gls{cct} table lookup has to go through all elements of that table in the same order.
%
Compilers giving such a guarantee are known to be robustly preserving~\cite{abate2019jour}.
These \emph{secure compilers} give a strong formal guarantee for security, since an arbitrary target-level attacker has no chance of violating our property of interest, such as \gls{cct}.\MK{cite CCT compiler that is robustly preserving}

For code to be \gls{cct}, \gls{ms} is a prerequisite, since information is easily leaked in a memory-unsafe world by just reading out of bounds.
The importance of \gls{ms} has given rise to lots of possible countermeasures.
Particularily interesting for this paper are \emph{compiler-based} countermeasures like CETS~\cite{nagarakatte2010cets}, SoftBound~\cite{nagarakatte2009soft}, or MSWasm~\cite{michael2023mswasm}, which describes a \emph{robust enforcement} of \gls{ms}.
Given \gls{cct} requires \gls{ms}, it is natural to combine compiler-based techniques to ensure both \gls{cct} and \gls{ms} hold.
Unfortunately, the composition of secure compilers is unclear.

This paper develops the theory of secure compiler compositions and introduces a secure compilation chain for a combination of \gls{ms} and \gls{cct}.
We do this via 4 contributions:

\contribution{\contribnum Composition of Properties (\Cref{sec:compprop})}
We present a formal framework to reason about compositions of safety properties.
First, we assume there is a most general trace semantics in which a property can be defined and that there is a monitor for that property.
The monitor is defined on a restricted trace semantics without unneccessary information.
The composition of trace properties is then modelled by the composition of monitors.
The two-level approach, i.e., the two different trace semantics allow for a modular extension without touching any already existing definition.

\contribution{\contribnum Composition of Secure Compilers (\Cref{sec:compcomp})}
We introduce a formal framework for the composition of secure compilers.
Composition of secure compilers allows for modular proof engineering, since our framework can be used even if the compilers robustly preserve different properties.
A simple corollary of our main composition theorem is the fact that the composition of secure compilers with equal source and target programming lanugages can be freely swapped without compromising security.
In turn, formal compiler engineers can optimize the compilation tool chain with regards to the phase ordering problem.

\contribution{\contribnum Compiler Robustly Preserving \gls{ms} and \gls{cct} (\Cref{sec:casestud})}
We introduce a case-study consisting of three secure compilers.
To this end, we present the language $\src{L_{TMS}}$ which has a type system with linear constraints to enforce \gls{tms} statically.
%We prove that any well-typed $\src{L_{TMS}}$ program is temporal memory safe.
Then, we introduce $\trg{L}$, a language without a type system, and a compiler $\llbracket\src{\bullet}\rrbracket^{\src{L_{TMS}}\to\trg{L}}$ from $\src{L_{TMS}}$ to $\trg{L}$ that robustly preserves \gls{tms}.
To get full \gls{ms}, we define another compiler $\llbracket\trg{\bullet}\rrbracket^{\trg{L}\to\trg{L_{MS}}}$ which introduces bounds-checks and compose it with $\llbracket\src{\bullet}\rrbracket^{\src{L_{TMS}}\to\trg{L}}$.
Lastly, we define $\obj{L_{sCCT}}$, a language with access to a \emph{model specific register} to enable \gls{cct} execution, and a compiler $\llbracket\irl{\bullet}\rrbracket^{\trg{L_{MS}}\to\obj{L_{sCCT}}}$ which we compose with the other ones.
Finally, this gives us a three-step compilation chain with both \gls{ms} and \gls{cct}.


\contribution{\contribnum Coq Formalisation}
Parts of our results are formalized in the Coq proof assistant.
We use the symbol \CoqSymbol to demonstrate the fact that a definition or a proof is done in Coq.
Since our semantics and compiler definitions are extractable from Coq to Ocaml, we can run them.


We now describe the other sections in the paper.
\Cref{sec:background} establishes the notions of traces, (hyper-)properties, and robust preservation, while \Cref{sec:relwork} compares this work with others.
We conclude in \Cref{sec:concl} and give an outlook for future work.

\contribution{Open Source \& Technical Report} Our technical report and the Coq formalisation are
available as supplementary material and will be made available upon acceptance of this paper.


\section{Background}\label{sec:background}
\section{Composing Properties}\label{sec:compprop}
\section{Composing Robustly Secure Compilers}\label{sec:compcomp}
\section{Case Study: Memory Safe and Cryptographic Constant Time Code}\label{sec:casestud}

%\begin{lstlisting}[language=c,caption=``Wrong bounds check of two 64-bit integers.'']
%if at <= bounds {
%  x[at] = 42;
%}
%\end{lstlisting}
%Note the subtle bug of using \verb|<=| instead of \verb|<|, leading to an out-of-bounds access whenever \verb|at = bounds|.
%Furthermore, when \verb|at| and \verb|bounds| are 64-bit integers on a 32-bit architecture, the comparison may not be performed in constant-time: The compiler may bail out as soon as the lower 32 bits are unequal, not bothering to compare the higher 32 bits.


\section{Related Work}\label{sec:relwork}
\section{Conclusion}\label{sec:concl}


%% The acknowledgments section is defined using the "acks" environment
%% (and NOT an unnumbered section). This ensures the proper
%% identification of the section in the article metadata, and the
%% consistent spelling of the heading.
\begin{acks}
\end{acks}

%%
%% The next two lines define the bibliography style to be used, and
%% the bibliography file.
\bibliographystyle{ACM-Reference-Format}
\bibliography{main}

%%
%% If your work has an appendix, this is the place to put it.
\appendix

\end{document}
\endinput
%%
%% End of file `sample-sigconf.tex'.
