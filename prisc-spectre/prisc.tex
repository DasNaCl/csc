%% For double-blind review submission, w/o CCS and ACM Reference (max submission spaceomp
%\documentclass[sigplan,review,anonymous]{acmart}\settopmatter{printfolios=true,printccs=false,printacmref=false}
%% For double-blind review submission, w/ CCS and ACM Reference
%\documentclass[sigplan,review,anonymous]{acmart}\settopmatter{printfolios=true}
%% For single-blind review submission, w/o CCS and ACM Reference (max submission space)
%\documentclass[sigplan,review]{acmart}\settopmatter{printfolios=true,printccs=false,printacmref=false}
%% For single-blind review submission, w/ CCS and ACM Reference
%\documentclass[sigplan,review]{acmart}\settopmatter{printfolios=true}
%% For final camera-ready submission, w/ required CCS and ACM Reference
\documentclass[sigplan,dvipsnames]{acmart}\settopmatter{}
%% "not anonymous"
\settopmatter{printacmref=false} % Removes citation information below abstract
\renewcommand\footnotetextcopyrightpermission[1]{}

%% Conference information
%% Supplied to authors by publisher for camera-ready submission;
%% use defaults for review submission.
\acmConference[PriSC'23]{ACM SIGPLAN Workshop on Principles of Secure Compilation}{January 20nd, 2023}{London, United Kingdom}
\acmYear{2023}
%\acmISBN{} % \acmISBN{978-x-xxxx-xxxx-x/YY/MM}
%\acmDOI{} % \acmDOI{10.1145/nnnnnnn.nnnnnnn}
%\startPage{1} 

%% Copyright information
%% Supplied to authors (based on authors' rights management selection;
%% see authors.acm.org) by publisher for camera-ready submission;
%% use 'none' for review submission.
\setcopyright{none}
%\setcopyright{acmcopyright}
%\setcopyright{acmlicensed}
%\setcopyright{rightsretained}
%\copyrightyear{2022}           %% If different from \acmYear

%% Bibliography style
\bibliographystyle{ACM-Reference-Format}
%% Citation style
%\citestyle{acmauthoryear}  %% For author/year citations
%\citestyle{acmnumeric}     %% For numeric citations
%\setcitestyle{nosort}      %% With 'acmnumeric', to disable automatic
                            %% sorting of references within a single citation;
                            %% e.g., \cite{Smith99,Carpenter05,Baker12}
                            %% rendered as [14,5,2] rather than [2,5,14].
%\setcitesyle{nocompress}   %% With 'acmnumeric', to disable automatic
                            %% compression of sequential references within a
                            %% single citation;
                            %% e.g., \cite{Baker12,Baker14,Baker16}
                            %% rendered as [2,3,4] rather than [2-4].


%%%%%%%%%%%%%%%%%%%%%%%%%%%%%%%%%%%%%%%%%%%%%%%%%%%%%%%%%%%%%%%%%%%%%%
%% Note: Authors migrating a paper from traditional SIGPLAN
%% proceedings format to PACMPL format must update the
%% '\documentclass' and topmatter commands above; see
%% 'acmart-pacmpl-template.tex'.
%%%%%%%%%%%%%%%%%%%%%%%%%%%%%%%%%%%%%%%%%%%%%%%%%%%%%%%%%%%%%%%%%%%%%%

\newcommand\hmmax{0}
\newcommand\bmmax{0}

%% Some recommended packages.
\usepackage[colorinlistoftodos]{todonotes}
\newcommand{\MK}[1]{\todo[color=orange!30]{TODO: #1}}
\newcommand{\MKin}[1]{\todo[inline,color=orange!30]{TODO: #1}}
\newcommand{\MP}[1]{\todo[color=blue!30]{TODO: #1}}
\newcommand{\MPin}[1]{\todo[inline,color=blue!30]{TODO: #1}}

\usepackage{booktabs}   %% For formal tables:
                        %% http://ctan.org/pkg/booktabs
\usepackage{subcaption} %% For complex figures with subfigures/subcaptions
                        %% http://ctan.org/pkg/subcaption
\usepackage{semantic}
\usepackage{stmaryrd}
\usepackage{mathrsfs}
\usepackage{mathtools}
\usepackage{xcolor}
\usepackage{amsmath}
\usepackage{amsthm}
\usepackage{glossaries}
\usepackage{tikz}
\usepackage{cleveref}
\usepackage{listings}
\usepackage{mmmacros}
\usepackage[switch]{lineno}
%\modulolinenumbers[2]
\renewcommand{\linenumberfont}{\normalfont\bfseries\small\color{red}}

\usetikzlibrary{decorations,positioning,arrows}

\newacronym{rc}{RC}{Robust Compilation}
\newacronym{ltlp}{LTL+Past}{Linear Temporal Logic with Past}
\newacronym{imd}{IMD}{Index-Masking Defense}
\newacronym{lfence}{LFENCE}{Insertion of \texttt{lfence}s}
\newacronym{slh}{SLH}{Speculative Load Hardening}
\newacronym{retpoline}{Retpoline}{Return Trampoline}
\newacronym{speccfi}{SpecCFI}{Speculative Control Flow Integrity}
\newacronym{msr}{MSR}{Set Model Specific Register Flags}
\newacronym{pi}{PI}{Process Isolation}
\newacronym{ss}{SS}{Speculative Safety}

\begin{document}

%% Title information
\title{Secure Composition of SPECTRE Mitigations}         %% [Short Title] is optional;
                                        %% when present, will be used in
                                        %% header instead of Full Title.
%\titlenote{with title note}             %% \titlenote is optional;
                                        %% can be repeated if necessary;
                                        %% contents suppressed with 'anonymous'
%\subtitle{Subtitle}                     %% \subtitle is optional
%\subtitlenote{with subtitle note}       %% \subtitlenote is optional;
                                        %% can be repeated if necessary;
                                        %% contents suppressed with 'anonymous'


%% Author information
%% Contents and number of authors suppressed with 'anonymous'.
%% Each author should be introduced by \author, followed by
%% \authornote (optional), \orcid (optional), \affiliation, and
%% \email.
%% An author may have multiple affiliations and/or emails; repeat the
%% appropriate command.
%% Many elements are not rendered, but should be provided for metadata
%% extraction tools.

%% Author with single affiliation.
\author{Matthis Kruse}
%\authornote{with author1 note}          %% \authornote is optional;
                                        %% can be repeated if necessary
\orcid{0000-0003-4062-9666}             %% \orcid is optional
\affiliation{
%  \position{Position1}
%  \department{Department1}              %% \department is recommended
  \institution{CISPA Helmholtz Center for Information Security}            %% \institution is required
%  \streetaddress{Street1 Address1}
%  \city{City1}
%  \state{State1}
%  \postcode{Post-Code1}
  \country{Germany}                    %% \country is recommended
}
\email{matthis.kruse@cispa.de}          %% \email is recommended

%% Author with two affiliations and emails.
\author{Michael Backes}
%\authornote{with author2 note}          %% \authornote is optional;
                                        %% can be repeated if necessary
%\orcid{0000-0003-3411-9678}             %% \orcid is optional
\affiliation{
%  \position{Position2a}
%  \department{Department2a}             %% \department is recommended
  \institution{CISPA Helmholtz Center for Information Security}            %% \institution is required
%  \streetaddress{Street2a Address2a}
%  \city{City2a}
%  \state{State2a}
%  \postcode{Post-Code2a}
  \country{Germany}                   %% \country is recommended
}
\email{director@cispa.de}         %% \email is recommended

%% Abstract
%% Note: \begin{abstract}...\end{abstract} environment must come
%% before \maketitle command
%\begin{abstract}
%Text of abstract \ldots.
%\end{abstract}


%% 2012 ACM Computing Classification System (CSS) concepts
%% Generate at 'http://dl.acm.org/ccs/ccs.cfm'.
%\begin{CCSXML}
%<ccs2012>
%<concept>
%<concept_id>10011007.10011006.10011008</concept_id>
%<concept_desc>Software and its engineering~General programming languages</concept_desc>
%<concept_significance>500</concept_significance>
%</concept>
%<concept>
%<concept_id>10003456.10003457.10003521.10003525</concept_id>
%<concept_desc>Social and professional topics~History of programming languages</concept_desc>
%<concept_significance>300</concept_significance>
%</concept>
%</ccs2012>
%\end{CCSXML}

%\ccsdesc[500]{Software and its engineering~General programming languages}
%\ccsdesc[300]{Social and professional topics~History of programming languages}
%% End of generated code


%% Keywords
%% comma separated list
%\keywords{compilers, security}  %% \keywords are mandatory in final camera-ready submission


%% \maketitle
%% Note: \maketitle command must come after title commands, author
%% commands, abstract environment, Computing Classification System
%% environment and commands, and keywords command.
\maketitle

\linenumbers

\section{Introduction}

Some modern programming languages enjoy strong security guarantees, for example the Rust programming languages has memory safety guarantees given by its compiler performing a semantic analysis pass.
While programmers may expect that these guarantees hold even after translating their program to the target programming language, it has been shown that this expectation is false in the general case~\cite{barthe18}.

Correct compilers do not necessarily provide satisfactory guarantees~\cite{barthe18} and thus one has to resort to secure compilers~\cite{abate19,abate20,patrignani21} which preserve property satisfaction even when the compiled program is linked with arbitrary target-level code, i.e., the compiled program is robust.
A recent framework~\cite{kruse23} describes how to compose secure compilers, thus allowing a divide-and-conquer approach to the engineering of secure compilers. 
This framework primarily focuses on compilers that do not change the traces of the original program.
However, real-world compilers perform source-code transformations that may change the trace, such as applying source-code instrumentations that enhance security by, e.g., inserting bounds-checks. 
Other work~\cite{abate21} showed that there are essentially two approaches to this, where the robust preservation~\cite{abate19,abate20,patrignani21} statement is changed to lift the restriction of a unified trace-model as follows:
{
  \footnotesize
\[
  \begin{array}{l|l}
    \textbf{Top-down Approach} & \textbf{Bottom-up Approach} \\\hline
     \text{if }\src{p}\text{ robustly satisfies }\src{\pi}, & \text{if }\src{p}\text{ robustly satisfies }\sigma_\sim(\trg{\pi}), \\
     \text{then }\gamma(\src{p})\text{ robustly satisfies }\tau_\sim(\src{\pi}) & \text{then }\gamma(\src{p})\text{ robustly satisfies }\trg{\pi}
  \end{array}
\]
}
Hereby, $\gamma$ is a compiler, $\sim$ a cross-language trace relation from $\src{S}$-level traces to $\trg{T}$-level traces describing the semantic effect of $\gamma$, and $\tau_\sim$/$\sigma_\sim$ project a source/target property to a target/source property.
While the compositionality framework~\cite{kruse23} does consider the top-down approach, it does not entail composition of the bottom-up one.
However, the bottom-up approach is crucial for security properties that can only be described in the target-level, such as absence of SPECTRE vulnerabilities~\cite{kocher19}, since higher-level languages do not model speculation in their semantics.

It is worth noting that some compiler compositions may not give the wanted security properties, such as when composing with \gls{imd}~\cite{pizlo18} that prevents SPECTREv1 attacks that exploit speculative execution of loads happening after a branch. 
While \gls{imd} prevents SPECTREv1 attacks, it can introduce SPECTREv4 vulnerabilities and, thus, can render a SPECTREv4 mitigation run prior to \gls{imd} useless.

Compilation passes that do not violate the security properties of earlier ones fulfill a notion of compatibility of a cross-language trace relation that describes the effect of the compiler. 
Because compatiblity is defined on cross-language trace relations, there is no need to reason about the concrete, syntactic changes a compilation pass does. 

\begin{definition}[Compatibility]\label{def:compat}
  Given languages $\src{S}$ and $\trg{T}$, a cross-language trace relation $\sim$ between traces of $\src{S}$ and $\trg{T}$, and a $\trg{T}-$level collection of properties $\trg{\mathbb{C}}$, then $\sim$ is compatible with $\trg{\mathbb{C}}$ iff for any $\trg{\pi}\in\trg{\mathbb{C}}$ it holds that $\sigma_\sim(\trg{\pi})\in\sigma_\sim(\trg{\mathbb{C}})$.
 
%Definition gwfσ {S I : Language}
%  (rel : Trace (Event S) -> Trace (Event I) -> Prop)
%  (C : Class (Event I)) :=
%    forall (Π : Hyperproperty (Event I)),
%      belongs_to (Event I) Π C ->
%      belongs_to (Event S) (σ~ rel Π) (σ~ rel C)
\end{definition}

%Compatibility states that any source-level trace $\src{\trace}$ must have a corresponding target-level trace $\trg{\trace}$ such that the traces are related $\src{\trace}\sim\trg{\trace}$ and $\trg{\trace}$ satisfies a target-level property of interest.

This extended abstract extends prior work~\cite{kruse23} to consider bottom-up secure compiler composition and aims to apply that theory to a selection of mitigations for SPECTRE variants. 
It is demonstrated that it suffices to setup a cross-language trace relation that describes the semantic effect of a secure compiler and prove compatibility with properties of interest in order to compose the secure compiler without giving up on security guarantees. 
%
%
%This work aims to investigate the compatibility, within the framework of recent work\MK{kruse et al}, of source-code instrumentation passes that insert SPECTRE mitigations. 

\section{Composing Secure Compilers}

The composition of secure compilers requires two theorems to be proven:
(1) robust preservation, either with a unified trace-model~\cite{abate19}, top-down, or bottom-up, and (2) \Thmref{def:compat}.
The rest of the paper assumes that the presented SPECTRE mitigations have been proven secure as in (1).
The property that all mitigations aim to robustly preserve is a variant of \gls{ss}~\cite{patrignani21b}, which relies on a taint-tracking mechanism and taints ($\sigma$) that tag events as safe ($S$) or unsafe.
Contrary to the original definition of \gls{ss}, this paper states \gls{ss} such that tags should be unequal to the tag of the kind of variant ($\text{vX}$) that one is interested in:

\begin{definition}[\gls{ss} for variant $\trg{vX}$]\label{def:ss}
  \small$
    \trg{\pi}_{\trg{vX}} = \left\{ \trg{\trace} \mid \forall \trg{\event^\sigma}\in\trg{\trace}, \trg{\sigma} \not= \trg{vX} \right\}
  $
\end{definition}
\noindent The original \gls{ss}~\cite{patrignani21b}, hereby named $\trg{\pi}_{ss}$, can be recovered:
$$\trg{\pi}_{ss} = \displaystyle\bigcap_{\trg{vX}}\trg{\pi}_{\trg{vX}}$$

Robust preservation (only for top-down or bottom-up) and compatiblity (\Cref{def:compat}) require a cross-language trace relation that describes the effect of a corresponding compiler semantically.
Therefore, for the rest of the extended abstract, it is assumed that there are source ($\src{S}$) and target ($\trg{T}$) languages, which share a large portion of their trace model. 
The trace models must have some kind of allocation ($\text{Alloc}{(\loc;n)}$), memory load/store ($\text{Get}{(\loc;idx;n)}$ and $\text{Set}{(\loc;idx;n;v)}$), branch ($\text{If}(b)$), and indirect branch events ($\text{Ibranch}(v)$), jumps ($\text{Jmp}(v)$), as well as a marker event for a barrier ($\times$), and a rollback event ($\text{Rlb}$)~\cite{patrignani21b}.
Trace events are annotated with taint tags and for sake of readability, trace events tagged with the secure tag (S) are written without the tag.
This is enough setup to sketch the cross-language trace relations describing the semantic changes each considered mitigation does:

\begin{itemize}
  \item \acrfull{imd}~\cite{pizlo18} (\texttt{v1})
$$
\begin{array}{rcl}
  \src{\trace} \sim^{v1}_{\texttt{\tiny IMD}} \trg{\trace} & \equiv & \text{if }\src{Alloc(\loc;n)},\src{Get(\loc;idx;v)}\in\src{\trace} \\
                                                           && \text{then } \exists \trg{\loc\ n\ idx\ v}, \trg{Alloc(\loc;n)^\sigma}\in\trg{\trace}\\
                                                           && \text{s.t. } \trg{Get(\loc;idx;v)^{\sigma'}}\in\trg{\trace} \\
                                                           && \text{and } \exists m, \trg{n} = 2^m \text{ and } \trg{idx} \le 2^m \\
                                                           && \text{and } \src{\loc}\approx\trg{\loc} \text{ and }\src{v}\approx\trg{v}
\end{array}
$$
  \gls{imd} changes memory allocation to be powers-of-2 and masks all indices with the bounds of the array.

  \item \gls{lfence}~\cite{taram22} (\texttt{v1}, \texttt{v4})
$$
\begin{array}{rcl}
  \src{\trace} \sim^{v1,v4}_{\texttt{\tiny LFENCE}} \trg{\trace} & \equiv & (\text{if }\forall i,\trg{\trace}[i]=\trg{If(\_)^\sigma}\text{ then }\trg{\trace}[i+1]=\trg{\times^{\sigma'}})\\
                                                                 & \text{and}&(\text{if }\forall i>0,\trg{\trace}[i]=\trg{Get(\_)^{\sigma''}}\\
                                                                 && \text{ then }\trg{\trace}[i-1]=\trg{\times^{\sigma'''}})
\end{array}
$$
  \gls{lfence} simply puts a barrier after any branch or before any load instruction. 
  While the literature provides (partial) solutions that do not insert the barrier \textit{everywhere}, due to the significant performance penalty, the considered pass is simple and puts the barrier ,,everywhere'', i.e., in front of loads and after branches.
  Since the $\src{S}$-level trace is completely irrelevant, this is an example for an enforcement.

%   \item \gls{slh} (\texttt{v1})
% $$
% \src{\trace} \sim^{v1}_{\texttt{\tiny SLH}} \trg{\trace} \equiv ?
% $$

  \item \gls{retpoline}~\cite{turnerret} (\texttt{v2})

\begin{center}
  \typerule{retpol-ibranch}{\src{v}\approx\trg{v} &
    \trg{\trace}=\trg{Set(\loc;sp;v)^\sigma}\cdot\trg{Ret^{\sigma'}}\cdot\trg{\overline{Jmp}^{\overline{\sigma''}}}\cdot\trg{Rlb}\cdot\trg{\trace'}
  }{\src{Ibranch(v)}\cdot\src{\trace}\sim^{v2}_{\texttt{\tiny Retpoline}}\trg{\trace}}{retpol-ibranch}
\end{center}
The \gls{retpoline} applies for every indirect branch on the source-level trace. 
Each indirect branch at source-level must have an associated retpoline at target-level, as sketched with the \cref{tr:retpol-ibranch}. 
That is, the address of the indirect call must be pushed onto the stack to be used in the return instruction and speculation busy waits until the rollback happens.


%   \item \gls{speccfi} (\texttt{v2}, \texttt{v5})
% $$
% \src{\trace} \sim^{v2}_{\texttt{\tiny SpecCFI}} \trg{\trace} \equiv ?
% $$

  \item \gls{msr}~\cite{redhatmsr} (\texttt{\scriptsize v2, v4, v5})
$$
\src{\trace} \sim^{v2,v4,v5}_{\texttt{\tiny MSR}} \trg{\trace} \equiv \forall\trg{\event^\sigma}, \trg{\sigma}\not\in\{\trg{v2},\trg{v4},\trg{v5}\}
$$
Modern processors have flags to turn off speculation features, resulting in complete absence of speculation (for these variants). 
This is another example of an enforcement.

%   \item Hunter (v1, v4)
% $$
% \src{\trace} \sim^{all}_{\texttt{\tiny Hunter}} \trg{\trace} \equiv ?
% $$

%   \item Serberus (all versions)
% $$
% \src{\trace} \sim^{all}_{\texttt{\tiny SERBERUS}} \trg{\trace} \equiv ?
% $$
\end{itemize}

It remains to show \Thmref{def:compat}.
Without a proof of \Thmref{def:compat}, the composition of mitigations may not provide the security guarantees of interest, since one could intuitively ,,undo'' what another one did.
This extended abstract does not provide formal proof for all possible compositions of above mitigations, but sketches the anticipated proofs of compatibility theorems in \Cref{fig:rels}.

\begin{figure}[h]
  \begin{tikzpicture}
    \node (imd) [inner sep = 5pt]{\acrshort{imd}};
    \node[right=1.5 of imd, inner sep = 5pt] (lfence) {\acrshort{lfence}};
    \node[below=1.5 of imd, inner sep = 5pt] (retpoline) {\acrshort{retpoline}};
    \node[right=1.5 of retpoline, inner sep = 5pt] (msr) {\acrshort{msr}};

    \draw[->,thick] (lfence) edge[loop right,right] node {$\trg{\pi_{v1,v4}}$} (lfence);
    \draw[->,thick] (msr) edge[loop right,right] node {$\trg{\pi_{v2,v4,v5}}$} (msr);
    \draw[->,thick] (imd) edge[loop left,left] node {$\trg{\pi_{v1}}$} (imd);
    \draw[->,thick] (retpoline) edge[loop left,left] node {$\trg{\pi_{v2}}$} (retpoline);

    \draw[->,thick] (imd) to[bend left=20,above] node[above] {$\trg{\pi_{v1,v4}}$} (lfence);
    \draw[->,thick] (retpoline) to[bend right=20,below] node[below] {$\trg{\pi_{v2,v4,v5}}$} (msr);
    \draw[<->,thick] (msr) to[bend right=20,right] node[right] {$\trg{\pi}_{ss}$} (lfence);
    \draw[<->,thick] (retpoline) to[bend left=20,below right] node[below = 5.5pt, pos = .9, ] {$\trg{\pi_{v1,v2,v4}}$} (lfence);
    \draw[<->,thick] (msr) to[bend left=20,left] node[above, pos=.15] {$\trg{\pi}_{ss}$} (imd);
    \draw[<->,thick] (retpoline) to[bend left=20,left] node[left] {$\trg{\pi_{v1,v2}}$} (imd);
  \end{tikzpicture}
  \caption{Compatibility of source-code instrumentations to prevent attacks of individual or multiple SPECTRE-variants.
  Nodes are mitigations that perform the respective source-code instrumentation. 
  Edges are directed and represent compatibility of the composition. 
  The origin of an edge is the compiler that should be run first, the target of an edge is the compiler that should be run afterwards.
  Edge labels indicate the \gls{ss} variants.}
  \label{fig:rels}
\end{figure}

% To see how a compatibility proof can work, consider as an example the self-composition of \gls{msr}. 
% As indicated in \Cref{fig:rels}, this should be compatible with $\lceil\trg{\pi_{v2,v4,v5}}\rceil$, the powerset of $\trg{\pi_{v2,v4,v5}}$ representing the lifting to classes of properties. 
% So, by \Thmref{def:compat}, assume $\trg{\pi}\in\lceil\trg{\pi_{v2,v4,v5}}\rceil$ and show that $\sigma_\sim(\trg{\pi})\in\sigma_\sim(\lceil\trg{\pi_{v2,v4,v5}}\rceil)$.
% Unfolding $\sigma_\sim$, let $\trg{\pi'}$ such that $\sigma_\sim(\trg{\pi})=\sigma_\sim(\trg{\pi'})$ and $\trg{\trace}\in\trg{\pi'}$, where the goal now changed to $\trg{\trace}\in\trg{\pi_{v2,v4,v5}}$.
% By $\trg{\pi}\in\lceil\trg{\pi_{v2,v4,v5}}\rceil$, the goal can be rewritten to $\trg{\trace}\in\trg{\pi}$.


% \section{Conclusion}

% This short paper sketched a real-world application of the compositionality framework for secure compilers as presented in earlier work~\cite{abate21,kruse22,kruse23}. 
% While few source-code transformations have been formally proven secure, it is nonetheless interesting to investigate their compatiblity relationships with other transformations on a semantic level, pretending the passes have been proven secure.
% Future work considers more source-code instrumentations, not only on just speculation. 
% Most interestingly, the authors would like to investigate the compatiblity of common compiler optimisations.

\newpage

%% Bibliography
\bibliography{library}
% \bibliography{./../library}


%% Appendix
%\appendix
%\section{Appendix}

%Text of appendix \ldots

\end{document}
