% Fixing: Too many math alphabets used in version normal.
\newcommand\hmmax{0}
\newcommand\bmmax{0}
\documentclass[12pt]{article}
\usepackage[left=1.0cm,top=1.5cm,right=1.0cm,bottom=1.5cm]{geometry}
\usepackage{parskip}
\usepackage{enumitem}
\usepackage[numbers]{natbib}
\usepackage{trimclip}
\makeatletter
\DeclareRobustCommand{\circbullet}{\mathbin{\vphantom{\circ}\text{\circbullet@}}}
\newcommand{\circbullet@}{%
  \check@mathfonts
  \m@th\ooalign{%
    \clipbox{0 0 0 {\dimexpr\height-\fontdimen22\textfont2}}{$\bullet$}\cr
    $\circ$\cr
  }%
}
\DeclareRobustCommand{\bulletcirc}{\mathbin{\text{\bulletcirc@}}}
\newcommand{\bulletcirc@}{%
  \check@mathfonts
  \m@th\ooalign{%
    \raisebox{\fontdimen22\textfont2}{\clipbox{0 {\fontdimen22\textfont2} 0 0}{$\bullet$}}\cr
    $\circ$\cr
  }%
}
\makeatother

% https://tex.stackexchange.com/questions/648845/sans-serif-uppercase-greek-no-longer-showing-in-acmart
\DeclareMathAlphabet{\mathsf}{OT1}{LibertinusSans-LF}{m}{n}
\SetMathAlphabet{\mathsf}{bold}{OT1}{LibertinusSans-LF}{bx}{n}
\DeclareMathAlphabet{\mathtt}{OT1}{lmtt}{m}{n}
\SetMathAlphabet{\mathtt}{bold}{OT1}{lmtt}{bx}{n}

\usepackage[T1]{fontenc}
\usepackage[dvipsnames]{xcolor}
\usepackage{halloweenmath}
\usepackage{fontawesome5}
\usepackage{listofitems}
\usepackage{breakcites}
\usepackage{mathpartir}
\usepackage{glossaries}
\usepackage{mathtools}
\usepackage{tcolorbox}
\usepackage{etoolbox}
\usepackage{graphicx}
\usepackage{stmaryrd}
\usepackage{marvosym}
\usepackage{enumitem}
\usepackage{listings}
\usepackage{hyperref}
\usepackage[nosort]{cleveref}
\usepackage{mdframed}
\usepackage{makecell}
\usepackage{amsmath}
\usepackage{amssymb} %uncomment for ieee
\usepackage{nameref}
\usepackage{xspace}
\usepackage{xfrac}
\usepackage{array}
\usepackage{tikz}
\usepackage{soul}
\usepackage{bm}

\usepackage{noindentafter}

%%%% patch for enumitem
%% https://github.com/jbezos/enumitem/issues/48
\makeatletter
\@namedef{enitkv@enumitem-resume@resume*@default}{%
   \let\enit@resuming\thr@@
  \enit@ifunset{enit@resumekeys@\@currenvir}
    % Nothing to resume if this is the first occurrance of \@currenvir.
    % An empty \enit@resumekeys results in \enitkv@setkeys{enumitem}{,resume}
    % called in \enit@setresume.
    {\def\enit@resumekeys{}}
    {\let\enit@resuming\thr@@
     \expandafter\let\expandafter\enit@resumekeys
       \csname enit@resumekeys@\@currenvir\endcsname
     \@nameuse{enit@resume@\@currenvir}\relax}%
  }
\makeatother
%%%%

%% Line numbering. Remember to use \linenumbers after \begin{document}
\usepackage[right]{lineno}
\renewcommand\thelinenumber{\color{red}\arabic{linenumber}}

%%%%
% TODO macros. not using todonotes package, since it's damn slow
\newcommand{\todo}[2][red]{\colorbox{#1}{\begin{minipage}{0.9\textwidth}{#2}\end{minipage}}}
\newcommand{\MKin}[1]{\todo[orange!30]{MK: #1}}
\newcommand{\MPin}[1]{\todo[blue!30]{MP: #1}}

%% General utility
% List of contributions
\newcounter{contrib}
\newcommand{\contribnum}[0]{\stepcounter{contrib}{\arabic{contrib}}.~}
\newcommand{\contribution}[1]{\smallskip\noindent\textbf{{#1.}\xspace}}

 % fonts
\newcommand{\mi}[1]{\ensuremath{\mathit{#1}}}
\newcommand{\mr}[1]{\ensuremath{\mathrm{#1}}}
\newcommand{\mt}[1]{\ensuremath{\texttt{#1}}}
\newcommand{\mtt}[1]{\ensuremath{\mathtt{#1}}}
\newcommand{\mf}[1]{\ensuremath{\mathbf{#1}}}
\newcommand{\mk}[1]{\ensuremath{\mathfrak{#1}}}
\newcommand{\mc}[1]{\ensuremath{\mathcal{#1}}}
\newcommand{\ms}[1]{\ensuremath{\mathsf{#1}}}
\newcommand{\mb}[1]{\ensuremath{\mathbb{#1}}}
\newcommand{\msc}[1]{\ensuremath{\mathscr{#1}}}

% underlines
\newcommand{\bul}[1]{{\setulcolor{RoyalBlue}\ul{#1}}}
\newcommand{\rul}[1]{{\setulcolor{RedOrange}\ul{#1}}}
\newcommand{\iul}[1]{{\setulcolor{Apricot}\ul{#1}}}
\newcommand{\oul}[1]{{\setulcolor{Emerald}\ul{#1}}}
\newcommand{\pul}[1]{{\setulcolor{CarnationPink}\ul{#1}}}

% Colors
\newcommand{\neutcol}[0]{black}
\newcommand{\stlccol}[0]{RoyalBlue}
\newcommand{\irccol}[0]{Apricot}
\newcommand{\ulccol}[0]{RedOrange}
\newcommand{\objcol}[0]{Emerald} %CarnationPink}
\newcommand{\commoncol}[0]{black}
\newcommand{\irdcol}[0]{CarnationPink}

\newcommand{\col}[2]{\ensuremath{{\color{#1}{#2}}}}

\newcommand{\com}[1]{\ensuremath\mathit{\col{\neutcol}{#1}}}
\newcommand{\src}[1]{\ensuremath\mathsf{\col{\stlccol}{#1}}}
\newcommand{\irl}[1]{\ensuremath\mathit{\col{\irccol}{#1}}}
\newcommand{\trg}[1]{\ensuremath\mathbf{\col{\ulccol}{#1}}}
\newcommand{\obj}[1]{\ensuremath\mathtt{\col{\objcol}{#1}}}
\newcommand{\ird}[1]{\ensuremath\mathit{\col{\irdcol}{#1}}}

%% theorem stuff
\newenvironment{proof}[1][\textbf{Proof}]%
{\everypar={{\setbox0=\lastbox}\everypar{}}\restartlist{passumptionslist}\mbox{\textsc{#1}.}$\;$\noindent\everypar={{\setbox0=\lastbox}\everypar{}}}
{$\;$\hfill$\square$\\\ignorespacesafterend%
\ifx\paslistend\undefined\else\global\let\paslistend\undefined\fi%
\vspace{-2ex}\hrule%
}
\newcommand{\incompleteProof}[1][]{\begin{center}todo #1\end{center}}
% Theorem environments
\Crefname{exampleenv}{Example}{Examples}
\Crefname{definitionenv}{Definition}{Definitions}
\Crefname{lemmaenv}{Lemma}{Lemmata}
\Crefname{theoremenv}{Theorem}{Theorems}
\Crefname{corollaryenv}{Corollary}{Corollaries}
\Crefname{axiomenv}{Axiom}{Axioms}
\newcounter{example}
\newcounter{definition}
\newcounter{lemma}
\newcounter{theorem}
\newcounter{corollary}
\newcounter{axiom}

% A symbol for Coq-verified theorems.
\newcommand{\BareCoqSymbol}{\includegraphics[height=0.9em]{coq.pdf}}
\newcommand{\CoqSymbol}{\raisebox{-.2ex}{\BareCoqSymbol\,}}
\newcommand{\Coqed}{{\hfill\CoqSymbol}}

% defining theorem environments for their specialized versions
\makeatletter
\newenvironment{base@thm}[4][\unskip]%
{\ifx\paslistend\undefined\else\global\let\paslistend\undefined\fi%
  $\;$\linebreak\noindent\mbox{$\triangleright\;\textsc{\textbf{\large #2} #3 (#4) #1.}$}}
{\hfill$\;$\linebreak}
\makeatother

% example environments take a label and a name
\crefname{example}{example}{examples}
\makeatletter
\newenvironment{example}[2][\unskip]%
{\refstepcounter{example}\begin{base@thm}[#1]{Example}{\theexample}{#2}}
{\end{base@thm}\noindent}
% referencing an example
\newcommand{\examplelabel}[2][PLACEHOLDER]{{#1}\label[example]{example:#2}\def\@currentlabel{#1}\label{t@example:#2}}
%\def\@currentlabel{#1}\label{t@example:#2}}
\newcommand{\exampleref}[1]{\Cref{example:#1}~(\ref{t@example:#1})}
\makeatother
% definition environments take a label and a name
\crefname{definition}{definition}{definitions}
\makeatletter
\newenvironment{definition}[2][\unskip]%
{\refstepcounter{definition}\begin{base@thm}[#1]{definition}{\thedefinition}{#2}}
{\end{base@thm}\noindent}
% referencing an definition
\newcommand{\definitionlabel}[2][PLACEHOLDER]{{#1}\label[definition]{definition:#2}\def\@currentlabel{#1}\label{t@definition:#2}}
%\def\@currentlabel{#1}\label{t@definition:#2}}
\newcommand{\definitionref}[1]{\Cref{definition:#1}~(\ref{t@definition:#1})}
\newcommand{\defref}[1]{\Cref{definition:#1}}
\makeatother
% lemma environments take a label and a name
\crefname{lemma}{lemma}{lemmas}
\makeatletter
\newenvironment{lemma}[2][\unskip]%
{\refstepcounter{lemma}\begin{base@thm}[#1]{lemma}{\thelemma}{#2}}
{\end{base@thm}\noindent}
% referencing an lemma
\newcommand{\lemmalabel}[2][PLACEHOLDER]{{#1}\label[lemma]{lemma:#2}\def\@currentlabel{#1}\label{t@lemma:#2}}
%\def\@currentlabel{#1}\label{t@lemma:#2}}
\newcommand{\lemmaref}[1]{\Cref{lemma:#1}~(\ref{t@lemma:#1})}
\newcommand{\lemref}[1]{\Cref{lemma:#1}}
\makeatother
% corollary environments take a label and a name
\crefname{corollary}{corollary}{corollarys}
\makeatletter
\newenvironment{corollary}[2][\unskip]%
{\refstepcounter{corollary}\begin{base@thm}[#1]{corollary}{\thecorollary}{#2}}
{\end{base@thm}\noindent}
% referencing an corollary
\newcommand{\corollarylabel}[2][PLACEHOLDER]{{#1}\label[corollary]{corollary:#2}\def\@currentlabel{#1}\label{t@corollary:#2}}
%\def\@currentlabel{#1}\label{t@corollary:#2}}
\newcommand{\corollaryref}[1]{\Cref{corollary:#1}~(\ref{t@corollary:#1})}
\makeatother
% axiom environments take a label and a name
\crefname{axiom}{axiom}{axioms}
\makeatletter
\newenvironment{axiom}[2][\unskip]%
{\refstepcounter{axiom}\begin{base@thm}[#1]{axiom}{\theaxiom}{#2}}
{\end{base@thm}\noindent}
% referencing an axiom
\newcommand{\axiomlabel}[2][PLACEHOLDER]{{#1}\label[axiom]{axiom:#2}\def\@currentlabel{#1}\label{t@axiom:#2}}
%\def\@currentlabel{#1}\label{t@axiom:#2}}
\newcommand{\axiomref}[1]{\Cref{axiom:#1}~(\ref{t@axiom:#1})}
\makeatother

% proofs
\crefname{asm}{assumption}{assumptions}
\crefname{goal}{goal}{goals}

%%% stack of passumptions 
\newtoks\passumptionsliststack
\passumptionsliststack={\empty}

\def\pushpassumptionsliststack#1#2{%
  \edef\tmp{{#1}\the#2}%
  #2=\expandafter{\tmp}%
}

\def\poppassumptionsliststack#1#2{%
  \expandafter\splitpassumptionsliststack\the#1\stop{#1}{#2}%
}

\def\splitpassumptionsliststack#1#2\stop#3#4{% 
  \def\tmp{#1}%
  \ifx\tmp\empty
  \else
    \def#4{#1}\global#3={#2}%
  \fi
} 
%%% stack of passumptions

\newlist{passumptionslist}{enumerate}{1}
\setlist[passumptionslist]{label=(\alph*)}

\newenvironment{assumptions}
    {\ifx\paslistend\undefined\else\global\let\paslistend\undefined\fi%
      \begin{passumptionslist}[start=1]
    }
    {
    \end{passumptionslist}
  }
\NoIndentAfterEnv{assumptions}
\newenvironment{goals}
    {\begin{enumerate}[label=(\roman*)]
    }
    {
    \end{enumerate}
  }
\NoIndentAfterEnv{goals}

\newcommand{\IH}[1][]{I{\kern-1.5pt}H#1}
\newenvironment{passumptions}[1][H]
  {%\thepassumptionslisti
  \ifx\paslistend\undefined%
    %{\color{red}\Large \thepassumptionslisti}
  \begin{passumptionslist}[resume*,label={$\left(#1_{\arabic*}\right)$}]%
  \else%
    %{\color{red}\Large \thepassumptionslisti === \paslistend}
  \begin{passumptionslist}[resume*,label={$\left(#1_{\arabic*}\right)$},start=\paslistend]%
  \global\let\paslistend\undefined%
  \fi%
  }
  {%
  \end{passumptionslist}%
  }
\NoIndentAfterEnv{passumptions}
% individual cases
\newcommand{\asm}[2]{\item\label[asm]{asm:#1} {$#2$}}
\newcommand{\goal}[2]{\item\label[goal]{goal:#1} {$#2$}}
\newcommand{\asmref}[1]{\Cref{asm:#1}}
\newcommand{\goalref}[1]{\Cref{goal:#1}}
% case environment
\makeatletter
\newenvironment{proofcase}[2][Case ]{%
    \global\edef\passumptionsliststackname{\thepassumptionslisti}%
    \pushpassumptionsliststack{\passumptionsliststackname}{\passumptionsliststack}%
    \par\noindent\ignorespaces%
    {\fbox{\textbf{#1}{$ #2 $}\textbf{:}}} %
    \begin{mdframed}[%
        topline=false,%
        rightline=false,%
        bottomline=false,%
        innertopmargin=0.2em,%
        innerbottommargin=0.4em,%
        innerrightmargin=0.7em,%
        rightmargin=0.7em,%
        innerleftmargin=0.7em,%
        leftmargin=0.7em,%
        linewidth=.15em,%
    ]\@ifpackageloaded{lineno}{\internallinenumbers}{}%
    \setlength{\parindent}{0pt} %
}{%
    \end{mdframed}\ignorespacesafterend%
    \poppassumptionsliststack{\passumptionsliststack}{\passumptionsliststackname}%
    \ifx\paslistend\undefined%
    \global\def\paslistend{\passumptionsliststackname+1}%
    \else%
    \global\def\paslistend{\passumptionsliststackname+1}%
    \fi%
    \setcounter{passumptionslisti}{\passumptionsliststackname}%
    %{\color{red}\Large\paslistend}
}
\makeatother
\NoIndentAfterEnv{proofcase}



% some cool operators
\newcommand{\isdef}[0]{\ensuremath{\mathrel{\overset{\makebox[0pt]{\mbox{\normalfont\tiny\sffamily def}}}{=}}}}

%% PL typesetting stuff 
% Lists
\newcommand{\mklist}[1]{\ensuremath\overline{#1}}
% Box explaining judgement
\newcommand{\textgraybox}[1]{\boxed{#1}}
\newdimen\zzfontsz
\newcommand{\fontsz}[2]{\zzfontsz=#1%
{\fontsize{\zzfontsz}{1.2\zzfontsz}\selectfont{#2}}}
\newcommand{\mathsz}[2]{\text{\fontsz{#1}{$#2$}}}
\newcommand{\instsymColon}{%
     \raisebox{-0.09ex}{\text{\normalfont{:}}}}
\newcommand{\judgboxfontsize}[1]{%
        \mathsz{11pt}{#1}%
}
\newcommand{\judgbox}[2]{%
      {\hrulefill\raggedright \textgraybox{\ensuremath{\judgboxfontsize{#1}}}\!\;%
        \fontsz{9pt}{\begin{tabular}[c]{l} #2 \end{tabular}\hrulefill} %
}}
\newcommand{\judgboxb}[2]{%
  \judgbox{#1}{#2}\hspace{1ex}\hrulefill%
}
\mdfdefinestyle{judgframe}{%
  topline=false, %
  innertopmargin=-2.84ex, %
  innerleftmargin=-.1ex, %
  innerrightmargin=0ex, %
  linewidth=.5pt %
}
\newcounter{judgements}
\crefname{judgements}{judgement}{judgements}
\newcommand{\judgementsref}[1]{\Cref{judgements:#1}}
\newenvironment{judgframe}[4][-2.84ex]
    {\begin{mdframed}[style=judgframe,innertopmargin=#1]\noindent%
      \judgbox{#3}{,,#4''}\def\thejudgements{\ensuremath{#3}}\refstepcounter{judgements}\label[judgements]{judgements:#2}\hrulefill\\[-0.9mm]%
      \begin{center}
    }
  {\end{center}%
   \end{mdframed}%
   \noindent%
  }
% Typerules
\newcounter{typerule}
\crefname{typerule}{rule}{rules}
% separator (may be different depending on used package, so its a macro)
\newcommand{\rulesep}{\qquad} %{\qquad\\\xspace}
\newcommand{\rulenewline}{\ensuremath \\\\}
% inductive predicates
\newcommand{\typeruleInt}[5]{%
	\def\thetyperule{#1}%
	\refstepcounter{typerule}%
	\label{tr:#4}%
	%
  %\ensuremath{\begin{array}{c}#5 \inference{#2}{#3}\end{array}}
  \ensuremath{\inferrule{ #2 }{ #3 }\quad #5}
}
\newcommand{\typerule}[4]{%
  \typeruleInt{#1}{#2}{#3}{#4}{\textsf{\scriptsize ({#1})}  }
}
\newcommand{\typerulenolabel}[2]{%
  \ensuremath{\inferrule{ #1 }{ #2 }}
}
\newcommand{\typederivX}[4][Right]{%
  %\ensuremath{\begin{array}{c} \inference{#2}{#3} #1\end{array}}
  \ensuremath{\inferrule*[#1=\textsc{\scriptsize(#2)}]{ #3 }{ #4 }}
}
\newcommand{\typederiv}[4][Right]{\typederivX[#1]{\trref{#2}}{#3}{#4}}
\newcommand{\asmderiv}[4][Right]{%
  %\ensuremath{\begin{array}{c} \inference{#2}{#3} #1\end{array}}
  \ensuremath{\inferrule*[#1=\textsc{\scriptsize(\asmref{#2})}]{ #3 }{ #4 }}
}
\newcommand{\lemderiv}[3]{%
  %\ensuremath{\begin{array}{c} \inference{#2}{#3} #1\end{array}}
  \ensuremath{\inferrule*[Right=\textsc{\scriptsize(\lemref{#1})}]{ #2 }{ #3 }}
}
%%%%%%%
% coinductive predicates
\newcommand{\cotyperuleInt}[5]{%
	\def\thetyperule{#1}%
	\refstepcounter{typerule}%
	\label{tr:#4}%
	%
  %\ensuremath{\begin{array}{c}#5 \inference{#2}{#3}\end{array}}
  \ensuremath{\mprset{fraction={===}}\inferrule{ #2 }{ #3 }~#5}
}
\newcommand{\cotyperule}[4]{%
  \cotyperuleInt{#1}{#2}{#3}{#4}{\textsf{\scriptsize ({#1})}  }
}
\newcommand{\cotyperulenolabel}[3]{%
	\def\thetyperule{#1}%
	\refstepcounter{typerule}%
  %\ensuremath{\begin{array}{c} \inference{#2}{#3}\end{array}}
  \ensuremath{\coinferrrule{ #2 }{ #3 }}
}
\newcommand{\cotyperulederiv}[3]{%
  %\ensuremath{\begin{array}{c} \inference{#2}{#3} #1\end{array}}
  \ensuremath{\mprset{fraction={===}}\inferrule{ #2 }{ #3 }~\textsc{(#1)}}
}
% referencing a rule
\newcommand{\trref}[1]{\Cref{tr:#1}}
\newcommand{\trrefshandler}[1]{,tr:#1}
\DeclareListParser*\forsemicolonlist{;}
\newcommand{\trrefs}[1]{\Cref{\forsemicolonlist\trrefshandler{#1}}}

% sandwiching
\newcommand{\lift}[1]{\ensuremath\lfloor\xspace{#1}\xspace\rfloor}
\newcommand{\hole}[1]{\ensuremath{\left[#1\right]}}
\newcommand{\denot}[1]{\ensuremath\left\llbracket#1\right\rrbracket\xspace}
\newcommand{\lrpars}[1]{\ensuremath\left(#1\right)\xspace}
\newcommand{\lrbrackets}[1]{\ensuremath\left[#1\right]\xspace}
\newcommand{\lrbraces}[1]{\ensuremath\left\{#1\right\}\xspace}
\newcommand{\lrbbraces}[1]{\ensuremath\left\llbracket{#1}\right\rrbracket\xspace}

%\newcommand{\tup}[2]{\ensuremath (#1 %
%  \readlist\myterms{#2}%
%  \foreachitem\x\in\myterms{;\x}%
%  )%
%}

%%%%%%%%%%%%%%%%%%%%%%%%%%%%%%%%
%% Properties Names
\newcommand{\tmssafe}{\text{tms}}
\newcommand{\smssafe}{\text{sms}}
\newcommand{\mssafe}{\text{ms}}
\newcommand{\ctsafe}{\text{ct}}
\newcommand{\scctsafe}{\text{sct}}
\newcommand{\msscctsafe}{\text{mssct}}
\newcommand{\sssafe}{\text{ss}}
\newcommand{\specmssafe}{\text{specms}}


\newcommand{\observer}[1][]{\ensuremath \llbracket\bullet\rrbracket#1}
\newcommand{\observe}[3][]{\ensuremath \llbracket#2\rrbracket#1\lrpars{#3}}
\newcommand{\stronger}{\ensuremath \sqsupseteq}

%%
\newcommand{\emptyevent}{\ensuremath \varepsilon}
\newcommand{\ev}[1]{\ensuremath\ulcorner #1 \urcorner}
\newcommand{\tmsev}[1]{\ensuremath\ulcorner #1 \urcorner\;^{\kern-3pt{\tmssafe}}}
\newcommand{\smsev}[1]{\ensuremath\ulcorner #1 \urcorner\;^{\kern-3pt{\smssafe}}}
\newcommand{\msev}[1]{\ensuremath\ulcorner #1 \urcorner\;^{\kern-3pt{\mssafe}}}
\newcommand{\scctev}[1]{\ensuremath\ulcorner #1 \urcorner\;^{\kern-3pt{\scctsafe}}}
\newcommand{\specev}[1]{\ensuremath\ulcorner #1 \urcorner\;^{\kern-3pt{\sssafe}}}
\newcommand{\seqctev}[1]{\ensuremath\ulcorner #1 \urcorner\;^{\kern-3pt{\text{seq}}}_{\kern-3pt\ctsafe}}
\newcommand{\specctev}[1]{\ensuremath\ulcorner #1 \urcorner\;^{\kern-3pt{\text{spec}}}_{\kern-3pt\ctsafe}}
\newcommand{\seqarchev}[1]{\ensuremath\ulcorner #1 \urcorner\;^{\kern-3pt{\text{seq}}}_{\kern-3pt\text{arch}}}
\newcommand{\specarchev}[1]{\ensuremath\ulcorner #1 \urcorner\;^{\kern-3pt{\text{spec}}}_{\kern-3pt\text{arch}}}
\newcommand{\seqmemev}[1]{\ensuremath\ulcorner #1 \urcorner\;^{\kern-3pt{\text{seq}}}_{\kern-3pt\text{mem}}}
\newcommand{\specmemev}[1]{\ensuremath\ulcorner #1 \urcorner\;^{\kern-3pt{\text{spec}}}_{\kern-3pt\text{mem}}}

\newcommand{\specfenceev}[1]{\ensuremath\ulcorner \trg{#1} \urcorner\;^{\kern-3pt{\text{spec}}}_{\kern-3pt\text{fence}}}
\newcommand{\specslhev}[1]{\ensuremath\ulcorner \trg{#1} \urcorner\;^{\kern-3pt{\text{spec}}}_{\kern-3pt\text{slh}}}


\newcommand{\lock}{\ensuremath\text{\scriptsize\faIcon{lock}}}
\newcommand{\unlock}{\ensuremath\text{\scriptsize\faIcon{lock-open}}}

\newcommand{\emptyTrace}{\ensuremath\hole{\cdot}}
\newcommand{\consTrace}[2]{\ensuremath{#1},{#2}}

\newcommand{\xlangrel}[2]{\ensuremath\sim^{#1}_{#2}}
\newcommand{\xlangreltr}[2]{\ensuremath\approx^{#1}_{#2}}

\newcommand{\seqctTOspecct}[2]{\xlangrel{seqct\kern2pt\lrpars{#1}}{specct\kern2pt\lrpars{#2}}}
\newcommand{\seqctTOspeccttr}[2]{\xlangreltr{seqct\kern2pt\lrpars{#1}}{specct\kern2pt\lrpars{#2}}}
\newcommand{\seqmemTOspecmem}[2]{\xlangrel{seqmem\kern2pt\lrpars{#1}}{specmem\kern2pt\lrpars{#2}}}
\newcommand{\seqmemTOspecmemtr}[2]{\xlangreltr{seqmem\kern2pt\lrpars{#1}}{specmem\kern2pt\lrpars{#2}}}
\newcommand{\seqarchTOspecarch}[2]{\xlangrel{seqarch\kern2pt\lrpars{#1}}{specarch\kern2pt\lrpars{#2}}}
\newcommand{\seqarchTOspecarchtr}[2]{\xlangreltr{seqarch\kern2pt\lrpars{#1}}{specarch\kern2pt\lrpars{#2}}}

\newcommand{\specmemTOspecct}{\xlangrel{specmem}{specct}}
\newcommand{\specmemTOspeccttr}{\xlangreltr{specmem}{specct}}
\newcommand{\seqmemTOseqct}{\xlangrel{seqmem}{seqct}}
\newcommand{\seqmemTOseqcttr}{\xlangreltr{seqmem}{seqct}}
\newcommand{\seqmemTOspecct}{\xlangrel{seqmem}{specct}}
\newcommand{\seqmemTOspeccttr}{\xlangreltr{seqmem}{specct}}
\newcommand{\specmemTOseqct}{\xlangrel{specmem}{seqct}}
\newcommand{\specmemTOseqcttr}{\xlangreltr{specmem}{seqct}}

\newcommand{\specctTOspecarch}{\xlangrel{specct}{specarch}}
\newcommand{\specctTOspecarchtr}{\xlangreltr{specct}{specarch}}
\newcommand{\seqctTOseqarch}{\xlangrel{seqct}{seqarch}}
\newcommand{\seqctTOseqarchtr}{\xlangreltr{seqct}{seqarch}}
\newcommand{\seqctTOspecarch}{\xlangrel{seqct}{specarch}}
\newcommand{\seqctTOspecarchtr}{\xlangreltr{seqct}{specarch}}
\newcommand{\specctTOseqarch}{\xlangrel{specct}{seqarch}}
\newcommand{\specctTOseqarchtr}{\xlangreltr{specct}{seqarch}}

\newcommand{\specmemTOspecarch}{\xlangrel{specmem}{specarch}}
\newcommand{\specmemTOspecarchtr}{\xlangreltr{specmem}{specarch}}
\newcommand{\seqmemTOseqarch}{\xlangrel{seqmem}{seqarch}}
\newcommand{\seqmemTOseqarchtr}{\xlangreltr{seqmem}{seqarch}}
\newcommand{\seqmemTOspecarch}{\xlangrel{seqmem}{specarch}}
\newcommand{\seqmemTOspecarchtr}{\xlangreltr{seqmem}{specarch}}
\newcommand{\specmemTOseqarch}{\xlangrel{specmem}{seqarch}}
\newcommand{\specmemTOseqarchtr}{\xlangreltr{specmem}{seqarch}}


\newcommand{\specarchTOspecfence}[2]{\xlangrel{specarch\kern2pt\lrpars{#1}}{specfence\kern2pt\lrpars{#2}}}
\newcommand{\specarchTOspecfencetr}[2]{\xlangreltr{specarch\kern2pt\lrpars{#1}}{specfence\kern2pt\lrpars{#2}}}
\newcommand{\specarchTOspecslh}[2]{\xlangrel{specarch\kern2pt\lrpars{#1}}{specslh\kern2pt\lrpars{#2}}}
\newcommand{\specarchTOspecslhtr}[2]{\xlangreltr{specarch\kern2pt\lrpars{#1}}{specslh\kern2pt\lrpars{#2}}}

%% Language-Specific events
\newcommand{\makeConcreteEvent}[3]{\ensuremath\ev{\textnormal{#1}\;\lrpars{#2}}\;^{\kern-3pt{#3}}}

\newcommand{\msEvent}[1][]{\ensuremath \varEvent[_{\mssafe}#1]}
\newcommand{\msTrace}[1][]{\ensuremath \varTrace[_{\mssafe}#1]}
\newcommand{\msAlloc}[1][\varLoc;n]{\ensuremath\makeConcreteEvent{Alloc}{#1}{\mssafe}}
\newcommand{\msDealloc}[1][\varLoc]{\ensuremath\makeConcreteEvent{Dealloc}{#1}{\mssafe}}
\newcommand{\msUse}[1][\varLoc;n]{\ensuremath\makeConcreteEvent{Use}{#1}{\mssafe}}

\newcommand{\smsEvent}[1][]{\ensuremath \varEvent[_{\smssafe}#1]}
\newcommand{\smsTrace}[1][]{\ensuremath \varTrace[_{\smssafe}#1]}
\newcommand{\smsAlloc}[1][\varLoc;n]{\ensuremath\makeConcreteEvent{Alloc}{#1}{\smssafe}}
\newcommand{\smsDealloc}[1][\varLoc]{\ensuremath\makeConcreteEvent{Dealloc}{#1}{\smssafe}}
\newcommand{\smsUse}[1][\varLoc;m]{\ensuremath\makeConcreteEvent{Use}{#1}{\smssafe}}

\newcommand{\tmsEvent}[1][]{\ensuremath \varEvent[_{\tmssafe}#1]}
\newcommand{\tmsTrace}[1][]{\ensuremath \varTrace[_{\tmssafe}#1]}
\newcommand{\tmsAlloc}[1][\varLoc]{\ensuremath\makeConcreteEvent{Alloc}{#1}{\tmssafe}}
\newcommand{\tmsDealloc}[1][\varLoc]{\ensuremath\makeConcreteEvent{Dealloc}{#1}{\tmssafe}}
\newcommand{\tmsUse}[1][\varLoc]{\ensuremath\makeConcreteEvent{Use}{#1}{\tmssafe}}

\newcommand{\scctEvent}[1][]{\ensuremath \varEvent[_{\scctsafe}#1]}
\newcommand{\scctTrace}[1][]{\ensuremath \varTrace[_{\scctsafe}#1]}
\newcommand{\scctAny}[1][\varSecuritytag]{\ensuremath\makeConcreteEvent{Any}{#1}{\scctsafe}}

\newcommand{\specEvent}[1][]{\ensuremath \varEvent[_{\mathghost}#1]}
\newcommand{\specTrace}[1][]{\ensuremath \varTrace[_{\mathghost}#1]}
\newcommand{\specAny}[1][\varSecuritytag]{\ensuremath\makeConcreteEvent{Any}{#1}{\mathghost}}

%% Language-Specific
\newcommand{\ctxtocomp}{\xspace ? \xspace}
\newcommand{\comptoctx}{\xspace ! \xspace}

\newcommand{\varLoc}[1][]{\ensuremath \ell_{#1}}
\newcommand{\varSecuritytag}[1][]{\ensuremath\sigma_{#1}}

\newcommand{\varWholeProg}[1][]{\ensuremath w_{#1}}
\newcommand{\varComponent}[1][]{\ensuremath p_{#1}}
\newcommand{\varContext}[1][]{\ensuremath C_{#1}}
\newcommand{\varRuntimeTerm}[1][]{\ensuremath r_{#1}}

\newcommand{\varEvent}[1][]{\ensuremath a_{#1}}
\newcommand{\varTrace}[1][]{\ensuremath \overline{a_{#1}}}
\newcommand{\varProperty}[1][]{\ensuremath \pi_{#1}}
\newcommand{\varHyperProperty}[1][]{\ensuremath \Pi_{#1}}
\newcommand{\varClass}[1][]{\ensuremath \mathbb{C}_{#1}}

\newcommand{\seqctEvent}[1][]{\ensuremath a^{\text{seq}}_{\text{ct}}\kern-3pt{#1}}
\newcommand{\specctEvent}[1][]{\ensuremath a^{\text{spec}}_{\text{ct}}\kern-3pt{#1}}
\newcommand{\seqmemEvent}[1][]{\ensuremath a^{\text{seq}}_{\text{mem}}\kern-3pt{#1}}
\newcommand{\specmemEvent}[1][]{\ensuremath a^{\text{spec}}_{\text{mem}}\kern-3pt{#1}}
\newcommand{\seqarchEvent}[1][]{\ensuremath a^{\text{seq}}_{\text{arch}}\kern-3pt{#1}}
\newcommand{\specarchEvent}[1][]{\ensuremath a^{\text{spec}}_{\text{arch}}\kern-3pt{#1}}

\newcommand{\specfenceEvent}[1][]{\ensuremath \trg{a}^{\text{spec}}_{\text{fence}}\kern-3pt{#1}}
\newcommand{\specslhEvent}[1][]{\ensuremath \trg{a}^{\text{spec}}_{\text{slh}}\kern-3pt{#1}}

\newcommand{\seqctTrace}[1][]{\ensuremath \overline{a^{\text{seq}}_{\text{ct}}\kern-3pt{#1}}}
\newcommand{\specctTrace}[1][]{\ensuremath \overline{a^{\text{spec}}_{\text{ct}}\kern-3pt{#1}}}
\newcommand{\seqmemTrace}[1][]{\ensuremath \overline{a^{\text{seq}}_{\text{mem}}\kern-3pt{#1}}}
\newcommand{\specmemTrace}[1][]{\ensuremath \overline{a^{\text{spec}}_{\text{mem}}\kern-3pt{#1}}}
\newcommand{\seqarchTrace}[1][]{\ensuremath \overline{a^{\text{seq}}_{\text{arch}}\kern-3pt{#1}}}
\newcommand{\specarchTrace}[1][]{\ensuremath \overline{a^{\text{spec}}_{\text{arch}}\kern-3pt{#1}}}

\newcommand{\specfenceTrace}[1][]{\ensuremath \trg{\overline{a}}^{\text{spec}}_{\text{fence}}\kern-3pt{#1}}
\newcommand{\specslhTrace}[1][]{\ensuremath \trg{\overline{a}}^{\text{spec}}_{\text{slh}}\kern-3pt{#1}}


\newcommand{\varMonitor}[1][]{\ensuremath T_{#1}}
\newcommand{\tmsMonitor}[1][]{\ensuremath\varMonitor[\tmssafe]#1}
\newcommand{\smsMonitor}[1][]{\ensuremath\varMonitor[\smssafe]#1}
\newcommand{\msMonitor}[1][]{\ensuremath\varMonitor[\mssafe]#1}
\newcommand{\scctMonitor}[1][]{\ensuremath\varMonitor[\scctsafe]#1}
\newcommand{\msscctMonitor}[1][]{\ensuremath\varMonitor[\msscctsafe]#1}
\newcommand{\specMonitor}[1][]{\ensuremath\varMonitor[\sssafe]#1}
\newcommand{\specmsMonitor}[1][]{\ensuremath\varMonitor[\specmssafe]#1}
\newcommand{\dualMonitor}[1][]{\ensuremath\varMonitor[A;B]#1}

\newcommand{\noncrashMonitor}[1]{\ensuremath{{\,^\circ}#1}}

% language specific versions of above
%% src
\newcommand{\srcWholeProg}[1][]{\ensuremath \src{\varWholeProg[#1]}}
\newcommand{\srcComponent}[1][]{\ensuremath \src{\varComponent[#1]}}
\newcommand{\srcContext}[1][]{\ensuremath \src{\varContext[#1]}}
\newcommand{\srcRuntimeTerm}[1][]{\ensuremath \src{\varRuntimeTerm[#1]}}

\newcommand{\srcEvent}[1][]{\ensuremath \src{\varEvent[#1]}}
\newcommand{\srcTrace}[1][]{\ensuremath \src{\varTrace[#1]}}
\newcommand{\srcProperty}[1][]{\ensuremath \src{\varProperty[#1]}}
\newcommand{\srcHyperProperty}[1][]{\ensuremath \src{\varHyperProperty[#1]}}
\newcommand{\srcClass}[1][]{\ensuremath \src{\varClass[#1]}}
%% trg
\newcommand{\trgWholeProg}[1][]{\ensuremath \trg{\varWholeProg[#1]}}
\newcommand{\trgComponent}[1][]{\ensuremath \trg{\varComponent[#1]}}
\newcommand{\trgContext}[1][]{\ensuremath \trg{\varContext[#1]}}
\newcommand{\trgRuntimeTerm}[1][]{\ensuremath \trg{\varRuntimeTerm[#1]}}

\newcommand{\trgEvent}[1][]{\ensuremath \trg{\varEvent[#1]}}
\newcommand{\trgTrace}[1][]{\ensuremath \trg{\varTrace[#1]}}
\newcommand{\trgProperty}[1][]{\ensuremath \trg{\varProperty[#1]}}
\newcommand{\trgHyperProperty}[1][]{\ensuremath \trg{\varHyperProperty[#1]}}
\newcommand{\trgClass}[1][]{\ensuremath \trg{\varClass[#1]}}
%% irl
\newcommand{\irlWholeProg}[1][]{\ensuremath \irl{\varWholeProg[#1]}}
\newcommand{\irlComponent}[1][]{\ensuremath \irl{\varComponent[#1]}}
\newcommand{\irlContext}[1][]{\ensuremath \irl{\varContext[#1]}}
\newcommand{\irlRuntimeTerm}[1][]{\ensuremath \irl{\varRuntimeTerm[#1]}}

\newcommand{\irlEvent}[1][]{\ensuremath \irl{\varEvent[#1]}}
\newcommand{\irlTrace}[1][]{\ensuremath \irl{\varTrace[#1]}}
\newcommand{\irlProperty}[1][]{\ensuremath \irl{\varProperty[#1]}}
\newcommand{\irlHyperProperty}[1][]{\ensuremath \irl{\varHyperProperty[#1]}}
\newcommand{\irlClass}[1][]{\ensuremath \irl{\varClass[#1]}}

% bops
\newcommand{\bopLink}[2]{\ensuremath {#1}\bowtie{#2}}
\newcommand{\fncompo}[2]{\ensuremath {#1}\circ{#2}}

% different kinds of satifsaction
\newcommand{\sat}[2]{\ensuremath \vdash {#1} : {#2}}
\newcommand{\rsat}[2]{\ensuremath \vdash_R {#1} : {#2}}

\newcommand{\rtp}[2]{\ensuremath\;\vdash{#1}:{#2}}
\newcommand{\rtpUniversal}[3]{\ensuremath\;\vdash^{\forall}_{{#3}}{#1}:{#2}}
\newcommand{\rtpExistential}[3]{\ensuremath\;\vdash^{\exists}_{{#3}}{#1}:{#2}}


\newcommand{\progstepto}[3]{\ensuremath{#1}\xRightarrow{#3}{#2}}


\newcommand{\cc}[3][\gamma]{\ensuremath {#1}^{#2}_{#3}}
\newcommand{\ccST}[1][\gamma]{\ensuremath \cc[#1]{\src{S}}{\trg{T}}}
\newcommand{\ccSI}[1][\gamma]{\ensuremath \cc[#1]{\src{S}}{\irl{I}}}
\newcommand{\ccIT}[1][\gamma]{\ensuremath \cc[#1]{\irl{I}}{\trg{T}}}
\newcommand{\ccII}[1][\gamma]{\ensuremath \cc[#1]{\irl{I}}{\irl{I}}}


\newcommand{\mapUniversal}[2]{\ensuremath \overset{#1}{\varsigma}\left({#2}\right)}
\newcommand{\mapExistential}[2]{\ensuremath \overset{#1}{\tau}\left({#2}\right)}
\newcommand{\xrelTraces}[2]{\ensuremath {#1}\sim{#2}}


\newcommand{\mapUniversalWF}[2]{\ensuremath \vdash {#1} : \mathit{WF}^{\forall}_{#2}}
\newcommand{\mapExistentialWF}[2]{\ensuremath \vdash {#1}\ ; \mathit{WF}^{\exists}_{#2}}


% Reductions
\usetikzlibrary{calc,decorations.pathmorphing,shapes,positioning}

\newcounter{sarrow}
\newcommand\xrsquigarrow[1]{%
\stepcounter{sarrow}%
\mathrel{\begin{tikzpicture}[baseline= {( $ (current bounding box.south) + (0,-0.5ex) $ )}]
\node[inner sep=.5ex] (\thesarrow) {$\scriptstyle #1$};
\path[draw,<-,decorate,
  decoration={zigzag,amplitude=0.7pt,segment length=1.2mm,pre=lineto,pre length=4pt}] 
    (\thesarrow.south east) -- (\thesarrow.south west);
\end{tikzpicture}}%
}
\newcommand{\monitorcheck}[4][{\kern-3.5pt}^*]{%
  \vdash\xspace{#2}\xspace \xrsquigarrow{#4}{#1}\xspace{#3}\xspace%
}




\newcommand{\ccseqct}{\ensuremath\llbracket\cdot\rrbracket^{\text{seq}}_{\text{ct}}}
\newcommand{\ccseqmem}{\ensuremath\llbracket\cdot\rrbracket^{\text{seq}}_{\text{mem}}}
\newcommand{\ccspecmem}{\ensuremath\llbracket\cdot\rrbracket^{\text{spec}}_{\text{mem}}}
\newcommand{\ccspecct}{\ensuremath\llbracket\cdot\rrbracket^{\text{spec}}_{\text{ct}}}
\newcommand{\ccspecpathct}{\ensuremath\llbracket\cdot\rrbracket^{\text{spec}}_{\text{ct}}}
\newcommand{\ccseqarch}{\ensuremath\llbracket\cdot\rrbracket^{\text{seq}}_{\text{arch}}}
\newcommand{\ccspecarch}{\ensuremath\llbracket\cdot\rrbracket^{\text{spec}}_{\text{arch}}}
\newcommand{\ccseqspecctpc}{\ensuremath\llbracket\cdot\rrbracket^{\text{seq-spec}}_{\text{ct-pc}}}
\newcommand{\ccbot}{\ensuremath\llbracket\cdot\rrbracket_\top}

\newcommand{\partialsec}{\ensuremath\circbullet}
\newcommand{\fullsec}{\ensuremath\bullet}

\newcommand{\proven}{\ensuremath\checkmark}
\newcommand{\partiallyproven}{\ensuremath(\checkmark)}
\newcommand{\informal}{\ensuremath\times}

\newcommand{\specpht}{\ensuremath\text{PHT}}
\newcommand{\specssb}{\ensuremath\text{SSB}} %is this relevant?
\newcommand{\specrsb}{\ensuremath\text{RSB}}
\newcommand{\specbtb}{\ensuremath\text{BTB}}
\newcommand{\specstl}{\ensuremath\text{STL}}
\newcommand{\specpsf}{\ensuremath\text{PSF}}

\loadglsentries{acronyms}
\makeglossaries

\begin{document}


\begin{center}
  \begin{tabular}{|c|c|c|c|c|c|c|}
    \hline
    Ref. &
    Cont. &
    Variants &
    Mitigations &
    Proof &
    Security &
    Platform
    \\\hline
    %
    \cite{vassena2021blade} &
    \acrshort{ccspecct} &
    $\specpht,\specrsb$ &
    \acrshort{fi}, \acrshort{slh} &
    $\partiallyproven$ &
    $\fullsec$ &
    \acrshort{s}
    \\
    %
    \cite{elatali2024cmr} &
    \acrshort{ccspecct} &
    $\specpht,\specrsb$ &
    \acrshort{cmr} &
    $\informal$ &
    $\partialsec$ &
    \acrshort{hs}
    \\
    %
    \cite{yu2019stt} &
    \acrshort{ccspecarch} &
    $\specpht$ &
    \acrshort{dtt} &
    $\partiallyproven$ &
    $\partialsec$ &
    \acrshort{h}
    \\
    %
    \cite{lfence} &
    ? &
    $\specpht$ &
    \acrshort{fi} &
    $\informal$ &
    $\fullsec$ &
    \acrshort{s}
    \\
    %
    \cite{lfence} &
    ? &
    $\specssb$ &
    \acrshort{msr} &
    $\informal$ &
    $\fullsec$ &
    \acrshort{s}/\acrshort{h}
    \\
    %
    \cite{mosier2023serberus} &
    \acrshort{ccspecct} &
    $\specpht$, $\specbtb$, $\specrsb$, $\specstl$, $\specpsf$ &
    \acrshort{fi},\acrshort{iso},\acrshort{rz} &
    $\partiallyproven$ &
    $\fullsec$ &
    \acrshort{s}
    \\
    %
    \cite{narayan2021swivel} &
    \acrshort{ccspecmem} &
    $\specpht$, $\specbtb$, $\specrsb$, $\specstl$, $\specpsf$ &
    \acrshort{iso},\acrshort{sm},\acrshort{ri} &
    $\informal$ &
    $\fullsec$ &
    \acrshort{s}/\acrshort{h}
    \\
    %
    \cite{barber2019specshield} &
    \acrshort{ccspecmem} &
    $\specpht$, $\specbtb$, $\specssb$, $\specstl$ &
    - &
    $\informal$ &
    $\partialsec$ &
    \acrshort{hs}
    \\
    %
    \cite{weisse2019nda} &
    ? &
    $\specpht$, $\specbtb$, $\specrsb$, $\specssb$, $\specstl$ &
    - &
    $\informal$ &
    $\fullsec$ &
    \acrshort{hs}
    \\
    %
    \cite{schwarz2019context} &
    ? &
    $\specpht$, $\specbtb$, $\specrsb$, $\specssb$, $\specstl$ &
    \acrshort{ntmmap} &
    $\informal$ &
    $\fullsec$ &
    \acrshort{s}/\acrshort{h}
    \\
    %
    \cite{khasawneh2018safespec} &
    \acrshort{ccspecct} &
    $\specpht$, $\specbtb$ &
    \acrshort{sm} &
    $\informal$ &
    $\fullsec$ &
    \acrshort{h}
    \\
    %
    % authors claim compositionality with fencing and sandboxing
    % however, it's not robust: they assume all code to follow a certain scheme
    \cite{shen2019venkman} &
    \acrshort{ccspecct} &
    $\specbtb$, $\specrsb$ &
    \acrshort{fi},\acrshort{iso} &
    $\informal$ &
    $\fullsec$ &
    \acrshort{s}
    \\
    %
    % claim it combines well with safespec and invisspec (both are hardware)
    \cite{koruyeh2019speccfi} &
    \acrshort{ccspecct} &
    $\specbtb$, $\specrsb$ &
    \acrshort{cfi} &
    $\informal$ &
    $\fullsec$ &
    \acrshort{s}/\acrshort{h}
    \\
    %
    \cite{yan2018invisispec} &
    \acrshort{ccspecct} &
    $\specpht$, $\specbtb$, $\specrsb$, $\specssb$, $\specstl$  &
    \acrshort{iso} &
    $\informal$ &
    $\fullsec$ &
    \acrshort{h}
    \\
    %
    \cite{eleksenko2018bypass} &
    ? &
    $\specpht$ &
    \acrshort{add} &
    $\informal$ &
    ? &
    \acrshort{s}
    \\
    %
    \cite{slh} &
    ? &
    $\specpht$ &
    \acrshort{slh} &
    $\informal$ &
    $\partialsec$ & % may leak data accessed non-speculativey
                    % misses variable time instructions
    \acrshort{s}
    \\
    %
    \cite{patrignani2021exorcising} &
    \acrshort{ccspecct} &
    $\specpht$ &
    \acrshort{sslh} &
    $\proven$ &
    $\partialsec$ & % misses variable time instructions
    \acrshort{s}
    \\
    %
    \cite{zhang2023uslh} &
    \acrshort{ccspecct} &
    $\specpht$ &
    \acrshort{uslh} &
    $\proven$ &
    $\fullsec$ &
    \acrshort{s}
    \\
    %
    \cite{taram2019csf} &
    \acrshort{ccspecct} &
    $\specpht$, $\specrsb$, $\specbtb$ &
    \acrshort{fi} &
    $\informal$ &
    $\fullsec$ &
    \acrshort{h}
    \\
    %
    \hline
  \end{tabular}
\end{center}

\footnotetext{
The above table uses execution modes \gls{seq}, \gls{seq-ooo}, \gls{spec}, and \gls{spec-ooo}.
Moreover, it relies on observers \gls{ctobs}, \gls{pathctobs}, \gls{memobs}, and \gls{archobs}.
}

\clearpage

\section{Preliminaries}

For all traces, we assume that if $\lightning$ occurs, it occurs exactly once and no events follow after.
The $\lightning$ event mimicks a crash.

% \begin{definition}{\definitionlabel[Stronger Observer]{stronger-observer}}
%   An observer $\observer[_1]$ is stronger than an observer $\observer[_2]$ (written $\observer[_1]\stronger\observer[_2]$) iff for any $\observer[_2]$-level trace $\src{\varTrace}$, there are two $\observer[_1]$-level traces $\trg{\varTrace[_1]},\trg{\varTrace[_2]}$ and a property $\trg{\varProperty}$ such that $\trg{\varTrace[_1]}\in\trg{\varProperty}$ but $\trg{\varTrace[_2]}\in\trg{\varProperty}$ while $\src{\varTrace}\sim\trg{\varTrace[_1]}$ and $\src{\varTrace}\sim\trg{\varTrace[_2]}$.
% \end{definition}

\begin{lemma}{\lemmalabel[Left-total and injective $\sim$ induces a Galois Insertion.]{general-induce-galois-insertion}}
  If
  \begin{assumptions}
    \asm{general-induce-galois-insertion-left-total}{\sim\text{ left total}}
    \asm{general-induce-galois-insertion-left-unique}{\sim\text{ left unique}}
  \end{assumptions}
  then
  \begin{goals}
    \goal{general-induce-galois-insertion}{\forall\varProperty,\mapUniversal{\sim}{\mapExistential{\sim}{\varProperty}}=\varProperty}
  \end{goals}
\end{lemma}
\begin{proof}
  % mapUniversal is abstraction
  \newcommand{\lpref}{general-induce-galois-insertion}
  \newcommand{\abstraction}[1]{\mapUniversal{\sim}{#1}}
  \newcommand{\concretization}[1]{\mapExistential{\sim}{#1}}

  Let $\varProperty$ be a set of source-level objects.
  Let $\src{\varEvent[S]}$ be a source-level object. 
  The proof goes by antisymmetry of set-inclusion:
  \begin{proofcase}{\subseteq}
    We know: 
    \begin{passumptions}
      \asm{\lpref-in-acprop}{\src{\varEvent[S]}\in\abstraction{\concretization{\varProperty}}}
    \end{passumptions}
    We need to show:
    \begin{goals}
      \goal{\lpref-in-prop}{\src{\varEvent[S]}\in\varProperty}
    \end{goals}

    From \asmref{\lpref-left-total} we know that there is a $\trg{\varEvent[T]}$ such that:
    \begin{passumptions}
      \asm{\lpref-le-initrel}{\src{\varEvent[S]}\sim\trg{\varEvent[T]}}
    \end{passumptions}
    Specialize \asmref{\lpref-in-acprop} with \asmref{\lpref-le-initrel}, obtaining:
    \begin{passumptions}
      \asm{\lpref-elc}{\trg{\varEvent[T]}\in\concretization{\varProperty}}
    \end{passumptions}
    From \asmref{\lpref-elc}, we know that there is a $\src{\varEvent[S]'}$ such that:
    \begin{passumptions}
      \asm{\lpref-le-rel}{\src{\varEvent[S]'}\sim\trg{\varEvent[T]}}
      \asm{\lpref-in-prop-prime}{\src{\varEvent[S]'}\in\varProperty}
    \end{passumptions}

    Now, \goalref{\lpref-in-prop} follows from \asmref{\lpref-in-prop-prime} by rewriting with \asmref{\lpref-left-unique} whose assumptions are satisfied via \asmref{\lpref-le-initrel} and \asmref{\lpref-le-rel}.
  \end{proofcase}
  \begin{proofcase}{\supseteq}
    We know: 
    \begin{passumptions}
      \asm{\lpref-in-prop}{\src{\varEvent[S]}\in\varProperty}
    \end{passumptions}
    We need to show:
    \begin{goals}
      \goal{\lpref-in-acprop}{\src{\varEvent[S]}\in\abstraction{\concretization{\varProperty}}}
    \end{goals}
    Unfold \goalref{\lpref-in-acprop}, so let $\trg{\varEvent[T]}$ such that:
    \begin{passumptions}
      \asm{\lpref-ge-initrel}{\src{\varEvent[S]}\sim\trg{\varEvent[T]}}
    \end{passumptions}
    What is left to prove is:
    \begin{goals}
      \goal{\lpref-in-propc}{\trg{\varEvent[T]}\in\concretization{\varProperty}}
    \end{goals}
    Simply instantiate the existential in \goalref{\lpref-in-propc} with $\seqarchEvent$, \asmref{\lpref-ge-initrel} and \asmref{\lpref-in-prop} resolve the remaining obligations.
  \end{proofcase}
\end{proof}

\section{Observers}

\[
  \begin{array}{rcl}
  \ccseqct &-&
    \begin{array}{rcl}
      \seqctEvent & = & \seqctev{\emptyevent} \mid %
                        \seqctev{\lightning} \mid %
                        \seqctev{Load\;\varLoc} \mid %
                        \seqctev{Store\;\varLoc} \mid %
                        \seqctev{Pc\;n}
    \end{array} \\
  \ccspecct &-&
    \begin{array}{rcl}
      \specctEvent & = & \specctev{\emptyevent} \mid %
                         \specctev{\lightning} \mid %
                         \specctev{Load\;\varLoc} \mid %
                         \specctev{Store\;\varLoc} \mid %
                         \specctev{Pc\;n} \mid %
                         \specctev{Spec} \mid %
                         \specctev{Rlb} 
    \end{array} \\
  \ccseqmem &-&
    \begin{array}{rcl}
      \seqmemEvent & = & \seqmemev{\emptyevent} \mid %
                         \seqmemev{\lightning} \mid %
                         \seqmemev{Load\;\varLoc} \mid %
                         \seqmemev{Store\;\varLoc}
    \end{array} \\
  \ccspecmem &-&
    \begin{array}{rcl}
      \specmemEvent & = & \specmemev{\emptyevent} \mid %
                          \specmemev{\lightning} \mid %
                          \specmemev{Load\;\varLoc} \mid %
                          \specmemev{Store\;\varLoc} \mid %
                          \specmemev{Spec} \mid %
                          \specmemev{Rlb} 
    \end{array} \\
  \ccseqarch &-&
    \begin{array}{rcl}
      \seqarchEvent & = & \seqarchev{\emptyevent} \mid %
                          \seqarchev{\lightning} \mid %
                          \seqarchev{Load\;\varLoc\;v} \mid %
                          \seqarchev{Store\;\varLoc\;v} \mid %
                          \seqarchev{Pc\;n}
    \end{array} \\
  \ccspecarch &-&
    \begin{array}{rcl}
      \specarchEvent & = & \specarchev{\emptyevent} \mid %
                           \specarchev{\lightning} \mid %
                           \specarchev{Load\;\varLoc\;v} \mid %
                           \specarchev{Store\;\varLoc\;v} \mid %
                           \specarchev{Pc\;n} \mid %
                           \specarchev{Spec} \mid %
                           \specarchev{Rlb} 
    \end{array} \\
  \end{array}
\]

\section{Mappings between Observers}

\subsection{Sequential to Speculative}

All relations presented in this section share the same key idea, so it is mostly repetitive. 
The key insight is to index the relations by natural numbers in order to keep track of the depth of nested speculation and ignore if the depth is non-zero, i.e., if there is speculation.
The trace-level versions have to do minor bookkeeping to count speculation depth, but the counting is done at event-level.

\paragraph{Constant-Time} $\;$\\

\begin{center}
  \judgbox{\seqctspecctlocmap : \seqctev{\varLoc}\to\specctev{\varLoc}}{,,Total function between memory locations across sequential constant-time and speculative constant-time events.''}
  \judgbox{\seqctTOspecct{n}{m} : \seqctEvent\to\specctEvent\to\mathbb{P}}{,,Map sequential constant time to speculative constant time events.\\%
    $n$ represents the current size of speculation depth and $m$ the new size.%
  ''}
  \typerule{seqct-to-specct-empty}{}{
    \seqctev{\emptyevent}\seqctTOspecct{n}{n}\specctev{\emptyevent}
  }{seqct-to-specct-empty}
  %
  \typerule{seqct-to-specct-crash}{}{
    \seqctev{\lightning}\seqctTOspecct{n}{n}\specctev{\lightning}
  }{seqct-to-specct-crash}
  %
  \typerule{seqct-to-specct-pc}{}{
    \seqctev{Pc\;n}\seqctTOspecct{0}{0}\specctev{Pc\;n}
  }{seqct-to-specct-pc}
  %
  \typerule{seqct-to-specct-store}{
    \seqctspecctlocmapsto{\varLoc}{\varLoc}
  }{
    \seqctev{Store\;\varLoc}\seqctTOspecct{0}{0}\specctev{Store\;\varLoc}
  }{seqct-to-specct-store}
  %
  \typerule{seqct-to-specct-load}{
    \seqctspecctlocmapsto{\varLoc}{\varLoc}
  }{
    \seqctev{Load\;\varLoc}\seqctTOspecct{0}{0}\specctev{Load\;\varLoc}
  }{seqct-to-specct-load}
  %
  \typerule{seqct-to-specct-spec}{
  }{
    \emptyevent\seqctTOspecct{n}{1+n}\specctev{Spec}
  }{seqct-to-specct-spec}
  %
  \typerule{seqct-to-specct-rlb}{
  }{
    \emptyevent\seqctTOspecct{1+n}{n}\specctev{Rlb}
  }{seqct-to-specct-rlb}
  %
  \typerule{seqct-to-specct-pc-ign}{}{
    \seqctev{\emptyevent}\seqctTOspecct{1+n}{1+n}\specctev{Pc\;m}
  }{seqct-to-specct-pc-ign}
  %
  \typerule{seqct-to-specct-store-ign}{
    \seqctspecctlocmapsto{\varLoc}{\varLoc}
  }{
    \seqctev{\emptyevent}\seqctTOspecct{1+n}{1+n}\specctev{Store\;\varLoc}
  }{seqct-to-specct-store-ign}
  %
  \typerule{seqct-to-specct-load-ign}{
    \seqctspecctlocmapsto{\varLoc}{\varLoc}
  }{
    \seqctev{\emptyevent}\seqctTOspecct{1+n}{1+n}\specctev{Load\;\varLoc}
  }{seqct-to-specct-load-ign}
\end{center}
\judgbox{\seqctTOspeccttr{n}{m} : \seqctTrace\to\specctTrace\to\mathbb{P}}{,,Map sequential constant time to speculative constant time events.\\
  $n$ represents the current size of speculation depth and $m$ the new size.%
''}
\begin{center}
  \typerule{seqct-to-specct-refl}{}{
    \hole{\cdot}\seqctTOspeccttr{0}{0}\hole{\cdot}
  }{seqct-to-specct-refl}
  %
  \typerule{seqct-to-specct-trans}{
    \seqctEvent\seqctTOspecct{n}{m'}\specctEvent 
    \rulesep
    \seqctTrace\seqctTOspeccttr{m'}{m}\specctTrace
  }{
    \seqctEvent\cdot\seqctTrace\seqctTOspeccttr{n}{m}\specctEvent\cdot\specctTrace
  }{seqct-to-specct-trans}
  %
  \typerule{seqct-to-specct-ignR}{
    \seqctEvent\seqctTOspecct{n}{m'}\specctev{\emptyevent}
    \rulesep
    \seqctTrace\seqctTOspeccttr{m'}{m}\specctTrace
  }{
    \seqctEvent\cdot\seqctTrace\seqctTOspeccttr{n}{m}\specctTrace
  }{seqct-to-specct-ignR}
  %
  \typerule{seqct-to-specct-ignL}{
    \seqctev{\emptyevent}\seqctTOspecct{n}{m'}\specctEvent
    \rulesep
    \seqctTrace\seqctTOspeccttr{m'}{m}\specctTrace
  }{
    \seqctTrace\seqctTOspeccttr{n}{m}\specctEvent\cdot\specctTrace
  }{seqct-to-specct-ignL}
\end{center}

\begin{lemma}{\lemmalabel[$\seqctTOspecct{n}{m}$ is left-unique.]{seqct-to-specct-left-unique}}
  $\seqctTOspecct{n}{m}$ is left-unique.
\end{lemma}
\begin{proof}
  \newcommand{\lpref}{seqct-to-specct-left-unique}
  Let $\seqctEvent$ and $\seqctEvent[']$ with $\specctEvent$ such that:
  \begin{passumptions}
    \asm{\lpref-rel1}{\seqctEvent\seqctTOspecct{n}{m}\specctEvent}
    \asm{\lpref-rel2}{\seqctEvent[']\seqctTOspecct{n}{m}\specctEvent}
  \end{passumptions}
  The goal is:
  \begin{goals}
    \goal{\lpref-eq}{\seqctEvent=\seqctEvent[']}
  \end{goals}

  Invert both \asmref{\lpref-rel1} and \asmref{\lpref-rel2}, then \goalref{\lpref-eq} follows by reflexivity.
\end{proof}
\begin{lemma}{\lemmalabel[$\seqctTOspecct{n}{m}$ is left-total.]{seqct-to-specct-left-total}}
  $\seqctTOspecct{n}{m}$ is left-total.
\end{lemma}
\begin{proof}
  Simple case analysis on the arguments.
\end{proof}
\begin{corollary}{\corollarylabel[$\seqctTOspecct{n}{m}$ induces a Galois Insertion.]{seqct-to-specct-induce-galois-insertion}}
  It holds that:
  \begin{goals}
    \goal{seqct-to-specct-induce-galois-insertion}{\forall\varProperty,\mapUniversal{\sim}{\mapExistential{\sim}{\varProperty}}=\varProperty}
  \end{goals}
\end{corollary}
\begin{proof}
  Immediate from \lemmaref{general-induce-galois-insertion} via \lemmaref{seqct-to-specct-left-total} and \lemmaref{seqct-to-specct-left-unique}.
\end{proof}

\paragraph{Memory} $\;$\\

\begin{center}
  \judgbox{\seqmemspecmemlocmap : \seqmemev{\varLoc}\to\specmemev{\varLoc}}{,,Total function between memory locations across sequential memory and speculative memory events.''}
  \judgbox{\seqmemTOspecmem{n}{m} : \seqmemEvent\to\specmemEvent\to\mathbb{P}}{,,Map sequential memory to speculative memory events.\\%
    $n$ represents the current size of speculation depth and $m$ the new size.%
  ''}
  \typerule{seqmem-to-specmem-empty}{}{
    \seqmemev{\emptyevent}\seqmemTOspecmem{n}{n}\specmemev{\emptyevent}
  }{seqmem-to-specmem-empty}
  %
  \typerule{seqmem-to-specmem-crash}{}{
    \seqmemev{\lightning}\seqmemTOspecmem{n}{n}\specmemev{\lightning}
  }{seqmem-to-specmem-crash}
  %
  \typerule{seqmem-to-specmem-store}{
    \seqmemspecmemlocmapsto{\varLoc}{\varLoc}
  }{
    \seqmemev{Store\;\varLoc}\seqmemTOspecmem{0}{0}\specmemev{Store\;\varLoc}
  }{seqmem-to-specmem-store}
  %
  \typerule{seqmem-to-specmem-load}{
    \seqmemspecmemlocmapsto{\varLoc}{\varLoc}
  }{
    \seqmemev{Load\;\varLoc}\seqmemTOspecmem{0}{0}\specmemev{Load\;\varLoc}
  }{seqmem-to-specmem-load}
  %
  \typerule{seqmem-to-specmem-spec}{
  }{
    \emptyevent\seqmemTOspecmem{n}{1+n}\specmemev{Spec}
  }{seqmem-to-specmem-spec}
  %
  \typerule{seqmem-to-specmem-rlb}{
  }{
    \emptyevent\seqmemTOspecmem{1+n}{n}\specmemev{Rlb}
  }{seqmem-to-specmem-rlb}
  %
  \typerule{seqmem-to-specmem-store-ign}{
    \seqmemspecmemlocmapsto{\varLoc}{\varLoc}
  }{
    \seqmemev{\emptyevent}\seqmemTOspecmem{1+n}{1+n}\specmemev{Store\;\varLoc}
  }{seqmem-to-specmem-store-ign}
  %
  \typerule{seqmem-to-specmem-load-ign}{
    \seqmemspecmemlocmapsto{\varLoc}{\varLoc}
  }{
    \seqmemev{\emptyevent}\seqmemTOspecmem{1+n}{1+n}\specmemev{Load\;\varLoc}
  }{seqmem-to-specmem-load-ign}
\end{center}
\judgbox{\seqmemTOspecmemtr{n}{m} : \seqmemTrace\to\specmemTrace\to\mathbb{P}}{,,Map sequential memory to speculative memory events.\\
  $n$ represents the current size of speculation depth and $m$ the new size.%
''}
\begin{center}
  \typerule{seqmem-to-specmem-refl}{}{
    \hole{\cdot}\seqmemTOspecmemtr{0}{0}\hole{\cdot}
  }{seqmem-to-specmem-refl}
  %
  \typerule{seqmem-to-specmem-trans}{
    \seqmemEvent\seqmemTOspecmem{n}{m'}\specmemEvent 
    \rulesep
    \seqmemTrace\seqmemTOspecmemtr{m'}{m}\specmemTrace
  }{
    \seqmemEvent\cdot\seqmemTrace\seqmemTOspecmemtr{n}{m}\specmemEvent\cdot\specmemTrace
  }{seqmem-to-specmem-trans}
  %
  \typerule{seqmem-to-specmem-ignR}{
    \seqmemEvent\seqmemTOspecmem{n}{m'}\specmemev{\emptyevent}
    \rulesep
    \seqmemTrace\seqmemTOspecmemtr{m'}{m}\specmemTrace
  }{
    \seqmemEvent\cdot\seqmemTrace\seqmemTOspecmemtr{n}{m}\specmemTrace
  }{seqmem-to-specmem-ignR}
  %
  \typerule{seqmem-to-specmem-ignL}{
    \seqmemev{\emptyevent}\seqmemTOspecmem{n}{m'}\specmemEvent
    \rulesep
    \seqmemTrace\seqmemTOspecmemtr{m'}{m}\specmemTrace
  }{
    \seqmemTrace\seqmemTOspecmemtr{n}{m}\specmemEvent\cdot\specmemTrace
  }{seqmem-to-specmem-ignL}
\end{center}

\begin{lemma}{\lemmalabel[$\seqmemTOspecmem{n}{m}$ is left-unique.]{seqmem-to-specmem-left-unique}}
  $\seqmemTOspecmem{n}{m}$ is left-unique.
\end{lemma}
\begin{proof}
  \newcommand{\lpref}{seqmem-to-specmem-left-unique}
  Let $\seqmemEvent$ and $\seqmemEvent[']$ with $\specmemEvent$ such that:
  \begin{passumptions}
    \asm{\lpref-rel1}{\seqmemEvent\seqmemTOspecmem{n}{m}\specmemEvent}
    \asm{\lpref-rel2}{\seqmemEvent[']\seqmemTOspecmem{n}{m}\specmemEvent}
  \end{passumptions}
  The goal is:
  \begin{goals}
    \goal{\lpref-eq}{\seqmemEvent=\seqmemEvent[']}
  \end{goals}

  Invert both \asmref{\lpref-rel1} and \asmref{\lpref-rel2}, then \goalref{\lpref-eq} follows by reflexivity.
\end{proof}
\begin{lemma}{\lemmalabel[$\seqmemTOspecmem{n}{m}$ is left-total.]{seqmem-to-specmem-left-total}}
  $\seqmemTOspecmem{n}{m}$ is left-total.
\end{lemma}
\begin{proof}
  Simple case analysis on the arguments.
\end{proof}
\begin{corollary}{\corollarylabel[$\seqmemTOspecmem{n}{m}$ induces a Galois Insertion.]{seqmem-to-specmem-induce-galois-insertion}}
  It holds that:
  \begin{goals}
    \goal{seqmem-to-specmem-induce-galois-insertion}{\forall\varProperty,\mapUniversal{\sim}{\mapExistential{\sim}{\varProperty}}=\varProperty}
  \end{goals}
\end{corollary}
\begin{proof}
  Immediate from \lemmaref{general-induce-galois-insertion} via \lemmaref{seqmem-to-specmem-left-total} and \lemmaref{seqmem-to-specmem-left-unique}.
\end{proof}

\paragraph{Architecture} $\;$\\


\begin{center}
  \judgbox{\seqarchspecarchlocmap : \seqarchev{\varLoc}\to\specarchev{\varLoc}}{,,Total function between archory locations across sequential architecture and speculative architecture events.''}
  \judgbox{\seqarchTOspecarch{n}{m} : \seqarchEvent\to\specarchEvent\to\mathbb{P}}{,,Map sequential architecture to speculative architecture events.\\%
    $n$ represents the current size of speculation depth and $m$ the new size.%
  ''}
  \typerule{seqarch-to-specarch-empty}{}{
    \seqarchev{\emptyevent}\seqarchTOspecarch{n}{n}\specarchev{\emptyevent}
  }{seqarch-to-specarch-empty}
  %
  \typerule{seqarch-to-specarch-crash}{}{
    \seqarchev{\lightning}\seqarchTOspecarch{n}{n}\specarchev{\lightning}
  }{seqarch-to-specarch-crash}
  %
  \typerule{seqarch-to-specarch-pc}{}{
    \seqarchev{Pc\;n}\seqarchTOspecarch{0}{0}\specarchev{Pc\;n}
  }{seqarch-to-specarch-pc}
  %
  \typerule{seqarch-to-specarch-store}{
    \seqarchspecarchlocmapsto{\varLoc}{\varLoc}
  }{
    \seqarchev{Store\;\varLoc\;v}\seqarchTOspecarch{0}{0}\specarchev{Store\;\varLoc\;v}
  }{seqarch-to-specarch-store}
  %
  \typerule{seqarch-to-specarch-load}{
    \seqarchspecarchlocmapsto{\varLoc}{\varLoc}
  }{
    \seqarchev{Load\;\varLoc\;v}\seqarchTOspecarch{0}{0}\specarchev{Load\;\varLoc\;v}
  }{seqarch-to-specarch-load}
  %
  \typerule{seqarch-to-specarch-spec}{
  }{
    \emptyevent\seqarchTOspecarch{n}{1+n}\specarchev{Spec}
  }{seqarch-to-specarch-spec}
  %
  \typerule{seqarch-to-specarch-rlb}{
  }{
    \emptyevent\seqarchTOspecarch{1+n}{n}\specarchev{Rlb}
  }{seqarch-to-specarch-rlb}
  %
  \typerule{seqarch-to-specarch-pc-ign}{}{
    \seqarchev{\emptyevent}\seqarchTOspecarch{1+n}{1+n}\specarchev{Pc\;m}
  }{seqarch-to-specarch-pc-ign}
  %
  \typerule{seqarch-to-specarch-store-ign}{
    \seqarchspecarchlocmapsto{\varLoc}{\varLoc}
  }{
    \seqarchev{\emptyevent}\seqarchTOspecarch{1+n}{1+n}\specarchev{Store\;\varLoc\;v}
  }{seqarch-to-specarch-store-ign}
  %
  \typerule{seqarch-to-specarch-load-ign}{
    \seqarchspecarchlocmapsto{\varLoc}{\varLoc}
  }{
    \seqarchev{\emptyevent}\seqarchTOspecarch{1+n}{1+n}\specarchev{Load\;\varLoc\;v}
  }{seqarch-to-specarch-load-ign}
\end{center}
\judgbox{\seqarchTOspecarchtr{n}{m} : \seqarchTrace\to\specarchTrace\to\mathbb{P}}{,,Map sequential architecture to speculative architecture events.\\
  $n$ represents the current size of speculation depth and $m$ the new size.%
''}
\begin{center}
  \typerule{seqarch-to-specarch-refl}{}{
    \hole{\cdot}\seqarchTOspecarchtr{0}{0}\hole{\cdot}
  }{seqarch-to-specarch-refl}
  %
  \typerule{seqarch-to-specarch-trans}{
    \seqarchEvent\seqarchTOspecarch{n}{m'}\specarchEvent 
    \rulesep
    \seqarchTrace\seqarchTOspecarchtr{m'}{m}\specarchTrace
  }{
    \seqarchEvent\cdot\seqarchTrace\seqarchTOspecarchtr{n}{m}\specarchEvent\cdot\specarchTrace
  }{seqarch-to-specarch-trans}
  %
  \typerule{seqarch-to-specarch-ignR}{
    \seqarchEvent\seqarchTOspecarch{n}{m'}\specarchev{\emptyevent}
    \rulesep
    \seqarchTrace\seqarchTOspecarchtr{m'}{m}\specarchTrace
  }{
    \seqarchEvent\cdot\seqarchTrace\seqarchTOspecarchtr{n}{m}\specarchTrace
  }{seqarch-to-specarch-ignR}
  %
  \typerule{seqarch-to-specarch-ignL}{
    \seqarchev{\emptyevent}\seqarchTOspecarch{n}{m'}\specarchEvent
    \rulesep
    \seqarchTrace\seqarchTOspecarchtr{m'}{m}\specarchTrace
  }{
    \seqarchTrace\seqarchTOspecarchtr{n}{m}\specarchEvent\cdot\specarchTrace
  }{seqarch-to-specarch-ignL}
\end{center}

\begin{lemma}{\lemmalabel[$\seqarchTOspecarch{n}{m}$ is left-unique.]{seqarch-to-specarch-left-unique}}
  $\seqarchTOspecarch{n}{m}$ is left-unique.
\end{lemma}
\begin{proof}
  \newcommand{\lpref}{seqarch-to-specarch-left-unique}
  Let $\seqarchEvent$ and $\seqarchEvent[']$ with $\specarchEvent$ such that:
  \begin{passumptions}
    \asm{\lpref-rel1}{\seqarchEvent\seqarchTOspecarch{n}{m}\specarchEvent}
    \asm{\lpref-rel2}{\seqarchEvent[']\seqarchTOspecarch{n}{m}\specarchEvent}
  \end{passumptions}
  The goal is:
  \begin{goals}
    \goal{\lpref-eq}{\seqarchEvent=\seqarchEvent[']}
  \end{goals}

  Invert both \asmref{\lpref-rel1} and \asmref{\lpref-rel2}, then \goalref{\lpref-eq} follows by reflexivity.
\end{proof}
\begin{lemma}{\lemmalabel[$\seqarchTOspecarch{n}{m}$ is left-total.]{seqarch-to-specarch-left-total}}
  $\seqarchTOspecarch{n}{m}$ is left-total.
\end{lemma}
\begin{proof}
  Simple case analysis on the arguments.
\end{proof}
\begin{corollary}{\corollarylabel[$\seqarchTOspecarch{n}{m}$ induces a Galois Insertion.]{seqarch-to-specarch-induce-galois-insertion}}
  It holds that:
  \begin{goals}
    \goal{seqarch-to-specarch-induce-galois-insertion}{\forall\varProperty,\mapUniversal{\sim}{\mapExistential{\sim}{\varProperty}}=\varProperty}
  \end{goals}
\end{corollary}
\begin{proof}
  Immediate from \lemmaref{general-induce-galois-insertion} via \lemmaref{seqarch-to-specarch-left-total} and \lemmaref{seqarch-to-specarch-left-unique}.
\end{proof}

\begin{lemma}{\lemmalabel[$\seqarchTOspecarchtr{n}{m}$ is left-unique.]{seqarch-to-specarchtr-left-unique}}
  $\seqarchTOspecarchtr{n}{m}$ is left-unique.
\end{lemma}
\begin{proof}
  \newcommand{\lpref}{seqarch-to-specarchtr-left-unique}
  Let $\seqarchTrace$ and $\seqarchTrace[']$ with $\specarchTrace$ such that:
  \begin{passumptions}
    \asm{\lpref-rel1}{\seqarchTrace\seqarchTOspecarchtr{n}{m}\specarchTrace}
    \asm{\lpref-rel2}{\seqarchTrace[']\seqarchTOspecarchtr{n}{m}\specarchTrace}
  \end{passumptions}
  The goal is:
  \begin{goals}
    \goal{\lpref-eq}{\seqarchTrace=\seqarchTrace[']}
  \end{goals}

  Induce on \asmref{\lpref-rel1} then invert \asmref{\lpref-rel2}. 
  The base-case follows by reflexivity and the induction case follows by inductive hypothesis and \lemmaref{seqarch-to-specarch-left-unique}.
\end{proof}
\begin{lemma}{\lemmalabel[$\seqarchTOspecarchtr{n}{m}$ is left-total.]{seqarch-to-specarchtr-left-total}}
  $\seqarchTOspecarchtr{n}{m}$ is left-total.
\end{lemma}
\begin{proof}
  Let $\seqarchTrace$ be a trace.
  We want to show that there are $n,m,\specarchTrace$ such that $\seqarchTrace\seqarchTOspecarchtr{n}{m}\specarchTrace$.
  Simply induce on $\seqarchTrace$ and use \lemmaref{seqarch-to-specarch-left-total}.
\end{proof}

\begin{corollary}{\corollarylabel[$\seqarchTOspecarchtr{n}{m}$ induces a Galois Insertion.]{seqarch-to-specarchtr-induce-galois-insertion}}
  It holds that:
  \begin{goals}
    \goal{seqarch-to-specarchtr-induce-galois-insertion}{\forall\varProperty,\mapUniversal{\seqarchTOspecarchtr{n}{m}}{\mapExistential{\seqarchTOspecarchtr{n}{m}}{\varProperty}}=\varProperty}
  \end{goals}
\end{corollary}
\begin{proof}
  Immediate from \lemmaref{general-induce-galois-insertion} via \lemmaref{seqarch-to-specarchtr-left-total} and \lemmaref{seqarch-to-specarchtr-left-unique}.
\end{proof}

\subsection{Across Observer Types}

\paragraph{Memory $\leftrightarrow$ Constant-time} $\;$

\begin{center}
  \judgbox{\specmemspecctlocmap : \specmemev{\varLoc}\to\specctev{\varLoc}}{,,Total function between locations across speculative memory and speculative constant-time events.''}
  \judgbox{\specmemTOspecct : \specmemEvent\to\specctEvent\to\mathbb{P}}{,,Map speculative memory to speculative constant-time events.}
  %
  \typerule{specmem-to-specct-empty}{}{
    \specmemev{\emptyevent}\specmemTOspecct\specctev{\emptyevent}
  }{specmem-to-specct-empty}
  %
  \typerule{specmem-to-specct-crash}{}{
    \specmemev{\lightning}\specmemTOspecct\specctev{\lightning}
  }{specmem-to-specct-crash}
  %
  \typerule{specmem-to-specct-load}{
    \specmemspecctlocmapsto{\varLoc}{\varLoc}
  }{
    \specmemev{Load\;\varLoc}\specmemTOspecct\specctev{Load\;\varLoc}
  }{specmem-to-specct-load}
  %
  \typerule{specmem-to-specct-store}{
    \specmemspecctlocmapsto{\varLoc}{\varLoc}
  }{
    \specmemev{Store\;\varLoc}\specmemTOspecct\specctev{Store\;\varLoc}
  }{specmem-to-specct-store}
  %
  \typerule{specmem-to-specct-pcign}{}{
    \specmemev{\emptyevent}\specmemTOspecct\specctev{Pc\;\varLoc}
  }{specmem-to-specct-pcign}
  %
  \typerule{specmem-to-specct-spec}{}{
    \specmemev{Spec}\specmemTOspecct\specctev{Spec}
  }{specmem-to-specct-spec}
  %
  \typerule{specmem-to-specct-rlb}{}{
    \specmemev{Rlb}\specmemTOspecct\specctev{Rlb}
  }{specmem-to-specct-rlb}
\end{center}
\begin{center}
  \judgbox{\specmemTOspeccttr : \specmemTrace\to\specctTrace\to\mathbb{P}}{,,Map speculative memory to speculative constant-time traces.''}
  %
  \typerule{specmem-to-specct-refl}{}{
    \hole{\cdot}\specmemTOspeccttr\hole{\cdot}
  }{specmem-to-specct-refl}
  %
  \typerule{specmem-to-specct-trans}{
    \specmemEvent\specmemTOspecct\specctEvent \rulesep
    \specmemTrace\specmemTOspeccttr\specctTrace
  }{
    \specmemEvent\cdot\specmemTrace\specmemTOspeccttr\specmemEvent\cdot\specmemTrace
  }{specmem-to-specct-trans}
  %
  \typerule{specmem-to-specct-ignR}{
    \specmemEvent\specmemTOspecct\specctev{\emptyevent}
    \rulesep
    \specmemTrace\specmemTOspeccttr\specctTrace
  }{
    \specmemEvent\cdot\specmemTrace\specmemTOspeccttr\specctTrace
  }{specmem-to-specct-ignR}
  %
  \typerule{specmem-to-specct-ignL}{
    \specmemev{\emptyevent}\seqmemTOspecct\specctEvent
    \rulesep
    \specmemTrace\specmemTOspeccttr\specctTrace
  }{
    \specmemTrace\specmemTOspeccttr\specctEvent\cdot\specctTrace
  }{specmem-to-specct-ignL}
\end{center}

Using existing relations, it is now possible to define sequential comparison without loss of precision as follows:

\[
  \begin{array}{rcl}
    \seqmemTrace\seqmemTOspeccttr\specctTrace &\isdef& \exists \specmemTrace\;,
        \seqmemTrace\seqmemTOspecmemtr{0}{0}\specmemTrace
      \;\operatorname{and}\;
        \specmemTrace\specmemTOspeccttr\specctTrace
    \\[1ex]
    \seqmemTrace\seqmemTOseqcttr\seqctTrace &\isdef& \exists \specctTrace\;,
        \seqmemTrace\seqmemTOspeccttr\specctTrace
      \;\operatorname{and}\;
        \seqctTrace\seqctTOspeccttr{0}{0}\specctTrace
  \end{array}
\]

Moreover:
\[
  \specmemTrace\specmemTOseqcttr\seqctTrace \isdef 
    \exists \specctTrace\;,
      \specmemTrace\specmemTOspeccttr\specctTrace
    \;\operatorname{and}\;
      \seqctTrace\seqctTOspeccttr{0}{0}\specctTrace
\]

\paragraph{Constant-time $\leftrightarrow$ Architecture} $\;$

\begin{center}
  \judgbox{\specctspecarchlocmap : \specctev{\varLoc}\to\specarchev{\varLoc}}{,,Total funarchion between locations across speculative constant-time and speculative architecture events.''}
  \judgbox{\specctTOspecarch : \specctEvent\to\specarchEvent\to\mathbb{P}}{,,Map speculative constant-time to speculative architecture events.}
  %
  \typerule{specct-to-specarch-empty}{}{
    \specctev{\emptyevent}\specctTOspecarch\specarchev{\emptyevent}
  }{specct-to-specarch-empty}
  %
  \typerule{specct-to-specarch-crash}{}{
    \specctev{\lightning}\specctTOspecarch\specarchev{\lightning}
  }{specct-to-specarch-crash}
  %
  \typerule{specct-to-specarch-load}{
    \specctspecarchlocmapsto{\varLoc}{\varLoc}
  }{
    \specctev{Load\;\varLoc}\specctTOspecarch\specarchev{Load\;\varLoc\;v}
  }{specct-to-specarch-load}
  %
  \typerule{specct-to-specarch-store}{
    \specctspecarchlocmapsto{\varLoc}{\varLoc}
  }{
    \specctev{Store\;\varLoc}\specctTOspecarch\specarchev{Store\;\varLoc\;v}
  }{specct-to-specarch-store}
  %
  \typerule{specct-to-specarch-pc}{}{
    \specctev{Pc\;n}\specctTOspecarch\specarchev{Pc\;n}
  }{specct-to-specarch-pc}
  %
  \typerule{specct-to-specarch-spec}{}{
    \specctev{Spec}\specctTOspecarch\specarchev{Spec}
  }{specct-to-specarch-spec}
  %
  \typerule{specct-to-specarch-rlb}{}{
    \specctev{Rlb}\specctTOspecarch\specarchev{Rlb}
  }{specct-to-specarch-rlb}
\end{center}
\begin{center}
  \judgbox{\specctTOspecarchtr : \specctTrace\to\specarchTrace\to\mathbb{P}}{,,Map speculative constant-time to speculative architecture traces.''}
  %
  \typerule{specct-to-specarch-refl}{}{
    \hole{\cdot}\specctTOspecarchtr\hole{\cdot}
  }{specct-to-specarch-refl}
  %
  \typerule{specct-to-specarch-trans}{
    \specctEvent\specctTOspecarch\specarchEvent \rulesep
    \specctTrace\specctTOspecarchtr\specarchTrace
  }{
    \specctEvent\cdot\specctTrace\specctTOspecarchtr\specctEvent\cdot\specctTrace
  }{specct-to-specarch-trans}
  %
  \typerule{specct-to-specarch-ignR}{
    \specctEvent\specctTOspecarch\specarchev{\emptyevent}
    \rulesep
    \specctTrace\specctTOspecarchtr\specarchTrace
  }{
    \specctEvent\cdot\specctTrace\specctTOspecarchtr\specarchTrace
  }{specct-to-specarch-ignR}
  %
  \typerule{specct-to-specarch-ignL}{
    \specctev{\emptyevent}\seqctTOspecarch\specarchEvent
    \rulesep
    \specctTrace\specctTOspecarchtr\specarchTrace
  }{
    \specctTrace\specctTOspecarchtr\specarchEvent\cdot\specarchTrace
  }{specct-to-specarch-ignL}
\end{center}

Using existing relations, it is now possible to define sequential comparison without loss of precision as follows:

\[
  \begin{array}{rcl}
    \seqctTrace\seqctTOspecarchtr\specarchTrace &\isdef& \exists \specctTrace\;,
        \seqctTrace\seqctTOspeccttr{0}{0}\specctTrace
      \;\operatorname{and}\;
        \specctTrace\specctTOspecarchtr\specarchTrace
    \\[1ex]
    \seqctTrace\seqctTOseqarchtr\seqarchTrace &\isdef& \exists \specarchTrace\;,
        \seqctTrace\seqctTOspecarchtr\specarchTrace
      \;\operatorname{and}\;
        \seqarchTrace\seqarchTOspecarchtr{0}{0}\specarchTrace
  \end{array}
\]
Moreover:
\[
  \specctTrace\specctTOseqarchtr\seqarchTrace \isdef 
    \exists \specarchTrace\;,
      \specctTrace\specctTOspecarchtr\specarchTrace
    \;\operatorname{and}\;
      \seqarchTrace\seqarchTOspecarchtr{0}{0}\specarchTrace
\]

\paragraph{Memory $\leftrightarrow$ Architecture} $\;$\\
These cases are plugging the previous relations into each other in a straightforward fashion:

\[
  \begin{array}{rcl}
    \seqmemTrace\seqmemTOseqarchtr\seqarchTrace &\isdef&
      \exists \seqctTrace\;,
        \seqmemTrace\seqmemTOseqcttr\seqctTrace
      \;\operatorname{and}\;
        \seqctTrace\seqctTOseqarchtr\seqarchTrace
    \\
    \seqmemTrace\seqmemTOspecarchtr\specarchTrace &\isdef&
      \exists \seqctTrace\;,
        \seqmemTrace\seqmemTOseqcttr\seqctTrace
      \;\operatorname{and}\;
        \seqctTrace\seqctTOspecarchtr\specarchTrace
    \\
    \specmemTrace\specmemTOseqarchtr\seqarchTrace &\isdef&
      \exists \specctTrace\;,
        \specmemTrace\specmemTOspeccttr\specctTrace
      \;\operatorname{and}\;
        \specctTrace\specctTOseqarchtr\seqarchTrace
    \\
    \specmemTrace\specmemTOspecarchtr\specarchTrace &\isdef&
      \exists \specctTrace\;,
        \specmemTrace\specmemTOspeccttr\specctTrace
      \;\operatorname{and}\;
        \specctTrace\specctTOspecarchtr\specarchTrace
    \\
  \end{array}
\]

\subsection{Observer Hierarchy}

\begin{center}
  \begin{tikzpicture}[node distance=.75cm]
    \node[draw] (SpecArch) {$\specarchEvent$};
    \node[draw] (SeqArch)[below right=of SpecArch] {$\seqarchEvent$};

    \node[draw] (SpecCt)[below left=1cm of SpecArch] {$\specctEvent$};
    \node[draw] (SeqCt)[below right=of SpecCt] {$\seqctEvent$};

    \node[draw] (SpecMem)[below left=1cm of SpecCt] {$\specmemEvent$};
    \node[draw] (SeqMem)[below right=of SpecMem] {$\seqmemEvent$};

    \draw[very thick,->] (SpecArch) -- (SeqArch);
    \draw[very thick,->] (SpecCt) -- (SeqCt);
    \draw[very thick,->] (SpecMem) -- (SeqMem);

    \draw[very thick,->] (SpecArch) -- (SpecCt);
    \draw[very thick,->] (SpecCt) -- (SpecMem);
    
    \draw[very thick,dotted,->] (SeqArch) -- (SeqCt);
    \draw[very thick,dotted,->] (SeqCt) -- (SeqMem);
  \end{tikzpicture}
\end{center}

\section{Machines}

\[
\begin{array}{rcl}
  \text{(Fencing)} &-&
    \begin{array}{rcl}
      \specfenceEvent & = & \specfenceev{\emptyevent} \mid
                            \specfenceev{Load\;\varLoc\;v} \mid %
                            \specfenceev{Store\;\varLoc\;v} \mid %
                            \specfenceev{Pc\;n} \\ &\mid&%
                            \specfenceev{Spec} \mid %
                            \specfenceev{Rlb} \mid %
                            \specfenceev{Fence}
    \end{array} \\[2em]
    %
  \text{(Speculative Load Hardening)} &-&
    \begin{array}{rcl}
      \specslhEvent & = & \specslhev{\emptyevent} \mid
                          \specslhev{Load\;\varLoc\;v} \mid %
                          \specslhev{Store\;\varLoc\;v} \mid %
                          \specslhev{Pc\;n} \mid \\ &\mid&%
                          \specslhev{Br\;b\;n} \mid %
                          \specslhev{Spec} \mid %
                          \specslhev{Rlb}
    \end{array} \\
\end{array}
\]

\section{Mappings from Observers to Machines}

Whenever the following trace-relations are applied at top-level, we assume that both sides are filtered with regards to $\emptyevent$.

\paragraph{Memory fencing} $\;$\\

Let $o$ be a stream of ,,$\text{yes}$'' and ,,$\text{no}$'', which associate to each $\specarchEvent$ in a given trace $\specarchTrace$.
The answer $o\lrpars{n}$ corresponds to the event at position $n$ in $\specarchTrace$\;.
\begin{center}
  \judgbox{\specarchspecfencelocmap : \specarchev{\varLoc}\to\specfenceev{\varLoc}}{,,Total function between locations across speculative architecture and speculative fence events.''}
  \judgbox{\specarchTOspecfence{n}{o} : \specarchEvent \to \specfenceTrace \to \mathbb{P}}{,,Map speculative architecture to memory-fencing using oracle $o$ for a given position $n$.''}
  %
  \typerule{specarch-to-fencing-empty}{}{
    \specarchev{\emptyevent}\specarchTOspecfence{n}{o}\specfenceev{\emptyevent}
  }{specarch-to-fencing-empty}
  %
  \typerule{specarch-to-fencing-crash}{}{
    \specarchev{\lightning}\specarchTOspecfence{n}{o}\specfenceev{\lightning}
  }{specarch-to-fencing-crash}
  %
  \typerule{specarch-to-fencing-load-yes}{
    o\lrpars{n} = \text{yes} \rulesep
    \specarchspecfencelocmapsto{\varLoc}{\varLoc}
  }{
    \specarchev{Load\;\varLoc\;v}\specarchTOspecfence{n}{o}\specfenceev{Fence}\cdot\specfenceev{Load\;\varLoc\;v}
  }{specarch-to-fencing-load-yes}
  %
  \typerule{specarch-to-fencing-store-yes}{
    o\lrpars{n} = \text{yes} \rulesep
    \specarchspecfencelocmapsto{\varLoc}{\varLoc}
  }{
    \specarchev{Store\;\varLoc\;v}\specarchTOspecfence{n}{o}\specfenceev{Fence}\cdot \specfenceev{Store\;\varLoc\;v}
  }{specarch-to-fencing-store-yes}
  %
  \typerule{specarch-to-fencing-load-no}{
    o\lrpars{n} = \text{no} \rulesep
    \specarchspecfencelocmapsto{\varLoc}{\varLoc}
  }{
    \specarchev{Load\;\varLoc\;v}\specarchTOspecfence{n}{o}\specfenceev{Load\;\varLoc\;v}
  }{specarch-to-fencing-load-no}
  %
  \typerule{specarch-to-fencing-store-no}{
    o\lrpars{n} = \text{no} \rulesep
    \specarchspecfencelocmapsto{\varLoc}{\varLoc}
  }{
    \specarchev{Store\;\varLoc\;v}\specarchTOspecfence{n}{o}\specfenceev{Store\;\varLoc\;v}
  }{specarch-to-fencing-store-no}
  %
  \typerule{specarch-to-fencing-pc}{}{
    \specarchev{Pc\;n}\specarchTOspecfence{n}{o}\specfenceev{Pc\;n}
  }{specarch-to-fencing-pc}
  %
  \typerule{specarch-to-fencing-spec}{}{
    \specarchev{Spec}\specarchTOspecfence{n}{o}\specfenceev{Spec}
  }{specarch-to-fencing-spec}
  %
  \typerule{specarch-to-fencing-rlb}{}{
    \specarchev{Rlb}\specarchTOspecfence{n}{o}\specfenceev{Rlb}
  }{specarch-to-fencing-rlb}
\end{center}
\begin{center}
  \judgbox{\specarchTOspecfence{n}{o} : \specarchEvent \to \specfenceEvent \to \mathbb{P}}{,,Map speculative architecture to memory-fencing using oracle $o$ for a given position $n$.''}
  %
  \typerule{specarch-to-fencing-refl}{}{
    \hole{\cdot}\specarchTOspecfencetr{n}{o}\trg{\hole{\cdot}}
  }{specarch-to-fencing-refl}
  %
  \typerule{specarch-to-fencing-trans}{
    \specarchEvent\specarchTOspecfence{n}{o}\specfenceTrace \rulesep
    \specarchTrace\specarchTOspecfence{1+n}{o}\specfenceTrace['] 
  }{
    \specarchEvent\cdot\specarchTrace\specarchTOspecfencetr{n}{o}\specfenceTrace\cdot\specfenceTrace[']
  }{specarch-to-fencing-trans}
\end{center}

\paragraph{Speculative Load Hardening} $\;$

Let $s$ be a set of secret memory addresses, i.e., we say $\varLoc$ is secret iff $\varLoc\in s$.
Note that we assume speculative execution to use always-mispredict semantics.

\begin{center}
  \judgbox{\specarchspecslhlocmap : \specarchev{\varLoc}\to\specslhev{\varLoc}}{,,Total function between locations across speculative architecture and speculative slh events.''}
  \judgbox{\specarchTOspecslh{n;s;\trg{b}}{m;\trg{b}} : \specarchEvent \to \specslhTrace \to \mathbb{P}}{,,Map speculative architecture to speculative load hardening. %
  $s$ is a set of secret locations. \\%
  $\trg{b}$ is an optional condition.''}
  %
  \typerule{specarch-to-specslh-empty}{}{
    \specarchev{\emptyevent}\specarchTOspecslh{n;s;\trg{b}}{n;\trg{b}}\specslhev{\emptyevent}
  }{specarch-to-specslh-empty}
  %
  \typerule{specarch-to-specslh-crash}{}{
    \specarchev{\lightning}\specarchTOspecslh{n;s;\trg{b}}{n;\trg{b}}\specslhev{\lightning}
  }{specarch-to-specslh-crash}
  %
  \typerule{specarch-to-specslh-specmode-branch}{}{
    \specarchev{Pc\;n'}\specarchTOspecslh{1+n;s;\trg{b'}}{1+n;\trg{b}}\specslhev{Br\;b\;n'}
  }{specarch-to-specslh-specmode-branch}
  %
  \typerule{specarch-to-specslh-specmode-store}{
    \varLoc\in s \rulesep
    \trg{v}=v\operatorname{bitor}\trg{b} \rulesep
    \specarchspecslhlocmapsto{\varLoc}{\varLoc}
  }{
    \specarchev{Store\;\varLoc\;v}\specarchTOspecslh{1+n;s;\trg{b}}{1+n;\trg{b}}\specslhev{Store\;\varLoc\;v}
  }{specarch-to-specslh-specmode-store}
  %
  \typerule{specarch-to-specslh-specmode-load}{
    \varLoc\in s \rulesep
    \trg{v}=v\operatorname{bitor}\trg{b} \rulesep
    \specarchspecslhlocmapsto{\varLoc}{\varLoc}
  }{
    \specarchev{Load\;\varLoc\;v}\specarchTOspecslh{1+n;s;\trg{b}}{1+n;\trg{b}}\specslhev{Load\;\varLoc\;v}
  }{specarch-to-specslh-specmode-load}
  %
  \typerule{specarch-to-specslh-store}{
    \specarchspecslhlocmapsto{\varLoc}{\varLoc}
  }{
    \specarchev{Store\;\varLoc\;v}\specarchTOspecslh{n;s;\trg{b}}{n;\trg{b}}\specslhev{Store\;\varLoc\;v}
  }{specarch-to-specslh-store}
  %
  \typerule{specarch-to-specslh-load}{
    \specarchspecslhlocmapsto{\varLoc}{\varLoc}
  }{
    \specarchev{Load\;\varLoc\;v}\specarchTOspecslh{n;s;\trg{b}}{n;\trg{b}}\specslhev{Load\;\varLoc\;v}
  }{specarch-to-specslh-load}
  %
  \typerule{specarch-to-specslh-spec}{}{
    \specarchev{Spec}\specarchTOspecslh{n;s;\trg{b}}{1+n;\trg{b}}\specslhev{Spec}
  }{specarch-to-specslh-spec}
  %
  \typerule{specarch-to-specslh-rlb}{}{
    \specarchev{Rlb}\specarchTOspecslh{n;s;\trg{b}}{1+n;\trg{b}}\specslhev{Rlb}
  }{specarch-to-specslh-rlb}
  %
  \typerule{specarch-to-specslh-pc}{}{
    \specarchev{Pc\;n'}\specarchTOspecslh{n;s;\trg{b}}{n;\trg{b}}\specslhev{Pc\;n'}
  }{specarch-to-specslh-pc}
  %
  \typerule{specarch-to-specslh-branch}{}{
    \specarchev{Pc\;n'}\specarchTOspecslh{n;s;\trg{b}}{n;\trg{b}}\specslhev{Branch\;b'\;n'}
  }{specarch-to-specslh-branch}
\end{center}

\begin{center}
  \judgbox{\specarchTOspecslh{n;s;\trg{b}}{m} : \specarchEvent \to \specslhTrace \to \mathbb{P}}{,,Map speculative architecture to speculative load hardening. %
  $s$ is a set of secret locations.''}
  %
  \typerule{specarch-to-specslh-refl}{}{
    \hole{\cdot}\specarchTOspecslhtr{0;s;\trg{\hole{\cdot}}}{0}\hole{\cdot}
  }{specarch-to-specslh-refl}
  %
  \typerule{specarch-to-specslh-trans}{
    \specarchEvent\specarchTOspecslh{n;s;\trg{b}}{n';\trg{b'}}\specslhEvent\rulesep
    \specarchTrace\specarchTOspecslhtr{n';s;\trg{b'}}{m}\specslhTrace
  }{
    \specarchEvent\cdot\specarchTrace\specarchTOspecslhtr{n;s;\trg{b}}{m;\trg{b''}}\specslhEvent\cdot\specslhTrace
  }{specarch-to-specslh-trans}
\end{center}

\section{Clustering}

\printglossary

\clearpage
\bibliographystyle{ACM-Reference-Format}
\bibliography{main}

\end{document}
\endinput
