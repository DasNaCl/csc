\subsection{Properties}\label{subsec-cs-properties}

The properties considered in this case-study are a mixture of trace-properties as welll as hyperproperties.
However, to carry out proofs, all properties are overapproximated by a monitor.

\subsubsection{Formal Definition}\label{subsubsec-cs-properties-formaldef}

For simple memory safety composed of temporal and spatial memory safety, the trace model defines events ($\varEvent[_{\mssafe}]$) as either the empty event ($\emptyevent$), a crash ($\lightning$), or a base-event ($\varEvent[_{\mssafe}]$).

\begin{definition}{\definitionlabel[\glsfirst{ms} Trace Model]{mstraces}}
\[
  \begin{array}{rcll}
    \text{(Base-Events)}&\varEvent[_{\mssafe}] &:=& \msAlloc \mid \msDealloc \mid \msUse \\
    \text{(Events)}&\varEvent[_{\mssafe}] &:=& \varEvent[_{\mssafe}] \mid \emptyevent \mid \lightning \\ 
  \end{array}
\]
\end{definition}

Base-events describe the actual kind of event that happened.
For the basic memory-safety properties, these are three variants:
First, the allocation event ($\msAlloc$) that fires whenever a program claims $n$ cells of memory and stores them at address $\varLoc$, where addresses are assumed to be unique.
Second, deallocation ($\msDealloc$) announces that the object at location $\varLoc$ is freed.
Third, an event to describe reads from and writes to the $n$-th memory cell from address $\varLoc$ ($\msUse$).

\paragraph{Temporal Memory Safety}

\gls*{tms}~\cite{nagarakatte2010cets} is a safety property that describes that an unallocated object must not be (re-)used.
Since it does not care about bounds of an object in memory, we emphasize this fact by deviating from the unified, memory-safety trace model and instead introduce traces for temporal memory safety:
\begin{definition}{\definitionlabel[\glsfirst{tms} Traces]{tmstraces}}
\[
  \begin{array}{rcll}
    \text{(Base-Events)}&\tmsEvent &:=& \tmsAlloc \mid \tmsDealloc \mid \tmsUse \\
    \text{(Events)}&\tmsEvent &:=& \tmsEvent \mid \emptyevent \mid \lightning \\ 
  \end{array}
\]
\end{definition}

With this, \gls*{tms} is defined as follows.
\begin{definition}{\definitionlabel[\glsfirst{tms}]{tmsdef}}
  \[
  \tmssafe:=\left\{\tmsTrace
    \left| 
      \begin{array}{rcl}
        \tmsAlloc&\le_{\tmsTrace}&\tmsDealloc \\
        \tmsUse&\le_{\tmsTrace}&\tmsDealloc \\
      \end{array}
    \right.
  \right\}
  \]
\end{definition}
Hereby, the notation $\varEvent[_{1}]\le_{\varTrace}\varEvent[_{2}]$ means that if $\varEvent[_{1}]$ is in $\varTrace$ and if $\varEvent[_{2}]$ is in $\varTrace$, then $\varEvent[_{1}]$ appears before $\varEvent[_{2}]$.

\paragraph{Spatial Memory Safety}
\gls*{sms}~\cite{nagarakatte2009soft} is a safety property that prohibits out-of-bounds accesses.
For emphasis, we also use a custom trace-model for spatial memory safety.
\begin{definition}{\definitionlabel[\glsfirst{sms} Traces]{smstraces}}
\[
  \begin{array}{rcll}
    \text{(Base-Events)}&\smsEvent &:=& \smsAlloc \mid \smsUse \\
    \text{(Events)}&\smsEvent &:=& \smsEvent \mid \emptyevent \mid \lightning \\ 
  \end{array}
\]
\end{definition}

With this, \gls*{sms} is defined as follows.
\begin{definition}{\definitionlabel[\glsfirst{sms}]{smsdef}}
  \[
  \smssafe:=\left\{\smsTrace \left|\begin{array}{rcl}
      \text{If }\smsAlloc\le_{\smsTrace}\smsUse, \text{ then }m<n
  \end{array}\right.\right\}
  \]
\end{definition}

\paragraph{Memory Safety}

In spirit of earlier work~\cite{nagarakatte2009soft,nagarakatte2010cets,jim2002cyclone,necula2005ccured,michael2023mswasm}, full \gls*{ms} is the intersection of \definitionref{tmsdef} and \definitionref{smsdef}.
However, both properties use a different trace model, so intersecting them alone would trivially yield the empty set.
To remedy this, we define cross-language trace relations that lift both \definitionref{tmstraces} and \definitionref{smstraces} to \definitionref{mstraces}.
\begin{definition}{\definitionlabel[Mapping \glsfirst{ms} to \glsfirst{tms}]{mapping-tms-ms}}
  \newcommand{\mmap}[2]{{#1}\;{\xlangrel{\mssafe}{\tmssafe}}\;{#2}}
  \begin{center}
    \judgbox{\mmap{\msEvent}{\tmsEvent}}{,,Translate \gls*{ms}-events to \gls*{tms}-events.''}

    \typerule{mapping-ms-tms-empty}{}{
      \mmap{\msev{\emptyevent}}{\tmsev{\emptyevent}}
    }{mapping-ms-tms-empty}
    %
    \typerule{mapping-ms-tms-crash}{}{
      \mmap{\msev{\lightning}}{\tmsev{\lightning}}
    }{mapping-ms-tms-crash}
    %
    \typerule{mapping-ms-tms-alloc}{}{
      \mmap{\msAlloc}{\tmsAlloc}
    }{mapping-ms-tms-alloc}
    %
    \typerule{mapping-ms-tms-dealloc}{}{
      \mmap{\msDealloc}{\tmsDealloc}
    }{mapping-ms-tms-dealloc}
    %
    \typerule{mapping-ms-tms-use}{}{
      \mmap{\msUse}{\tmsUse}
    }{mapping-ms-tms-use}
  \end{center}
  \begin{center}
    \judgbox{\mmap{\msTrace}{\tmsTrace}}{,,Translate \gls*{ms}-traces to \gls*{tms}-traces.''}
    
    \typerule{mapping-ms-tms-refl}{}{
      \mmap{\emptyTrace}{\emptyTrace}
    }{mapping-ms-tms-refl}
    %
    \typerule{mapping-ms-tms-cons}{}{
      \mmap{\consTrace{\msEvent}{\msTrace}}{\consTrace{\tmsEvent}{\tmsTrace}}
    }{mapping-ms-tms-cons}
  \end{center}
\end{definition}
\begin{definition}{\definitionlabel[Mapping \glsfirst{ms} to \glsfirst{sms}]{mapping-sms-ms}}
  \newcommand{\mmap}[2]{{#1}\;{\xlangrel{\mssafe}{\smssafe}}\;{#2}}
  \begin{center}
    \judgbox{\mmap{\msEvent}{\smsEvent}}{,,Translate \gls*{ms}-events to \gls*{sms}-events.''}

    \typerule{mapping-ms-sms-empty}{}{
      \mmap{\msev{\emptyevent}}{\smsev{\emptyevent}}
    }{mapping-ms-sms-empty}
    %
    \typerule{mapping-ms-sms-crash}{}{
      \mmap{\msev{\lightning}}{\smsev{\lightning}}
    }{mapping-ms-sms-crash}
    %
    \typerule{mapping-ms-sms-alloc}{}{
      \mmap{\msAlloc}{\smsAlloc}
    }{mapping-ms-sms-alloc}
    %
    \typerule{mapping-ms-sms-dealloc}{}{
      \mmap{\msDealloc}{\emptyevent}
    }{mapping-ms-sms-dealloc}
    %
    \typerule{mapping-ms-sms-use}{
      n=m
    }{
      \mmap{\msUse}{\smsUse}
    }{mapping-ms-sms-use}
  \end{center}
  \begin{center}
    \judgbox{\mmap{\msTrace}{\smsTrace}}{,,Translate \gls*{ms}-traces to \gls*{sms}-traces.''}
    
    \typerule{mapping-ms-sms-refl}{}{
      \mmap{\emptyTrace}{\emptyTrace}
    }{mapping-ms-sms-refl}
    %
    \typerule{mapping-ms-sms-cons}{}{
      \mmap{\consTrace{\msEvent}{\msTrace}}{\consTrace{\smsEvent}{\smsTrace}}
    }{mapping-ms-sms-cons}
  \end{center}
\end{definition}

\begin{definition}{\definitionlabel[\glsfirst{ms}]{msdef}}
  \[
    \mssafe:=\mapUniversal{\xlangrel{\mssafe}{\tmssafe}}{\tmssafe} \cap \mapUniversal{\xlangrel{\mssafe}{\smssafe}}{\smssafe}
  \]
\end{definition}

Note that \definitionref{msdef} ignores data isolation, so there may still be memory-safety issues introduced by side-channels.

\subsubsection{Strict Cryptographic Constant Time}

\begin{definition}{\definitionlabel[\glsfirst{scct} Traces]{sccttraces}}
\[
  \begin{array}{rcll}
    \text{(Secrecy Tags)}&\varSecuritytag &:=& \unlock \mid \lock \\
    \text{(Events)}&\tmsEvent &:=& \scctAny \mid \emptyevent \mid \lightning \\ 
  \end{array}
\]
\end{definition}

\gls*{cct} is a hypersafety property~\cite{barthe2018sec} and, thus, difficult to check with monitors.
This is because, intuitively, hypersafety properties can relate multiple execution traces with each other, but monitors work on a single execution.
It is a common trick to sidestep this issue by means of overapproximation: this section defines the property \gls*{scct}, a stricter variant of \gls*{cct} (inspired by earlier work~\cite{almeida2017jasmin}) that enforces the policy that no secret appears on a trace.
Programs that satisfy \gls*{scct} also satisfy \gls*{cct}, but programs that satisfy \gls*{cct} may not satisfy \gls*{scct}.

\begin{definition}{\definitionlabel[\glsfirst{scct}]{scctdef}}
  \noindent\[
  \scctsafe:=\left\{\scctTrace 
      \left|
        \begin{array}{l}
          \scctTrace=\hole{\cdot} \text{ or } \scctTrace=\lightning \\
          \exists\scctTrace['],\left(\scctTrace=\scctAny[\unlock]\cdot\varTrace[_{\ctsafe}'] \text{ or } \right.\\
          \hspace{2.8em}\left.\scctTrace=\emptyevent\cdot\scctTrace[']
          \wedge \varTrace[_{\ctsafe}']\in\scctsafe \right)
        \end{array}
      \right.
    \right\}
  \]
\end{definition}

\gls*{scct} may appear overly strict, since it seems that secrets must not occur on a trace (since $\varSecuritytag$ is forced to be $\unlock$). 
However, this is considered standard practice in terms of coding guidelines~\cite{ctguidelines}.
Moreover, programs that have been compiled with FaCT~\cite{cauligi2019fact} and run with a ``data independent timing mode''~\cite{arm-refman,intel-refman} enabled do not leak secrets. 
We'd like to emphasize that our goal is to motivate the core theory presented in \Cref{sec:rtpc} and not develop a new programming language.

\paragraph{\gls*{ms}, Strict Cryptographic Constant Time}

The combination of \gls*{ms} and \gls*{scct} is the intersection of these properties, \gls*{msscct}.
However, \gls*{ms} uses a different trace model than \gls*{scct}, so intersecting them would trivially yield the empty set. 
To remedy this issue, we introduce $\xlangrel{\scctsafe}{\mssafe} : \scctEvent \to \msEvent \to \mathbb{P}$, a cross-language trace relation (whose key cases are presented below), that we use to intuitively unify the trace model in which the two properties are expressed:

\begin{definition}{\definitionlabel[Mapping \glsfirst{ms} to \glsfirst{scct}]{mapping-ms-scct}}
  \newcommand{\mmap}[2]{{#1}\;{\xlangrel{\mssafe}{\scctsafe}}\;{#2}}
  \begin{center}
    \judgbox{\mmap{\msEvent}{\scctEvent}}{,,Translate \gls*{ms}-events to \gls*{scct}-events.''}

    \typerule{mapping-ms-scct-empty}{}{
      \mmap{\msev{\emptyevent}}{\scctev{\emptyevent}}
    }{mapping-ms-scct-empty}
    %
    \typerule{mapping-ms-scct-crash}{}{
      \mmap{\msev{\lightning}}{\scctev{\lightning}}
    }{mapping-ms-scct-crash}
    %
    \typerule{mapping-ms-scct-alloc}{}{
      \mmap{\msAlloc}{\scctAny[\unlock]}
    }{mapping-ms-scct-alloc}
    %
    \typerule{mapping-ms-scct-dealloc}{}{
      \mmap{\msDealloc}{\scctAny[\unlock]}
    }{mapping-ms-scct-dealloc}
    %
    \typerule{mapping-ms-scct-use}{}{
      \mmap{\msUse}{\scctAny[\unlock]}
    }{mapping-ms-scct-use}
  \end{center}
  \begin{center}
    \judgbox{\mmap{\msTrace}{\scctTrace}}{,,Translate \gls*{ms}-traces to \gls*{scct}-traces.''}
    
    \typerule{mapping-ms-scct-refl}{}{
      \mmap{\emptyTrace}{\emptyTrace}
    }{mapping-ms-scct-refl}
    %
    \typerule{mapping-ms-scct-cons}{}{
      \mmap{\consTrace{\msEvent}{\msTrace}}{\consTrace{\scctEvent}{\scctTrace}}
    }{mapping-ms-scct-cons}
  \end{center}
\end{definition}

%Essentially, $\sim_{\ctsafe}$ ignores both the new $\ev{Branch\ n}$ and $\ev{Binop\ n}$ base-events as it relates security-insensitive actions ($\unlock$) to their equivalent counterparts.
\gls*{ms} traces trivially satisfy \gls*{scct}, since it is assumed here that they are the result of a cryptographic constant time computation.
This may be ensured by a compiler, such as FaCT, or the semantics of the language itself.
We also don't have any other information in \gls*{scct} traces besides the secrecy tags, since the semantics of a language can simply tag relevant events with the respective vulnerability, such as a leak of secret data.
% 
It is now possible to define \gls*{msscct} using the universal image:

\begin{definition}{\definitionlabel[\glsfirst{ms} and \glsfirst{scct}]{msscctdef}}
  \[
    \msscctsafe:=\mssafe\cap\mapUniversal{\xlangrel{\mssafe}{\scctsafe}}{\scctsafe}
  \]
\end{definition}

\paragraph{Extending the Trace Model with Speculation}

So far, the considered trace models do not let us express speculative execution attacks such as Spectre~\cite{kocher2019spectre}. 
For this, we extend the earlier trace model so that the secrecy tags ($\varSecuritytag{}$) carry additional information about the kind of private data leakage, i.e., the type of speculative leak.
This tag may be emitted by the semantics of a concrete programming language, as seen in earlier work~\cite{fabian2022automatic}. 
%Moreover, we add base-events signalling the beginning of a speculative execution ($\ev{Spec}$), a barrier ($\ev{Barrier}$) that signals that any speculative execution may not go past it, as well as a rollback event ($\ev{Rlb}$), which signals that execution resumes to where speculation started.

\begin{definition}{\definitionlabel[\glsfirst{ss} traces]{sstraces}}
{
\[
  \begin{array}{rrcl}
    (\text{Spectre Variants}) & vX &:=& \operatorname{NONE} \mid \operatorname{PHT} \mid \dots \\
    (\text{Secrecy Tags}) & \varSecuritytag{} &:=& \lock_{vX} \mid \unlock\\ 
    (\text{Events}) & \specEvent &:=& \specEvent \mid \emptyevent \mid \lightning \\ 
  \end{array}
\]
}
\end{definition}

%Even though the considered Spectre variants are just SPECTRE-PHT~\cite{kocher2019spectre}, NONE just describes secret data as in \gls*{scct} (see \Cref{subsec:scct:tracemodel}), the trace model is general enough to allow for potential future extension with different variants~\cite{kocher2019spectre,maisuradze2018ret2spec,horn2019zero}.
NONE describes secret data as in \gls*{scct} (see \Cref{subsec:scct:tracemodel}).
Any other secrecy tag, such as PHT, describes the concrete reason for the respective leak.
The trace model in \defref{sstraces} is general enough to allow for potential future extension with different variants~\cite{kocher2019spectre,maisuradze2018ret2spec,horn2019zero}.
%For sake of readability, this paper just uses the notation $\lock$ in place of $\lock_{\text{PHT}}$.

\subsubsection{Speculation Safety}

\gls*{ss}~\cite{patrignani2021exorcising}, similar to \gls*{scct}, is a sound overapproximation of a variant of noninterference.

\begin{definition}{\definitionlabel[\glsfirst{ss}]{ssdef}}
  \[
    \sssafe := \left\{
      \specTrace
        \left|
          \begin{array}{l}
            \specTrace=\hole{\cdot} 
              \text{ or } 
            \specTrace=\lightning
              \text{ or }
            \exists\specTrace['].\\
            %
            \left(\specTrace=\specAny[\unlock]\cdot\specTrace['] 
              \text{or }
            \varTrace[_{\mathghost}]=\specAny[\lock_{\text{NONE}}]\cdot\specTrace[']\right.\\
            \hspace{2em}\text{or }
            \left.\varTrace[_{\mathghost}]=\emptyevent\cdot\specTrace[']\right)\\
            %
            \text{and }%
            \ \varTrace[_{\mathghost}']\in\sssafe
          \end{array}
        \right.
      \right\}
  \] 
\end{definition}
The technical setup so far leads to the above definition, where only locks annotated with $\text{SPECTRE-PHT}$ are disallowed to occur on the trace.
That way, programs attaining \gls*{ss} do not necessarily attain \gls*{scct}.

\subsubsection{Speculation Memory Safety}\label{sec:spec-ms-rel}

As before, we need to relate the different trace models with each other, so that the memory safety property without speculation can be lifted to speculation. 
To this end, let $\xlangrel{ms}{ss}: \msEvent \to \specEvent\to\mathbb{P}$ be a cross-language trace relation whose key cases are below.
The intuition is that \gls*{ss} is trivially satisfied in \gls*{ms}, since speculation is inexpressible there and a language attaining \gls*{ms} by means of semantics can enforce \gls*{ss} simply by not having a speculative dynamic semantics. 
In turn, all base events tagged with $\lock_{\text{PHT}}$ are simply dropped. 

\begin{definition}{\definitionlabel[Mapping \glsfirst{ms} to \glsfirst{ss}]{mapping-ms-spec}}
  \newcommand{\mmap}[2]{{#1}\;{\xlangrel{\mssafe}{\sssafe}}\;{#2}}
  \begin{center}
    \judgbox{\mmap{\msEvent}{\specEvent}}{,,Translate \gls*{ms}-events to \gls*{ss}-events.''}

    \typerule{mapping-ms-spec-empty}{}{
      \mmap{\msev{\emptyevent}}{\specev{\emptyevent}}
    }{mapping-ms-spec-empty}
    %
    \typerule{mapping-ms-spec-crash}{}{
      \mmap{\msev{\lightning}}{\specev{\lightning}}
    }{mapping-ms-spec-crash}
    %
    \typerule{mapping-ms-spec-alloc}{}{
      \mmap{\msAlloc}{\specAny[\unlock]}
    }{mapping-ms-spec-alloc}
    %
    \typerule{mapping-ms-spec-dealloc}{}{
      \mmap{\msDealloc}{\specAny[\unlock]}
    }{mapping-ms-spec-dealloc}
    %
    \typerule{mapping-ms-spec-use}{}{
      \mmap{\msUse}{\specAny[\unlock]}
    }{mapping-ms-spec-use}
  \end{center}
  \begin{center}
    \judgbox{\mmap{\msTrace}{\specTrace}}{,,Translate \gls*{ms}-traces to \gls*{ss}-traces.''}
    
    \typerule{mapping-ms-spec-refl}{}{
      \mmap{\emptyTrace}{\emptyTrace}
    }{mapping-ms-spec-refl}
    %
    \typerule{mapping-ms-spec-cons}{}{
      \mmap{\consTrace{\msEvent}{\msTrace}}{\consTrace{\specEvent}{\specTrace}}
    }{mapping-ms-spec-cons}
  \end{center}
\end{definition}

We conclude with the ultimate property of interest for our secure compiler: \gls*{specms}.
\begin{definition}{\definitionlabel[\glsfirst{specms}]{specmsdef}}
  \[
    \specmssafe := \msscctsafe\cap\mapUniversal{\xlangrel{\mssafe}{\sssafe}}{\sssafe}
  \]
\end{definition}
